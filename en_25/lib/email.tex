% Copyright (C) 2001-2006 Python Software Foundation
% Author: barry@python.org (Barry Warsaw)

\section{\module{email} ---
	 An email and MIME handling package}

\declaremodule{standard}{email}
\modulesynopsis{Package supporting the parsing, manipulating, and
    generating email messages, including MIME documents.}
\moduleauthor{Barry A. Warsaw}{barry@python.org}
\sectionauthor{Barry A. Warsaw}{barry@python.org}

\versionadded{2.2}

The \module{email} package is a library for managing email messages,
including MIME and other \rfc{2822}-based message documents.  It
subsumes most of the functionality in several older standard modules
such as \refmodule{rfc822}, \refmodule{mimetools},
\refmodule{multifile}, and other non-standard packages such as
\module{mimecntl}.  It is specifically \emph{not} designed to do any
sending of email messages to SMTP (\rfc{2821}), NNTP, or other servers; those
are functions of modules such as \refmodule{smtplib} and \refmodule{nntplib}.
The \module{email} package attempts to be as RFC-compliant as possible,
supporting in addition to \rfc{2822}, such MIME-related RFCs as
\rfc{2045}, \rfc{2046}, \rfc{2047}, and \rfc{2231}.

The primary distinguishing feature of the \module{email} package is
that it splits the parsing and generating of email messages from the
internal \emph{object model} representation of email.  Applications
using the \module{email} package deal primarily with objects; you can
add sub-objects to messages, remove sub-objects from messages,
completely re-arrange the contents, etc.  There is a separate parser
and a separate generator which handles the transformation from flat
text to the object model, and then back to flat text again.  There
are also handy subclasses for some common MIME object types, and a few
miscellaneous utilities that help with such common tasks as extracting
and parsing message field values, creating RFC-compliant dates, etc.

The following sections describe the functionality of the
\module{email} package.  The ordering follows a progression that
should be common in applications: an email message is read as flat
text from a file or other source, the text is parsed to produce the
object structure of the email message, this structure is manipulated,
and finally, the object tree is rendered back into flat text.

It is perfectly feasible to create the object structure out of whole
cloth --- i.e. completely from scratch.  From there, a similar
progression can be taken as above.

Also included are detailed specifications of all the classes and
modules that the \module{email} package provides, the exception
classes you might encounter while using the \module{email} package,
some auxiliary utilities, and a few examples.  For users of the older
\module{mimelib} package, or previous versions of the \module{email}
package, a section on differences and porting is provided.

\begin{seealso}
    \seemodule{smtplib}{SMTP protocol client}
    \seemodule{nntplib}{NNTP protocol client}
\end{seealso}

\subsection{Representing an email message}
\declaremodule{standard}{email.message}
\modulesynopsis{�Żҥ᡼��Υ�å�������ɽ��������쥯�饹}

\class{Message} ���饹�ϡ� \module{email} �ѥå��������濴�Ȥʤ륯�饹�Ǥ���
����� \module{email} ���֥������ȥ�ǥ�δ��쥯�饹�ˤʤäƤ��ޤ���
\class{Message} �ϥإå��ե�����ɤ򸡺��������å��������Τ˥����������뤿���
�ˤȤʤ뵡ǽ���󶡤��ޤ���

��ǰŪ�ˤϡ�(\module{email.message}�⥸�塼�뤫�饤��ݡ��Ȥ����)
\class{Message} ���֥������Ȥˤ� \emph{�إå�} �� \emph{�ڥ�������} ��
��Ǽ����Ƥ��ޤ����إå��ϡ�\rfc{2822} �����Υե������̾����ӥե�������ͤ�
������Ƕ��ڤ�줿��ΤǤ���������ϥե������̾�ޤ��ϥե�������ͤ�
�ɤ���ˤ�ޤޤ�ޤ���

�إå�����ʸ����ʸ������̤�����������¸����ޤ������إå�̾�����פ��뤫�ɤ����θ�����
��ʸ����ʸ������̤����ˤ����ʤ����Ȥ��Ǥ��ޤ���\emph{Unix-From} �إå��ޤ���
\code{From_} �إå��Ȥ����Τ��륨��٥����ץإå����ҤȤ�¸�ߤ��뤳�Ȥ⤢��ޤ���
�ڥ������ɤϡ�ñ��ʥ�å��������֥������Ȥξ���ñ�ʤ�ʸ����Ǥ�����
MIME ����ƥ�ʸ�� (\mimetype{multipart/*} �ޤ���
\mimetype{message/rfc822} �ʤ�) �ξ��� \class{Message} ���֥������Ȥ�
�ꥹ�ȤˤʤäƤ��ޤ���

\class{Message} ���֥������Ȥϡ���å������إå��˥����������뤿���
�ޥå� (����) �����Υ��󥿥ե������ȡ��إå�����ӥڥ������ɤ�ξ����
�����������뤿�������Ū�ʥ��󥿥ե��������󶡤��ޤ���
����ˤϥ�å��������֥������ȥĥ꡼����ե�åȤʥƥ�����ʸ���
���������ꡢ����Ū�˻Ȥ���إå��Υѥ�᡼���˥������������ꡢ�ޤ�
���֥������ȥĥ꡼��Ƶ�Ū�ˤ��ɤä��ꤹ�뤿��������ʥ᥽�åɤ�ޤߤޤ���

\class{Message} ���饹�Υ᥽�åɤϰʲ��ΤȤ���Ǥ�:

\begin{classdesc}{Message}{}
���󥹥ȥ饯���ϰ�����Ȥ�ޤ���
\end{classdesc}

\begin{methoddesc}[Message]{as_string}{\optional{unixfrom}}
��å��������Τ�ե�åȤ�ʸ����Ȥ����֤��ޤ���
���ץ���� \var{unixfrom} �� \code{True} �ξ�硢�֤����ʸ����ˤ�
����٥����ץإå���ޤޤ�ޤ���\var{unixfrom} �Υǥե���Ȥ� \code{False} �Ǥ���

���Υ᥽�åɤϼ�ڤ����Ѥ�������Ǥ��ޤ�����ɬ����������̤�˥�å�������
�ե����ޥåȤ���Ȥϸ¤�ޤ��󡣤��Ȥ��С�����ϥǥե���ȤǤ� \code{From } ��
�Ϥޤ�Ԥ��ѹ����Ƥ��ޤ��ޤ����ʲ�����Τ褦��  \class{Generator} 
�Υ��󥹥��󥹤��������� \method{flatten()} �᥽�åɤ�ľ�ܸƤӽФ���
������ʽ�����Ԥ������Ǥ��ޤ���

\begin{verbatim}
from cStringIO import StringIO
from email.generator import Generator
fp = StringIO()
g = Generator(fp, mangle_from_=False, maxheaderlen=60)
g.flatten(msg)
text = fp.getvalue()
\end{verbatim}

\end{methoddesc}

\begin{methoddesc}[Message]{__str__}{}
\method{as_string(unixfrom=True)} ��Ʊ���Ǥ���
\end{methoddesc}

\begin{methoddesc}[Message]{is_multipart}{}
��å������Υڥ������ɤ��� \class{Message} ���֥������Ȥ���ʤ�
�ꥹ�ȤǤ���� \code{True} ���֤��������Ǥʤ���� \code{False} ���֤��ޤ���
\method{is_multipart()} �� False ���֤������ϡ��ڥ������ɤ�
ʸ���󥪥֥������ȤǤ���ɬ�פ�����ޤ���
\end{methoddesc}

\begin{methoddesc}[Message]{set_unixfrom}{unixfrom}
��å������Υ���٥����ץإå��� \var{unixfrom} �����ꤷ�ޤ��������ʸ����Ǥ���ɬ�פ�����ޤ���
\end{methoddesc}

\begin{methoddesc}[Message]{get_unixfrom}{}
��å������Υ���٥����ץإå����֤��ޤ���
����٥����ץإå������ꤵ��Ƥ��ʤ����� None ���֤���ޤ���
\end{methoddesc}

\begin{methoddesc}[Message]{attach}{payload}
Ϳ����줿 \var{payload} �򸽺ߤΥڥ������ɤ��ɲä��ޤ���
���λ����ǤΥڥ������ɤ� \code{None} �������뤤�� \class{Message} ���֥������Ȥ�
�ꥹ�ȤǤ���ɬ�פ�����ޤ������Υ᥽�åɤμ¹Ը塢�ڥ������ɤ�ɬ��
\class{Message} ���֥������ȤΥꥹ�Ȥˤʤ�ޤ����ڥ������ɤ�
�����顼���֥������� (ʸ����ʤ�) ���Ǽ���������ϡ�������
\method{set_payload()} ��ȤäƤ���������
\end{methoddesc}

\begin{methoddesc}[Message]{get_payload}{\optional{i\optional{, decode}}}
���ߤΥڥ������ɤؤλ��Ȥ��֤��ޤ�������� \method{is_multipart()} �� \code{True}
�ξ�� \class{Message} ���֥������ȤΥꥹ�Ȥˤʤꡢ\method{is_multipart()} ��
\code{False} �ξ���ʸ����ˤʤ�ޤ����ڥ������ɤ��ꥹ�Ȥξ�硢
�ꥹ�Ȥ��ѹ����뤳�ȤϤ��Υ�å������Υڥ������ɤ��ѹ����뤳�Ȥˤʤ�ޤ���

���ץ��������� \var{i} �������硢
\method{is_multipart()} �� \code{True} �ʤ�� \method{get_payload()} ��
�ڥ���������� 0 ��������� \var{i} ���ܤ����Ǥ��֤��ޤ���\var{i} ��
0 ��꾮������硢���뤤�ϥڥ������ɤθĿ��ʾ�ξ���
\exception{IndexError} ��ȯ�����ޤ����ڥ������ɤ�ʸ����
(�Ĥޤ� \method{is_multipart()} �� \code{False}) �ˤ⤫����餺
\var{i} ��Ϳ����줿�Ȥ��� \exception{TypeError} ��ȯ�����ޤ���

���ץ����� \var{decode} �Ϥ��Υڥ������ɤ�
\mailheader{Content-Transfer-Encoding} �إå��˽��ä�
�ǥ����ɤ����٤����ɤ�����ؼ�����ե饰�Ǥ���
�����ͤ� \code{True} �ǥ�å������� multipart �ǤϤʤ���硢
�ڥ������ɤϤ��Υإå����ͤ� \samp{quoted-printable} �ޤ���
\samp{base64} �ΤȤ��ˤ�����ǥ����ɤ���ޤ�������ʳ��Υ��󥳡��ǥ��󥰤�
�Ȥ��Ƥ����硢\mailheader{Content-Transfer-Encoding} �إå���
�ʤ���硢���뤤��ۣ���base64�ǡ������ޤޤ����ϡ��ڥ������ɤϤ��Τޤ� 
(�ǥ����ɤ��줺��) �֤���ޤ���
�⤷��å������� multipart �� \var{decode} �ե饰�� \code{True} �ξ���
\code{None} ���֤���ޤ���\var{decode} �Υǥե�����ͤ� \code{False} �Ǥ���
\end{methoddesc}

\begin{methoddesc}[Message]{set_payload}{payload\optional{, charset}}
��å��������ΤΥ��֥������ȤΥڥ������ɤ� \var{payload} �����ꤷ�ޤ���
�ڥ������ɤη�����ȤȤΤ���ΤϸƤӽФ�¦����Ǥ�Ǥ���
���ץ����� \var{charset} �ϥ�å������Υǥե����ʸ�����åȤ����ꤷ�ޤ���
�ܤ����� \method{set_charset()} �򻲾Ȥ��Ƥ���������

\versionchanged[\var{charset} �������ɲ�]{2.2.2}
\end{methoddesc}

\begin{methoddesc}[Message]{set_charset}{charset}
�ڥ������ɤ�ʸ�����åȤ� \var{charset} ���ѹ����ޤ���
�����ˤ� \class{Charset}���󥹥��� (\refmodule{email.charset} ����)��
ʸ�����å�̾�򤢤�魯ʸ���󡢤��뤤�� \code{None} �Τ����줫������Ǥ��ޤ���
ʸ�������ꤷ����硢����� \class{Charset} ���󥹥��󥹤��Ѵ�����ޤ���
\var{charset} �� \code{None} �ξ�硢\code{charset} �ѥ�᡼����
\mailheader{Content-Type} �إå���������ޤ���
����ʳ��Τ�Τ�ʸ�����åȤȤ��ƻ��ꤷ����硢
\exception{TypeError} ��ȯ�����ޤ���

�����Ǥ�����å������Ȥϡ�\var{charset.input_charset} �ǥ��󥳡��ɤ��줿
\mimetype{text/*} �����Τ�Τ��ꤷ�Ƥ��ޤ�������ϡ��⤷ɬ�פȤ����
�ץ졼��ƥ����ȷ������Ѵ����뤵���� \var{charset.output_charset} ��
���󥳡��ɤ��Ѵ�����ޤ���MIME �إå� (\mailheader{MIME-Version}, 
\mailheader{Content-Type}, \mailheader{Content-Transfer-Encoding})
��ɬ�פ˱������ɲä���ޤ���

\versionadded{2.2.2}
\end{methoddesc}

\begin{methoddesc}[Message]{get_charset}{}
���Υ�å�������Υڥ������ɤ� \class{Charset} ���󥹥��󥹤�
�֤��ޤ���
\versionadded{2.2.2}
\end{methoddesc}

�ʲ��Υ᥽�åɤϡ���å������� \rfc{2822} �إå��˥����������뤿���
�ޥå� (����) �����Υ��󥿥ե����������������ΤǤ���
�����Υ᥽�åɤȡ��̾�Υޥå� (����) ���Ϥޤä���Ʊ����̣���Ĥ櫓�Ǥ�
�ʤ����Ȥ����դ��Ƥ������������Ȥ��м��񷿤Ǥϡ�Ʊ��������ʣ�����뤳�Ȥ�
������Ƥ��ޤ��󤬡������Ǥ�Ʊ����å������إå���ʣ�������礬����ޤ���
�ޤ������񷿤Ǥ� \method{keys()} ���֤���륭���ν�����ݾڤ���Ƥ��ޤ��󤬡�
\class{Message} ���֥���������Υإå��ϤĤͤ˸��Υ�å��������
���줿��������뤤�Ϥ��Τ��Ȥ��ɲä��줿������֤���ޤ���������졢���θ�
�դ������ɲä��줿�إå��ϥꥹ�Ȥΰ��ֺǸ�˸���ޤ���

�������ä���̣�Τ������ϰտ�Ū�ʤ�Τǡ���������������Ĥ褦�ˤĤ����Ƥ��ޤ���

����: �ɤ�ʾ��⡢��å�������Υ���٥����ץإå���
���Υޥå׷����Υ��󥿥ե������ˤϴޤޤ�ޤ���

\begin{methoddesc}[Message]{__len__}{}
ʣ�����줿��Τ�դ���ƥإå����ι�פ��֤��ޤ���
\end{methoddesc}

\begin{methoddesc}[Message]{__contains__}{name}
��å��������֥������Ȥ� \var{name} �Ȥ���̾���Υե�����ɤ���äƤ���� true ���֤��ޤ���
���θ����Ǥ�̾������ʸ����ʸ���϶��̤���ޤ���\var{name} �ϺǸ�˥������դ���Ǥ��ƤϤ����ޤ���
���Υ᥽�åɤϰʲ��Τ褦�� \code{in} �黻�ҤǻȤ��ޤ�:

\begin{verbatim}
if 'message-id' in myMessage:
    print 'Message-ID:', myMessage['message-id']
\end{verbatim}
\end{methoddesc}

\begin{methoddesc}[Message]{__getitem__}{name}
���ꤵ�줿̾���Υإå��ե�����ɤ��ͤ��֤��ޤ���
\var{name} �ϺǸ�˥������դ���Ǥ��ƤϤ����ޤ���
���Υإå����ʤ����� \code{None} ���֤��졢\exception{KeyError} �㳰��ȯ�����ޤ���

����: ���ꤵ�줿̾���Υե�����ɤ���å������Υإå��� 2��ʾ帽��Ƥ����硢
�ɤ�����ͤ��֤���뤫��̤����Ǥ����إå���¸�ߤ���ե�����ɤ��ͤ򤹤٤�
���Ф��������� \method{get_all()} �᥽�åɤ�ȤäƤ���������
\end{methoddesc}

\begin{methoddesc}[Message]{__setitem__}{name, val}

��å������إå��� \var{name} �Ȥ���̾���� \var{val} �Ȥ����ͤ���
�ե�����ɤ򤢤餿���ɲä��ޤ������Υե�����ɤϸ��ߥ�å�������
¸�ߤ���ե�����ɤΤ����Ф����ɲä���ޤ���

����: ���Υ᥽�åɤǤϡ����Ǥ�Ʊ���̾����¸�ߤ���ե�����ɤ�
���\emph{����ޤ���}���⤷��å�������̾�� \var{name} ����
�ե�����ɤ�ҤȤĤ��������ʤ��褦�ˤ�������С��ǽ�ˤ�������Ƥ���������
���Ȥ���:

\begin{verbatim}
del msg['subject']
msg['subject'] = 'PythonPythonPython!'
\end{verbatim}
\end{methoddesc}

\begin{methoddesc}[Message]{__delitem__}{name}
��å������Υإå����顢 \var{name} �Ȥ���̾������
�ե�����ɤ򤹤٤ƽ���ޤ������Ȥ�����̾�����ĥإå���
¸�ߤ��Ƥ��ʤ��Ƥ��㳰��ȯ�����ޤ���
\end{methoddesc}

\begin{methoddesc}[Message]{has_key}{name}
��å������� \var{name} �Ȥ���̾������
�إå��ե�����ɤ���äƤ���п��򡢤����Ǥʤ���е����֤��ޤ���
\end{methoddesc}

\begin{methoddesc}[Message]{keys}{}
��å�������ˤ��뤹�٤ƤΥإå��Υե������̾�Υꥹ�Ȥ��֤��ޤ���
\end{methoddesc}

\begin{methoddesc}[Message]{values}{}
��å�������ˤ��뤹�٤ƤΥե�����ɤ��ͤΥꥹ�Ȥ��֤��ޤ���
\end{methoddesc}

\begin{methoddesc}[Message]{items}{}
��å�������ˤ��뤹�٤ƤΥإå��Υե������̾�Ȥ����ͤ�
2-���ץ�Υꥹ�ȤȤ����֤��ޤ���
\end{methoddesc}

\begin{methoddesc}[Message]{get}{name\optional{, failobj}}
���ꤵ�줿̾�����ĥե�����ɤ��ͤ��֤��ޤ���
����ϻ��ꤵ�줿̾�����ʤ��Ȥ��˥��ץ��������� \var{failobj} 
(�ǥե���ȤǤ� \code{None}) ���֤����Ȥ�Τ����С�\method{__getitem__()} ��Ʊ���Ǥ���
\end{methoddesc}

���Ω�ĥ᥽�åɤ򤤤��Ĥ��Ҳ𤷤ޤ�:

\begin{methoddesc}[Message]{get_all}{name\optional{, failobj}}
\var{name} ��̾�����ĥե�����ɤΤ��٤Ƥ��ͤ���ʤ�ꥹ�Ȥ��֤��ޤ���
��������̾���Υإå�����å�������˴ޤޤ�Ƥ��ʤ����� \var{failobj} 
(�ǥե���ȤǤ� \code{None}) ���֤���ޤ���
\end{methoddesc}

\begin{methoddesc}[Message]{add_header}{_name, _value, **_params}
��ĥ�إå����ꡣ���Υ᥽�åɤ� \method{__setitem__()} �Ȼ��Ƥ��ޤ�����
�ɲäΥإå����ѥ�᡼���򥭡���ɰ����ǻ���Ǥ���Ȥ�������äƤ��ޤ���
\var{_name} ���ɲä���إå��ե�����ɤ�\var{_value} �ˤ��Υإå���
\emph{�ǽ��}�ͤ��Ϥ��ޤ���

������ɰ������� \var{_params} �γƹ��ܤ��Ȥˡ�
���Υ������ѥ�᡼��̾�Ȥ��ư���졢����̾�ˤդ��ޤ��
��������������ϥϥ��ե���ִ�����ޤ� (�ʤ��ʤ�ϥ��ե��
�̾�� Python ���̻ҤȤ��ƤϻȤ��ʤ�����Ǥ�)���դĤ���
�ѥ�᡼�����ͤ� \code{None} �ʳ��ΤȤ��ϡ�\code{key="value"} ��
�����ɲä���ޤ����ѥ�᡼�����ͤ� \code{None} �ΤȤ��ϥ����Τߤ��ɲä���ޤ���

��򼨤��ޤ��礦:

\begin{verbatim}
msg.add_header('Content-Disposition', 'attachment', filename='bud.gif')
\end{verbatim}

��������ȥإå��ˤϰʲ��Τ褦���ɲä���ޤ���

\begin{verbatim}
Content-Disposition: attachment; filename="bud.gif"
\end{verbatim}
\end{methoddesc}

\begin{methoddesc}[Message]{replace_header}{_name, _value}
�إå����ִ���\var{_name} �Ȱ��פ���إå��Ǻǽ�˸��Ĥ��ä���Τ��֤������ޤ���
���ΤȤ��إå��ν���ȥե������̾����ʸ����ʸ������¸����ޤ���
���פ���إå����ʤ���硢 \exception{KeyError} ��ȯ�����ޤ���

\versionadded{2.2.2}
\end{methoddesc}

\begin{methoddesc}[Message]{get_content_type}{}
���Υ�å������� content-type ���֤��ޤ���
�֤��줿ʸ����϶���Ū�˾�ʸ���� \mimetype{maintype/subtype} �η������Ѵ�����ޤ���
��å�������� \mailheader{Content-Type} �إå����ʤ���硢�ǥե���Ȥ�
content-type �� \method{get_default_type()} ���֤��ͤˤ�ä�Ϳ�����ޤ���
\rfc{2045} �ˤ��Х�å������ϤĤͤ˥ǥե���Ȥ� content-type ��
��äƤ���Τǡ�\method{get_content_type()} �ϤĤͤˤʤ�餫���ͤ��֤��Ϥ��Ǥ���

\rfc{2045} �ϥ�å������Υǥե���� content-type ��
���줬 \mimetype{multipart/digest} ����ƥʤ˸���Ƥ���Ȥ��ʳ���
\mimetype{text/plain} �˵��ꤷ�Ƥ��ޤ��������å�������
\mimetype{multipart/digest} ����ƥ���ˤ����硢����
content-type �� \mimetype{message/rfc822} �ˤʤ�ޤ���
�⤷ \mailheader{Content-Type} �إå���Ŭ�ڤǤʤ� content-type �񼰤��ä���硢
\rfc{2045} �Ϥ���Υǥե���Ȥ� \mimetype{text/plain} �Ȥ��ư����褦
���Ƥ��ޤ���

\versionadded{2.2.2}
\end{methoddesc}

\begin{methoddesc}[Message]{get_content_maintype}{}
���Υ�å������μ� content-type ���֤��ޤ���
����� \method{get_content_type()} �ˤ�ä�
�֤����ʸ����� \mimetype{maintype} ��ʬ�Ǥ���

\versionadded{2.2.2}
\end{methoddesc}

\begin{methoddesc}[Message]{get_content_subtype}{}
���Υ�å��������� content-type (sub content-type��subtype) ���֤��ޤ���
����� \method{get_content_type()} �ˤ�ä�
�֤����ʸ����� \mimetype{subtype} ��ʬ�Ǥ���

\versionadded{2.2.2}
\end{methoddesc}

\begin{methoddesc}[Message]{get_default_type}{}
�ǥե���Ȥ� content-type ���֤��ޤ���
�ۤɤ�ɤΥ�å������Ǥϥǥե���Ȥ� content-type ��
\mimetype{text/plain} �Ǥ�������å������� \mimetype{multipart/digest} ����ƥʤ�
�ޤޤ�Ƥ���Ȥ������㳰Ū�� \mimetype{message/rfc822} �ˤʤ�ޤ���

\versionadded{2.2.2}
\end{methoddesc}

\begin{methoddesc}[Message]{set_default_type}{ctype}
�ǥե���Ȥ� content-type �����ꤷ�ޤ���
\var{ctype} �� \mimetype{text/plain} ���뤤�� \mimetype{message/rfc822}
�Ǥ���ɬ�פ�����ޤ����������ǤϤ���ޤ���
�ǥե���Ȥ� content-type �ϥإå��� \mailheader{Content-Type} �ˤ�
��Ǽ����ޤ���

\versionadded{2.2.2}
\end{methoddesc}

\begin{methoddesc}[Message]{get_params}{\optional{failobj\optional{,
    header\optional{, unquote}}}}
��å������� \mailheader{Content-Type} �ѥ�᡼����ꥹ�ȤȤ����֤��ޤ���
�֤����ꥹ�Ȥ� ����/�ͤ��Ȥ���ʤ� 2���ǥ��ץ뤬Ϣ�ʤä���ΤǤ��ꡢ
������ \character{=} �����ʬΥ����Ƥ��ޤ���\character{=} �κ�¦��
�����ˤʤꡢ��¦���ͤˤʤ�ޤ����ѥ�᡼����� \character{=} ���ʤ��ä���硢
�ͤ���ʬ�϶�ʸ����ˤʤꡢ�����Ǥʤ���Ф����ͤ� \method{get_param()} ��
��������Ƥ�������ˤʤ�ޤ����ޤ������ץ������� \var{unquote} ��
\code{True} (�ǥե����) �Ǥ����硢�����ͤ� unquote ����ޤ���

���ץ������� \var{failobj} �ϡ�\mailheader{Content-Type} �إå���
¸�ߤ��ʤ��ä������֤����֥������ȤǤ������ץ������� \var{header} �ˤ�
\mailheader{Content-Type} �Τ����˸������٤��إå�����ꤷ�ޤ���

\versionchanged[\var{unquote} ���ɲä���ޤ���]{2.2.2}
\end{methoddesc}

\begin{methoddesc}[Message]{get_param}{param\optional{,
    failobj\optional{, header\optional{, unquote}}}}
��å������� \mailheader{Content-Type} �إå���Υѥ�᡼�� \var{param} ��
ʸ����Ȥ����֤��ޤ������Υ�å�������� \mailheader{Content-Type} �إå���
¸�ߤ��ʤ��ä���硢 \var{failobj}  (�ǥե���Ȥ� \code{None}) ���֤���ޤ���

���ץ������� \var{header} ��Ϳ����줿��硢
\mailheader{Content-Type} �Τ����ˤ��Υإå������Ѥ���ޤ���

�ѥ�᡼���Υ�����ӤϾ����ʸ����ʸ������̤��ޤ���
�֤��ͤ�ʸ���� 3 ���ǤΥ��ץ�ǡ����ץ�ˤʤ�Τϥѥ�᡼���� \rfc{2231} 
���󥳡��ɤ���Ƥ�����Ǥ���3 ���ǥ��ץ�ξ�硢�����Ǥ��ͤ�
\code{(CHARSET, LANGUAGE, VALUE)} �η����ˤʤäƤ��ޤ���
\code{CHARSET} �� \code{LAGUAGE} �� \code{None} �ˤʤ뤳�Ȥ����ꡢ���ξ��
\code{VALUE} �� \code{us-ascii} ʸ�����åȤǥ��󥳡��ɤ���Ƥ���Ȥߤʤ��ͤ�
�ʤ�ʤ��Τ����դ��Ƥ������������ʤ� \code{LANGUAGE} ��̵��Ǥ��ޤ���

���δؿ���Ȥ����ץꥱ������󤬡��ѥ�᡼���� \rfc{2231} ������
���󥳡��ɤ���Ƥ��뤫�ɤ����򵤤ˤ��ʤ��ΤǤ���С�\function{email.Utils.collapse_rfc2231_value()} ��
\method{get_param()} ���֤��ͤ��Ϥ��ƸƤӽФ����Ȥǡ����Υѥ�᡼����ҤȤĤˤޤȤ�뤳�Ȥ��Ǥ��ޤ���
�����ͤ����ץ�ʤ�Ф��δؿ���Ŭ�ڤ˥ǥ����ɤ��줿 Unicode ʸ������֤���
�����Ǥʤ����� unquote ���줿����ʸ������֤��ޤ������Ȥ���:

\begin{verbatim}
rawparam = msg.get_param('foo')
param = email.Utils.collapse_rfc2231_value(rawparam)
\end{verbatim}

������ξ���ѥ�᡼�����ͤ� (ʸ����Ǥ��� 3���ǥ��ץ��
\code{VALUE} ���ܤǤ���) �Ĥͤ� unquote ����ޤ���
��������\var{unquote} �� \code{False} �˻��ꤵ��Ƥ������
unquote ����ޤ���

\versionchanged[\var{unquote} �������ɲá�3���ǥ��ץ뤬�֤��ͤˤʤ��ǽ������]{2.2.2}
\end{methoddesc}

\begin{methoddesc}[Message]{set_param}{param, value\optional{,
    header\optional{, requote\optional{, charset\optional{, language}}}}}

\mailheader{Content-Type} �إå���Υѥ�᡼�������ꤷ�ޤ���
���ꤵ�줿�ѥ�᡼�����إå���ˤ��Ǥ�¸�ߤ����硢�����ͤ�
\var{value} ���֤��������ޤ���\mailheader{Content-Type} �إå����ޤ�
���Υ�å��������¸�ߤ��Ƥ��ʤ���硢\rfc{2045} �ˤ������������ͤˤ�
\mimetype{text/plain} �����ꤵ�졢�������ѥ�᡼���ͤ��������ɲä���ޤ���

���ץ������� \var{header} ��Ϳ����줿��硢
\mailheader{Content-Type} �Τ����ˤ��Υإå������Ѥ���ޤ���
���ץ������� \var{unquote} �� \code{False} �Ǥʤ��¤ꡢ
�����ͤ� unquote ����ޤ� (�ǥե���Ȥ� \code{True})��

���ץ������� \var{charset} ��Ϳ������ȡ�
���Υѥ�᡼���� \rfc{2231} �˽��äƥ��󥳡��ɤ���ޤ���
���ץ������� \var{language} �� RFC 2231 �θ������ꤷ�ޤ�����
�ǥե���ȤǤϤ���϶�ʸ����Ȥʤ�ޤ��� \var{charset} ��
\var{language} �Ϥɤ����ʸ����Ǥ���ɬ�פ�����ޤ���

\versionadded{2.2.2}
\end{methoddesc}

\begin{methoddesc}[Message]{del_param}{param\optional{, header\optional{,
    requote}}}
���ꤵ�줿�ѥ�᡼���� \mailheader{Content-Type} �إå��椫�鴰����
�Ȥ�Τ����ޤ����إå��Ϥ��Υѥ�᡼�����ͤ��ʤ����֤˽񤭴������ޤ���
\var{requote} �� \code{False} �Ǥʤ��¤� (�ǥե���ȤǤ� \code{True} �Ǥ�)��
���٤Ƥ��ͤ�ɬ�פ˱����� quote ����ޤ������ץ�����ѿ� \var{header} ��Ϳ����줿��硢
\mailheader{Content-Type} �Τ����ˤ��Υإå������Ѥ���ޤ���

\versionadded{2.2.2}
\end{methoddesc}

\begin{methoddesc}[Message]{set_type}{type\optional{, header}\optional{,
    requote}}
\mailheader{Content-Type} �إå��� maintype �� subtype �����ꤷ�ޤ���
\var{type} �� \mimetype{maintype/subtype} �Ȥ�������ʸ����Ǥʤ���Фʤ�ޤ���
����ʳ��ξ��� \exception{ValueError} ��ȯ�����ޤ���

���Υ᥽�åɤ� \mailheader{Content-Type} �إå����֤������ޤ�����
���٤ƤΥѥ�᡼���Ϥ��Τޤޤˤ��ޤ���\var{requote} �� \code{False} �ξ�硢
����Ϥ��Ǥ�¸�ߤ���إå��� quote �������֤��ޤ����������Ǥʤ�����
��ưŪ�� quote ���ޤ� (�ǥե����ư��)��

���ץ�����ѿ� \var{header} ��Ϳ����줿��硢
\mailheader{Content-Type} �Τ����ˤ��Υإå������Ѥ���ޤ���
\mailheader{Content-Type} �إå������ꤵ�����ˤϡ�
\mailheader{MIME-Version} �إå���Ʊ�����ղä���ޤ���

\versionadded{2.2.2}
\end{methoddesc}

\begin{methoddesc}[Message]{get_filename}{\optional{failobj}}
���Υ�å�������� \mailheader{Content-Disposition} �إå��ˤ��롢
\code{filename} �ѥ�᡼�����ͤ��֤��ޤ�����Ū�Υإå���
\code{filename} �ѥ�᡼�����ʤ����ˤ� \code{name}�ѥ�᡼����õ����
���������̵�����ޤ��ϥإå���̵�����ˤ� \var{failobj} ���֤���ޤ���
�֤����ʸ����ϤĤͤ� \method{Utils.unquote()} �ˤ�ä� unquote ����ޤ���

\end{methoddesc}

\begin{methoddesc}[Message]{get_boundary}{\optional{failobj}}
���Υ�å�������� \mailheader{Content-Type} �إå��ˤ��롢
\code{boundary} �ѥ�᡼�����ͤ��֤��ޤ�����Ū�Υإå����礱�Ƥ����ꡢ
\code{boundary} �ѥ�᡼�����ʤ����ˤ� \var{failobj} ���֤���ޤ���
�֤����ʸ����ϤĤͤ� \method{Utils.unquote()} �ˤ�ä� unquote ����ޤ���
\end{methoddesc}

\begin{methoddesc}[Message]{set_boundary}{boundary}
��å�������� \mailheader{Content-Type} �إå��ˤ��롢
\code{boundary} �ѥ�᡼�����ͤ����ꤷ�ޤ���\method{set_boundary()} ��
ɬ�פ˱����� \var{boundary} �� quote ���ޤ������Υ�å�������
\mailheader{Content-Type} �إå���ޤ�Ǥ��ʤ���硢
\exception{HeaderParseError} ��ȯ�����ޤ���

����: ���Υ᥽�åɤ�Ȥ��Τϡ��Ť� \mailheader{Content-Type} �إå���
������ƿ����� boundary ���ä��إå��� \method{add_header()} ��
­���ΤȤϾ����㤤�ޤ���\method{set_boundary()} ��
��Ϣ�Υإå���Ǥ� \mailheader{Content-Type} �إå��ΰ��֤��ݤĤ���Ǥ���
������������ϸ��� \mailheader{Content-Type} �إå����¸�ߤ��Ƥ���
Ϣ³����Ԥν��֤ޤǤ� \emph{�ݤ��ޤ���}��
\end{methoddesc}

\begin{methoddesc}[Message]{get_content_charset}{\optional{failobj}}
���Υ�å�������� \mailheader{Content-Type} �إå��ˤ��롢
\code{charset} �ѥ�᡼�����ͤ��֤��ޤ����ͤϤ��٤ƾ�ʸ�����Ѵ�����ޤ���
��å�������� \mailheader{Content-Type} ���ʤ��ä��ꡢ���Υإå����
\code{boundary} �ѥ�᡼�����ʤ����ˤ� \var{failobj} ���֤���ޤ���

����: ����� \method{get_charset()} �᥽�åɤȤϰۤʤ�ޤ���
������Τۤ���ʸ����Τ����ˡ����Υ�å������ܥǥ��Υǥե����
���󥳡��ǥ��󥰤� \class{Charset} ���󥹥��󥹤��֤��ޤ���

\versionadded{2.2.2}
\end{methoddesc}

\begin{methoddesc}[Message]{get_charsets}{\optional{failobj}}
��å�������˴ޤޤ��ʸ�����åȤ�̾���򤹤٤ƥꥹ�Ȥˤ����֤��ޤ���
���Υ�å������� \mimetype{multipart} �Ǥ����硢�֤����ꥹ�Ȥ�
�����Ǥ����줾��� subpart �Υڥ������ɤ��б����ޤ�������ʳ��ξ�硢
�����Ĺ�� 1 �Υꥹ�Ȥ��֤��ޤ���

�ꥹ����γ����Ǥ�ʸ����Ǥ��ꡢ������б����� subpart ���
���줾��� \mailheader{Content-Type} �إå��ˤ��� \code{charset} ���ͤǤ���
������������ subpart �� \mailheader{Content-Type} ���äƤʤ�����
\code{charset} ���ʤ��������뤤�� MIME maintype �� \mimetype{text} �Ǥʤ�
�����줫�ξ��ˤϡ��ꥹ�Ȥ����ǤȤ��� \var{failobj} ���֤���ޤ���
\end{methoddesc}

\begin{methoddesc}[Message]{walk}{}
\method{walk()} �᥽�åɤ�¿��Ū�Υ����ͥ졼���ǡ�
����Ϥ����å��������֥������ȥĥ꡼��Τ��٤Ƥ� part ����� subpart ��
�錄���⤯�Τ˻Ȥ��ޤ�������Ͽ���ͥ��Ǥ��������餯ŵ��Ū����ˡ�ϡ�
\method{walk()} �� \code{for} �롼����ǤΥ��ƥ졼���Ȥ���
�Ȥ����ȤǤ��礦���롼�פ���ޤ�뤴�Ȥˡ����� subpart ���֤����ΤǤ���

�ʲ�����ϡ� multipart ��å������Τ��٤Ƥ� part �ˤ����ơ�
���� MIME �����פ�ɽ�����Ƥ�����ΤǤ���

\begin{verbatim}
>>> for part in msg.walk():
...     print part.get_content_type()
multipart/report
text/plain
message/delivery-status
text/plain
text/plain
message/rfc822
\end{verbatim}
\end{methoddesc}

\versionchanged[��������侩�᥽�å� \method{get_type()}��
\method{get_main_type()}��\method{get_subtype()} �Ϻ������ޤ�����]{2.5}

\class{Message} ���֥������Ȥϥ��ץ����Ȥ��� 2�ĤΥ��󥹥���°����
�Ȥ뤳�Ȥ��Ǥ��ޤ�������Ϥ��� MIME ��å���������ץ졼��ƥ����Ȥ�
��������Τ˻Ȥ����Ȥ��Ǥ��ޤ���

\begin{datadesc}{preamble}
MIME �ɥ�����Ȥη����Ǥϡ�
�إå�ľ��ˤ�����ԤȺǽ�� multipart �����򤢤�魯ʸ����Τ�������
�����餫�Υƥ����� (����: preamble, ��ʸ) ����ᤳ�ळ�Ȥ�����Ƥ��ޤ���
���Υƥ����Ȥ�ɸ��Ū�� MIME �����Ƥ���Ϥ߽Ф��Ƥ���Τǡ�
MIME ������ǧ������᡼�륽�եȤ��餳�����̾�ޤä��������ޤ���
��������å������Υƥ����Ȥ����Ǹ����硢���뤤�ϥ�å�������
MIME �б����Ƥ��ʤ��᡼�륽�եȤǸ����硢���Υƥ����Ȥ�
�ܤ˸����뤳�Ȥˤʤ�ޤ���

\var{preamble} °���� MIME �ɥ�����Ȥ˲ä���
���κǽ�� MIME �ϰϳ��ƥ����Ȥ�ޤ�Ǥ��ޤ���
\class{Parser} ������ƥ����Ȥ�إå��ʹߤ�ȯ����������
����Ϥޤ��ǽ�� MIME ����ʸ���󤬸���������ä���硢
�ѡ����Ϥ��Υƥ����Ȥ��å������� \var{preamble} °���˳�Ǽ���ޤ���
\class{Generator} ������ MIME ��å���������ץ졼��ƥ����ȷ�����
��������Ȥ�������Ϥ��Υƥ����Ȥ�إå��Ⱥǽ�� MIME �����δ֤��������ޤ���
�ܺ٤� \refmodule{email.parser} ����� \refmodule{email.Generator} ��
���Ȥ��Ƥ���������

����: ���Υ�å������� preamble ���ʤ���硢
\var{preamble} °���ˤ� \code{None} ����Ǽ����ޤ���
\end{datadesc}

\begin{datadesc}{epilogue}
\var{epilogue} °���ϥ�å������κǸ�� MIME ����ʸ���󤫤�
��å����������ޤǤΥƥ����Ȥ�ޤ��Τǡ�����ʳ��� \var{preamble} °����Ʊ���Ǥ���

\versionchanged[\class{Generator}�ǥե����뽪ü�˲��Ԥ���Ϥ��뤿�ᡢ
epilogue �˶�ʸ��������ꤹ��ɬ�פϤʤ��ʤ�ޤ�����]{2.5}
\end{datadesc}

\begin{datadesc}{defects}
\var{defects} °���ϥ�å���������Ϥ�������Ǹ��Ф��줿���٤Ƥ������� (defect���㳲) ��
�ꥹ�Ȥ��ݻ����Ƥ��ޤ����������ȯ�����줦��㳲�ˤĤ��ƤΤ��ܺ٤�������
\refmodule{email.errors} �򻲾Ȥ��Ƥ���������
 
\versionadded{2.4}
\end{datadesc}


\subsection{Parsing email messages}
\declaremodule{standard}{email.parser}
\modulesynopsis{Parse flat text email messages to produce a message
	        object structure.}

Message object structures can be created in one of two ways: they can be
created from whole cloth by instantiating \class{Message} objects and
stringing them together via \method{attach()} and
\method{set_payload()} calls, or they can be created by parsing a flat text
representation of the email message.

The \module{email} package provides a standard parser that understands
most email document structures, including MIME documents.  You can
pass the parser a string or a file object, and the parser will return
to you the root \class{Message} instance of the object structure.  For
simple, non-MIME messages the payload of this root object will likely
be a string containing the text of the message.  For MIME
messages, the root object will return \code{True} from its
\method{is_multipart()} method, and the subparts can be accessed via
the \method{get_payload()} and \method{walk()} methods.

There are actually two parser interfaces available for use, the classic
\class{Parser} API and the incremental \class{FeedParser} API.  The classic
\class{Parser} API is fine if you have the entire text of the message in
memory as a string, or if the entire message lives in a file on the file
system.  \class{FeedParser} is more appropriate for when you're reading the
message from a stream which might block waiting for more input (e.g. reading
an email message from a socket).  The \class{FeedParser} can consume and parse
the message incrementally, and only returns the root object when you close the
parser\footnote{As of email package version 3.0, introduced in
Python 2.4, the classic \class{Parser} was re-implemented in terms of the
\class{FeedParser}, so the semantics and results are identical between the two
parsers.}.

Note that the parser can be extended in limited ways, and of course
you can implement your own parser completely from scratch.  There is
no magical connection between the \module{email} package's bundled
parser and the \class{Message} class, so your custom parser can create
message object trees any way it finds necessary.

\subsubsection{FeedParser API}

\versionadded{2.4}

The \class{FeedParser}, imported from the \module{email.feedparser} module,
provides an API that is conducive to incremental parsing of email messages,
such as would be necessary when reading the text of an email message from a
source that can block (e.g. a socket).  The
\class{FeedParser} can of course be used to parse an email message fully
contained in a string or a file, but the classic \class{Parser} API may be
more convenient for such use cases.  The semantics and results of the two
parser APIs are identical.

The \class{FeedParser}'s API is simple; you create an instance, feed it a
bunch of text until there's no more to feed it, then close the parser to
retrieve the root message object.  The \class{FeedParser} is extremely
accurate when parsing standards-compliant messages, and it does a very good
job of parsing non-compliant messages, providing information about how a
message was deemed broken.  It will populate a message object's \var{defects}
attribute with a list of any problems it found in a message.  See the
\refmodule{email.errors} module for the list of defects that it can find.

Here is the API for the \class{FeedParser}:

\begin{classdesc}{FeedParser}{\optional{_factory}}
Create a \class{FeedParser} instance.  Optional \var{_factory} is a
no-argument callable that will be called whenever a new message object is
needed.  It defaults to the \class{email.message.Message} class.
\end{classdesc}

\begin{methoddesc}[FeedParser]{feed}{data}
Feed the \class{FeedParser} some more data.  \var{data} should be a
string containing one or more lines.  The lines can be partial and the
\class{FeedParser} will stitch such partial lines together properly.  The
lines in the string can have any of the common three line endings, carriage
return, newline, or carriage return and newline (they can even be mixed).
\end{methoddesc}

\begin{methoddesc}[FeedParser]{close}{}
Closing a \class{FeedParser} completes the parsing of all previously fed data,
and returns the root message object.  It is undefined what happens if you feed
more data to a closed \class{FeedParser}.
\end{methoddesc}

\subsubsection{Parser class API}

The \class{Parser} class, imported from the \module{email.parser} module,
provides an API that can be used to parse a message when the complete contents
of the message are available in a string or file.  The
\module{email.parser} module also provides a second class, called
\class{HeaderParser} which can be used if you're only interested in
the headers of the message. \class{HeaderParser} can be much faster in
these situations, since it does not attempt to parse the message body,
instead setting the payload to the raw body as a string.
\class{HeaderParser} has the same API as the \class{Parser} class.

\begin{classdesc}{Parser}{\optional{_class}}
The constructor for the \class{Parser} class takes an optional
argument \var{_class}.  This must be a callable factory (such as a
function or a class), and it is used whenever a sub-message object
needs to be created.  It defaults to \class{Message} (see
\refmodule{email.message}).  The factory will be called without
arguments.

The optional \var{strict} flag is ignored.  \deprecated{2.4}{Because the
\class{Parser} class is a backward compatible API wrapper around the
new-in-Python 2.4 \class{FeedParser}, \emph{all} parsing is effectively
non-strict.  You should simply stop passing a \var{strict} flag to the
\class{Parser} constructor.}

\versionchanged[The \var{strict} flag was added]{2.2.2}
\versionchanged[The \var{strict} flag was deprecated]{2.4}
\end{classdesc}

The other public \class{Parser} methods are:

\begin{methoddesc}[Parser]{parse}{fp\optional{, headersonly}}
Read all the data from the file-like object \var{fp}, parse the
resulting text, and return the root message object.  \var{fp} must
support both the \method{readline()} and the \method{read()} methods
on file-like objects.

The text contained in \var{fp} must be formatted as a block of \rfc{2822}
style headers and header continuation lines, optionally preceded by a
envelope header.  The header block is terminated either by the
end of the data or by a blank line.  Following the header block is the
body of the message (which may contain MIME-encoded subparts).

Optional \var{headersonly} is as with the \method{parse()} method.

\versionchanged[The \var{headersonly} flag was added]{2.2.2}
\end{methoddesc}

\begin{methoddesc}[Parser]{parsestr}{text\optional{, headersonly}}
Similar to the \method{parse()} method, except it takes a string
object instead of a file-like object.  Calling this method on a string
is exactly equivalent to wrapping \var{text} in a \class{StringIO}
instance first and calling \method{parse()}.

Optional \var{headersonly} is a flag specifying whether to stop
parsing after reading the headers or not.  The default is \code{False},
meaning it parses the entire contents of the file.

\versionchanged[The \var{headersonly} flag was added]{2.2.2}
\end{methoddesc}

Since creating a message object structure from a string or a file
object is such a common task, two functions are provided as a
convenience.  They are available in the top-level \module{email}
package namespace.

\begin{funcdesc}{message_from_string}{s\optional{, _class\optional{, strict}}}
Return a message object structure from a string.  This is exactly
equivalent to \code{Parser().parsestr(s)}.  Optional \var{_class} and
\var{strict} are interpreted as with the \class{Parser} class constructor.

\versionchanged[The \var{strict} flag was added]{2.2.2}
\end{funcdesc}

\begin{funcdesc}{message_from_file}{fp\optional{, _class\optional{, strict}}}
Return a message object structure tree from an open file object.  This
is exactly equivalent to \code{Parser().parse(fp)}.  Optional
\var{_class} and \var{strict} are interpreted as with the
\class{Parser} class constructor.

\versionchanged[The \var{strict} flag was added]{2.2.2}
\end{funcdesc}

Here's an example of how you might use this at an interactive Python
prompt:

\begin{verbatim}
>>> import email
>>> msg = email.message_from_string(myString)
\end{verbatim}

\subsubsection{Additional notes}

Here are some notes on the parsing semantics:

\begin{itemize}
\item Most non-\mimetype{multipart} type messages are parsed as a single
      message object with a string payload.  These objects will return
      \code{False} for \method{is_multipart()}.  Their
      \method{get_payload()} method will return a string object.

\item All \mimetype{multipart} type messages will be parsed as a
      container message object with a list of sub-message objects for
      their payload.  The outer container message will return
      \code{True} for \method{is_multipart()} and their
      \method{get_payload()} method will return the list of
      \class{Message} subparts.

\item Most messages with a content type of \mimetype{message/*}
      (e.g. \mimetype{message/delivery-status} and
      \mimetype{message/rfc822}) will also be parsed as container
      object containing a list payload of length 1.  Their
      \method{is_multipart()} method will return \code{True}.  The
      single element in the list payload will be a sub-message object.

\item Some non-standards compliant messages may not be internally consistent
      about their \mimetype{multipart}-edness.  Such messages may have a
      \mailheader{Content-Type} header of type \mimetype{multipart}, but their
      \method{is_multipart()} method may return \code{False}.  If such
      messages were parsed with the \class{FeedParser}, they will have an
      instance of the \class{MultipartInvariantViolationDefect} class in their
      \var{defects} attribute list.  See \refmodule{email.errors} for
      details.
\end{itemize}


\subsection{Generating MIME documents}
\declaremodule{standard}{email.generator}
\modulesynopsis{Generate flat text email messages from a message structure.}

One of the most common tasks is to generate the flat text of the email
message represented by a message object structure.  You will need to do
this if you want to send your message via the \refmodule{smtplib}
module or the \refmodule{nntplib} module, or print the message on the
console.  Taking a message object structure and producing a flat text
document is the job of the \class{Generator} class.

Again, as with the \refmodule{email.parser} module, you aren't limited
to the functionality of the bundled generator; you could write one
from scratch yourself.  However the bundled generator knows how to
generate most email in a standards-compliant way, should handle MIME
and non-MIME email messages just fine, and is designed so that the
transformation from flat text, to a message structure via the
\class{Parser} class, and back to flat text, is idempotent (the input
is identical to the output).

Here are the public methods of the \class{Generator} class, imported from the
\module{email.generator} module:

\begin{classdesc}{Generator}{outfp\optional{, mangle_from_\optional{,
    maxheaderlen}}}
The constructor for the \class{Generator} class takes a file-like
object called \var{outfp} for an argument.  \var{outfp} must support
the \method{write()} method and be usable as the output file in a
Python extended print statement.

Optional \var{mangle_from_} is a flag that, when \code{True}, puts a
\samp{>} character in front of any line in the body that starts exactly as
\samp{From }, i.e. \code{From} followed by a space at the beginning of the
line.  This is the only guaranteed portable way to avoid having such
lines be mistaken for a \UNIX{} mailbox format envelope header separator (see
\ulink{WHY THE CONTENT-LENGTH FORMAT IS BAD}
{http://home.netscape.com/eng/mozilla/2.0/relnotes/demo/content-length.html}
for details).  \var{mangle_from_} defaults to \code{True}, but you
might want to set this to \code{False} if you are not writing \UNIX{}
mailbox format files.

Optional \var{maxheaderlen} specifies the longest length for a
non-continued header.  When a header line is longer than
\var{maxheaderlen} (in characters, with tabs expanded to 8 spaces),
the header will be split as defined in the \module{email.header.Header}
class.  Set to zero to disable header wrapping.  The default is 78, as
recommended (but not required) by \rfc{2822}.
\end{classdesc}

The other public \class{Generator} methods are:

\begin{methoddesc}[Generator]{flatten}{msg\optional{, unixfrom}}
Print the textual representation of the message object structure rooted at
\var{msg} to the output file specified when the \class{Generator}
instance was created.  Subparts are visited depth-first and the
resulting text will be properly MIME encoded.

Optional \var{unixfrom} is a flag that forces the printing of the
envelope header delimiter before the first \rfc{2822} header of the
root message object.  If the root object has no envelope header, a
standard one is crafted.  By default, this is set to \code{False} to
inhibit the printing of the envelope delimiter.

Note that for subparts, no envelope header is ever printed.

\versionadded{2.2.2}
\end{methoddesc}

\begin{methoddesc}[Generator]{clone}{fp}
Return an independent clone of this \class{Generator} instance with
the exact same options.

\versionadded{2.2.2}
\end{methoddesc}

\begin{methoddesc}[Generator]{write}{s}
Write the string \var{s} to the underlying file object,
i.e. \var{outfp} passed to \class{Generator}'s constructor.  This
provides just enough file-like API for \class{Generator} instances to
be used in extended print statements.
\end{methoddesc}

As a convenience, see the methods \method{Message.as_string()} and
\code{str(aMessage)}, a.k.a. \method{Message.__str__()}, which
simplify the generation of a formatted string representation of a
message object.  For more detail, see \refmodule{email.message}.

The \module{email.generator} module also provides a derived class,
called \class{DecodedGenerator} which is like the \class{Generator}
base class, except that non-\mimetype{text} parts are substituted with
a format string representing the part.

\begin{classdesc}{DecodedGenerator}{outfp\optional{, mangle_from_\optional{,
    maxheaderlen\optional{, fmt}}}}

This class, derived from \class{Generator} walks through all the
subparts of a message.  If the subpart is of main type
\mimetype{text}, then it prints the decoded payload of the subpart.
Optional \var{_mangle_from_} and \var{maxheaderlen} are as with the
\class{Generator} base class.

If the subpart is not of main type \mimetype{text}, optional \var{fmt}
is a format string that is used instead of the message payload.
\var{fmt} is expanded with the following keywords, \samp{\%(keyword)s}
format:

\begin{itemize}
\item \code{type} -- Full MIME type of the non-\mimetype{text} part

\item \code{maintype} -- Main MIME type of the non-\mimetype{text} part

\item \code{subtype} -- Sub-MIME type of the non-\mimetype{text} part

\item \code{filename} -- Filename of the non-\mimetype{text} part

\item \code{description} -- Description associated with the
      non-\mimetype{text} part

\item \code{encoding} -- Content transfer encoding of the
      non-\mimetype{text} part

\end{itemize}

The default value for \var{fmt} is \code{None}, meaning

\begin{verbatim}
[Non-text (%(type)s) part of message omitted, filename %(filename)s]
\end{verbatim}

\versionadded{2.2.2}
\end{classdesc}

\versionchanged[The previously deprecated method \method{__call__()} was
removed]{2.5}


\subsection{Creating email and MIME objects from scratch}
\declaremodule{standard}{email.mime}
\declaremodule{standard}{email.mime.base}
\declaremodule{standard}{email.mime.nonmultipart}
\declaremodule{standard}{email.mime.multipart}
\declaremodule{standard}{email.mime.audio}
\declaremodule{standard}{email.mime.image}
\declaremodule{standard}{email.mime.message}
\declaremodule{standard}{email.mime.text}

�դĤ�����å��������֥������ȹ�¤�ϥե�����ޤ��ϲ���������
�ƥ����Ȥ�ѡ������̤����Ȥ������ޤ����ѡ�����Ϳ����줿
�ƥ����Ȥ���Ϥ�������Ȥʤ� root �Υ�å��������֥������Ȥ��֤��ޤ���
�������������ʥ�å��������֥������ȹ�¤�򲿤�ʤ��Ȥ�������������뤳�Ȥ�
�ޤ���ǽ�Ǥ������̤� \class{Message} ���Ǻ������뤳�Ȥ����Ǥ��ޤ���
�ºݤˤϡ����Ǥ�¸�ߤ����å��������֥������ȹ�¤��ȤäƤ��ơ�
�����˿����� \class{Message} ���֥������Ȥ��ɲä����ꡢ�����Τ�
�̤ΤȤ����ذ�ư��������Ǥ��ޤ�������� MIME ��å�������
�ڤä��ꤪ�������ꤹ�뤿������������ʥ��󥿡��ե��������󶡤��ޤ���

��������å��������֥������ȹ�¤�� \class{Message} ���󥹥��󥹤�
�������뤳�Ȥˤ����ޤ���������ź�եե�����䤽��¾Ŭ�ڤʤ�Τ�
���٤Ƽ�Dzä��Ƥ��Ф褤�ΤǤ���MIME ��å������ξ�硢
\module{email} �ѥå������Ϥ������ñ�ˤ����ʤ���褦�ˤ��뤿���
�����Ĥ��������ʥ��֥��饹���󶡤��Ƥ��ޤ���

�ʲ������Υ��֥��饹�Ǥ�:

\begin{classdesc}{MIMEBase}{_maintype, _subtype, **_params}
Module: \module{email.mime.base}

����Ϥ��٤Ƥ� MIME �ѥ��֥��饹�δ���Ȥʤ륯�饹�Ǥ���
�Ȥ��� \class{MIMEBase} �Υ��󥹥��󥹤�ľ�ܺ������뤳�Ȥ� 
(��ǽ�ǤϤ���ޤ���) �դĤ��Ϥ��ʤ��Ǥ��礦��\class{MIMEBase} ��
ñ�ˤ���ò����줿 MIME �ѥ��֥��饹�Τ�����ص�Ū�ʴ��쥯�饹�Ȥ����󶡤���Ƥ��ޤ���

\var{_maintype} �� \mailheader{Content-Type} �μ���� (maintype) �Ǥ���
(\mimetype{text} �� \mimetype{image} �ʤ�)��\var{_subtype} ��
\mailheader{Content-Type} �������� (subtype) �Ǥ�
(\mimetype{plain} �� \mimetype{gif} �ʤ�)��
\var{_params} �ϳƥѥ�᡼���Υ������ͤ��Ǽ��������Ǥ��ꡢ
�����ľ�� \method{Message.add_header()} ���Ϥ���ޤ���

\class{MIMEBase} ���饹�ϤĤͤ�
(\var{_maintype}�� \var{_subtype}�� ����� \var{_params} �ˤ�ȤŤ���)
\mailheader{Content-Type} �إå��ȡ�
\mailheader{MIME-Version} �إå� (ɬ�� \code{1.0} �����ꤵ���) ���ɲä��ޤ���
\end{classdesc}

\begin{classdesc}{MIMENonMultipart}{}
Module: \module{email.mime.nonmultipart}

\class{MIMEBase} �Υ��֥��饹�ǡ������ \mimetype{multipart} �����Ǥʤ�
MIME ��å������Τ�������Ū�ʴ��쥯�饹�Ǥ������Υ��饹�Τ������Ū�ϡ�
�̾� \mimetype{multipart} �����Υ�å��������Ф��ƤΤ߰�̣��ʤ�
\method{attach()} �᥽�åɤλ��Ѥ�դ������ȤǤ����⤷ \method{attach()} �᥽�åɤ�
�ƤФ줿��硢����� \exception{MultipartConversionError} �㳰��ȯ�����ޤ���

\versionadded{2.2.2}
\end{classdesc}

\begin{classdesc}{MIMEMultipart}{\optional{subtype\optional{,
    boundary\optional{, _subparts\optional{, _params}}}}}
Module: \module{email.mime.multipart}

\class{MIMEBase} �Υ��֥��饹�ǡ������ \mimetype{multipart} ������
MIME ��å������Τ�������Ū�ʴ��쥯�饹�Ǥ������ץ������� \var{_subtype} ��
�ǥե���ȤǤ� \mimetype{mixed} �ˤʤäƤ��ޤ��������Υ�å������������� (subtype) ��
���ꤹ��Τ˻Ȥ����Ȥ��Ǥ��ޤ�����å��������֥������Ȥˤ�
\mimetype{multipart/}\var{_subtype} �Ȥ����ͤ���
\mailheader{Content-Type} �إå��ȤȤ�ˡ�
\mailheader{MIME-Version} �إå����ɲä����Ǥ��礦��

���ץ������� \var{boundary} �� multipart �ζ���ʸ����Ǥ���
���줬 \code{None} �ξ�� (�ǥե����)��������ɬ�פ˱����Ʒ׻�����ޤ���

\var{_subparts} �Ϥ��Υڥ������ɤ� subpart �ν���ͤ���ʤ륷�����󥹤Ǥ���
���Υ������󥹤ϥꥹ�Ȥ��Ѵ��Ǥ���褦�ˤʤäƤ���ɬ�פ�����ޤ���
������ subpart �ϤĤͤ� \method{Message.attach()} �᥽�åɤ�Ȥä�
���Υ�å��������ɲäǤ���褦�ˤʤäƤ��ޤ���

\mailheader{Content-Type} �إå����Ф����ɲäΥѥ�᡼����
������ɰ��� \var{_params} ��𤷤Ƽ������뤤�����ꤵ��ޤ���
����ϥ�����ɼ���ˤʤäƤ��ޤ���

\versionadded{2.2.2}
\end{classdesc}

\begin{classdesc}{MIMEApplication}{_data\optional{, _subtype\optional{,
    _encoder\optional{, **_params}}}}
Module: \module{email.mime.application}

\class{MIMENonMultipart}�Υ��֥��饹�Ǥ��� \class{MIMEApplication} ��
�饹�� MIME ��å��������֥������ȤΥ᥸�㡼������
\mimetype{application} ��ɽ���ޤ���\var{_data}�����ΥХ��������ä�ʸ
����Ǥ������ץ������� \var{_subtype}�� MIME�Υ��֥����פ����ꤷ�ޤ���
���֥����פΥǥե���Ȥ� \mimetype{octet-stream} �Ǥ���

���ץ���������\var{_encoder}�ϸƤӽФ���ǽ�ʥ��֥�������(�ؿ��ʤ�)�ǡ�
�ǡ�����ž���˻Ȥ��ºݤΥ��󥳡��ɽ�����Ԥ��ޤ���
���θƤӽФ���ǽ�ʥ��֥������Ȥϰ�����1�ļ�ꡢ�����
\class{MIMEApplication}�Υ��󥹥��󥹤Ǥ���
�ڥ������ɤ򥨥󥳡��ɤ��줿�������ѹ����뤿���\method{get_payload()}
��\method{set_payload()}��Ȥ���
ɬ�פ˱�����\mailheader{Content-Transfer-Encoding}�䤽��¾�Υإå�����
���������֥������Ȥ��ɲä���٤��Ǥ����ǥե���ȤΥ��󥳡��ɤ�base64��
�����Ȥ߹��ߤΥ��󥳡����ΰ����� \refmodule{email.encoders} �⥸�塼��
�򸫤Ƥ���������

\var{_params} �� ���쥯�饹�Υ��󥹥ȥ饯���ˤ��Τޤ��Ϥ���ޤ���
\versionadded{2.5}
\end{classdesc}



\begin{classdesc}{MIMEAudio}{_audiodata\optional{, _subtype\optional{,
    _encoder\optional{, **_params}}}}
Module: \module{email.mime.audio}

\class{MIMEAudio} ���饹�� \class{MIMENonMultipart} �Υ��֥��饹�ǡ�
����� (maintype) �� \mimetype{audio} �� MIME ���֥������Ȥ��������Τ˻Ȥ��ޤ���
\var{_audiodata} �ϼºݤβ����ǡ������Ǽ����ʸ����Ǥ���
�⤷���Υǡ�����ɸ��� Python �⥸�塼�� \refmodule{sndhdr} �ˤ�ä�
ǧ���Ǥ����ΤǤ���С�\mailheader{Content-Type} �إå���
������ (subtype) �ϼ�ưŪ�˷��ꤵ��ޤ���
�����Ǥʤ����Ϥ��β����η��� (subtype) �� \var{_subtype} ��
����Ū�˻��ꤹ��ɬ�פ�����ޤ�������������ưŪ�˷���Ǥ�����
\var{_subtype} �λ����ʤ����ϡ�\exception{TypeError} ��ȯ�����ޤ���

���ץ������� \var{_encoder} �ϸƤӽФ���ǽ�ʥ��֥������� (�ؿ��ʤ�) �ǡ�
�ȥ�󥹥ݡ��ȤΤ����˲����μºݤΥ��󥳡��ɤ򤪤��ʤ��ޤ���
���Υ��֥������Ȥ� \class{MIMEAudio} ���󥹥��󥹤ΰ�����ҤȤĤ�����뤳�Ȥ��Ǥ��ޤ���
���δؿ��ϡ�Ϳ����줿�ڥ������ɤ򥨥󥳡��ɤ��줿�������Ѵ�����Τ�
\method{get_payload()} ����� \method{set_payload()} ��Ȥ�ɬ�פ�����ޤ���
�ޤ��������ɬ�פ˱����� \mailheader{Content-Transfer-Encoding} ���뤤��
���Υ�å�������Ŭ�������餫�Υإå����ɲä���ɬ�פ�����ޤ���
�ǥե���ȤΥ��󥳡��ǥ��󥰤� base64 �Ǥ����Ȥ߹��ߤΥ��󥳡����ξܺ٤ˤĤ��Ƥ�
\refmodule{email.encoders} �򻲾Ȥ��Ƥ���������

\var{_params} �� \class{MIMEBase} ���󥹥ȥ饯����ľ���Ϥ���ޤ���
\end{classdesc}

\begin{classdesc}{MIMEImage}{_imagedata\optional{, _subtype\optional{,
    _encoder\optional{, **_params}}}}
Module: \module{email.mime.image}

\class{MIMEImage} ���饹�� \class{MIMENonMultipart} �Υ��֥��饹�ǡ�
����� (maintype) �� \mimetype{image} �� MIME ���֥������Ȥ��������Τ˻Ȥ��ޤ���
\var{_imagedata} �ϼºݤβ����ǡ������Ǽ����ʸ����Ǥ��� 
�⤷���Υǡ�����ɸ��� Python �⥸�塼�� \refmodule{imghdr} �ˤ�ä�
ǧ���Ǥ����ΤǤ���С�\mailheader{Content-Type} �إå���
������ (subtype) �ϼ�ưŪ�˷��ꤵ��ޤ���
�����Ǥʤ����Ϥ��β����η��� (subtype) �� \var{_subtype} ��
����Ū�˻��ꤹ��ɬ�פ�����ޤ�������������ưŪ�˷���Ǥ�����
\var{_subtype} �λ����ʤ����ϡ�\exception{TypeError} ��ȯ�����ޤ���

���ץ������� \var{_encoder} �ϸƤӽФ���ǽ�ʥ��֥������� (�ؿ��ʤ�) �ǡ�
�ȥ�󥹥ݡ��ȤΤ����˲����μºݤΥ��󥳡��ɤ򤪤��ʤ��ޤ���
���Υ��֥������Ȥ� \class{MIMEImage} ���󥹥��󥹤ΰ�����ҤȤĤ�����뤳�Ȥ��Ǥ��ޤ���
���δؿ��ϡ�Ϳ����줿�ڥ������ɤ򥨥󥳡��ɤ��줿�������Ѵ�����Τ�
\method{get_payload()} ����� \method{set_payload()} ��Ȥ�ɬ�פ�����ޤ���
�ޤ��������ɬ�פ˱����� \mailheader{Content-Transfer-Encoding} ���뤤��
���Υ�å�������Ŭ�������餫�Υإå����ɲä���ɬ�פ�����ޤ���
�ǥե���ȤΥ��󥳡��ǥ��󥰤� base64 �Ǥ����Ȥ߹��ߤΥ��󥳡����ξܺ٤ˤĤ��Ƥ�
\refmodule{email.encoders} �򻲾Ȥ��Ƥ���������

\var{_params} �� \class{MIMEBase} ���󥹥ȥ饯����ľ���Ϥ���ޤ���
\end{classdesc}

\begin{classdesc}{MIMEMessage}{_msg\optional{, _subtype}}
Module: \module{email.mime.message}

\class{MIMEMessage} ���饹�� \class{MIMENonMultipart} �Υ��֥��饹�ǡ�
����� (maintype) �� \mimetype{message} ��
MIME ���֥������Ȥ��������Τ˻Ȥ��ޤ����ڥ������ɤȤ��ƻȤ����å�������
\var{_msg} �ˤʤ�ޤ�������� \class{Message} ���饹 (���뤤�Ϥ��Υ��֥��饹) ��
���󥹥��󥹤Ǥʤ���Ф����ޤ��󡣤����Ǥʤ���硢���δؿ���
\exception{TypeError} ��ȯ�����ޤ���

���ץ������� \var{_subtype} �Ϥ��Υ�å������������� (subtype) �����ꤷ�ޤ���
�ǥե���ȤǤϤ���� \mimetype{rfc822} �ˤʤäƤ��ޤ���
\end{classdesc}

\begin{classdesc}{MIMEText}{_text\optional{, _subtype\optional{, _charset}}}
Module: \module{email.mime.text}

\class{MIMEText} ���饹�� \class{MIMENonMultipart} �Υ��֥��饹�ǡ�
����� (maintype) �� \mimetype{text} ��
MIME ���֥������Ȥ��������Τ˻Ȥ��ޤ����ڥ������ɤ�ʸ�����
\var{_text} �ˤʤ�ޤ���\var{_subtype} �ˤ������� (subtype) ����ꤷ��
�ǥե���Ȥ� \mimetype{plain} �Ǥ���\var{_charset} �ϥƥ����Ȥ�
ʸ�����åȤǡ�\class{MIMENonMultipart} ���󥹥ȥ饯���˰����Ȥ����Ϥ���ޤ���
�ǥե���ȤǤϤ����ͤ� \code{us-ascii} �ˤʤäƤ��ޤ���
�ƥ����ȥǡ������Ф��Ƥ�ʸ�������ɤο���䥨�󥳡��ɤϤޤä����Ԥ��ޤ���

\versionchanged[�������侩����ʤ������Ǥ��ä� \var{_encoding} ��ű���ޤ�����
���󥳡��ǥ��󥰤� \var{_charset} �������Ȥˤ��ư��ۤΤ����˷��ꤵ��ޤ���]{2.4}
\end{classdesc}


\subsection{Internationalized headers}
\declaremodule{standard}{email.header}
\modulesynopsis{Representing non-ASCII headers}

\rfc{2822} is the base standard that describes the format of email
messages.  It derives from the older \rfc{822} standard which came
into widespread use at a time when most email was composed of \ASCII{}
characters only.  \rfc{2822} is a specification written assuming email
contains only 7-bit \ASCII{} characters.

Of course, as email has been deployed worldwide, it has become
internationalized, such that language specific character sets can now
be used in email messages.  The base standard still requires email
messages to be transferred using only 7-bit \ASCII{} characters, so a
slew of RFCs have been written describing how to encode email
containing non-\ASCII{} characters into \rfc{2822}-compliant format.
These RFCs include \rfc{2045}, \rfc{2046}, \rfc{2047}, and \rfc{2231}.
The \module{email} package supports these standards in its
\module{email.header} and \module{email.charset} modules.

If you want to include non-\ASCII{} characters in your email headers,
say in the \mailheader{Subject} or \mailheader{To} fields, you should
use the \class{Header} class and assign the field in the
\class{Message} object to an instance of \class{Header} instead of
using a string for the header value.  Import the \class{Header} class from the
\module{email.header} module.  For example:

\begin{verbatim}
>>> from email.message import Message
>>> from email.header import Header
>>> msg = Message()
>>> h = Header('p\xf6stal', 'iso-8859-1')
>>> msg['Subject'] = h
>>> print msg.as_string()
Subject: =?iso-8859-1?q?p=F6stal?=


\end{verbatim}

Notice here how we wanted the \mailheader{Subject} field to contain a
non-\ASCII{} character?  We did this by creating a \class{Header}
instance and passing in the character set that the byte string was
encoded in.  When the subsequent \class{Message} instance was
flattened, the \mailheader{Subject} field was properly \rfc{2047}
encoded.  MIME-aware mail readers would show this header using the
embedded ISO-8859-1 character.

\versionadded{2.2.2}

Here is the \class{Header} class description:

\begin{classdesc}{Header}{\optional{s\optional{, charset\optional{,
    maxlinelen\optional{, header_name\optional{, continuation_ws\optional{,
    errors}}}}}}}
Create a MIME-compliant header that can contain strings in different
character sets.

Optional \var{s} is the initial header value.  If \code{None} (the
default), the initial header value is not set.  You can later append
to the header with \method{append()} method calls.  \var{s} may be a
byte string or a Unicode string, but see the \method{append()}
documentation for semantics.

Optional \var{charset} serves two purposes: it has the same meaning as
the \var{charset} argument to the \method{append()} method.  It also
sets the default character set for all subsequent \method{append()}
calls that omit the \var{charset} argument.  If \var{charset} is not
provided in the constructor (the default), the \code{us-ascii}
character set is used both as \var{s}'s initial charset and as the
default for subsequent \method{append()} calls.

The maximum line length can be specified explicit via
\var{maxlinelen}.  For splitting the first line to a shorter value (to
account for the field header which isn't included in \var{s},
e.g. \mailheader{Subject}) pass in the name of the field in
\var{header_name}.  The default \var{maxlinelen} is 76, and the
default value for \var{header_name} is \code{None}, meaning it is not
taken into account for the first line of a long, split header.

Optional \var{continuation_ws} must be \rfc{2822}-compliant folding
whitespace, and is usually either a space or a hard tab character.
This character will be prepended to continuation lines.
\end{classdesc}

Optional \var{errors} is passed straight through to the
\method{append()} method.

\begin{methoddesc}[Header]{append}{s\optional{, charset\optional{, errors}}}
Append the string \var{s} to the MIME header.

Optional \var{charset}, if given, should be a \class{Charset} instance
(see \refmodule{email.charset}) or the name of a character set, which
will be converted to a \class{Charset} instance.  A value of
\code{None} (the default) means that the \var{charset} given in the
constructor is used.

\var{s} may be a byte string or a Unicode string.  If it is a byte
string (i.e. \code{isinstance(s, str)} is true), then
\var{charset} is the encoding of that byte string, and a
\exception{UnicodeError} will be raised if the string cannot be
decoded with that character set.

If \var{s} is a Unicode string, then \var{charset} is a hint
specifying the character set of the characters in the string.  In this
case, when producing an \rfc{2822}-compliant header using \rfc{2047}
rules, the Unicode string will be encoded using the following charsets
in order: \code{us-ascii}, the \var{charset} hint, \code{utf-8}.  The
first character set to not provoke a \exception{UnicodeError} is used.

Optional \var{errors} is passed through to any \function{unicode()} or
\function{ustr.encode()} call, and defaults to ``strict''.
\end{methoddesc}

\begin{methoddesc}[Header]{encode}{\optional{splitchars}}
Encode a message header into an RFC-compliant format, possibly
wrapping long lines and encapsulating non-\ASCII{} parts in base64 or
quoted-printable encodings.  Optional \var{splitchars} is a string
containing characters to split long ASCII lines on, in rough support
of \rfc{2822}'s \emph{highest level syntactic breaks}.  This doesn't
affect \rfc{2047} encoded lines.
\end{methoddesc}

The \class{Header} class also provides a number of methods to support
standard operators and built-in functions.

\begin{methoddesc}[Header]{__str__}{}
A synonym for \method{Header.encode()}.  Useful for
\code{str(aHeader)}.
\end{methoddesc}

\begin{methoddesc}[Header]{__unicode__}{}
A helper for the built-in \function{unicode()} function.  Returns the
header as a Unicode string.
\end{methoddesc}

\begin{methoddesc}[Header]{__eq__}{other}
This method allows you to compare two \class{Header} instances for equality.
\end{methoddesc}

\begin{methoddesc}[Header]{__ne__}{other}
This method allows you to compare two \class{Header} instances for inequality.
\end{methoddesc}

The \module{email.header} module also provides the following
convenient functions.

\begin{funcdesc}{decode_header}{header}
Decode a message header value without converting the character set.
The header value is in \var{header}.

This function returns a list of \code{(decoded_string, charset)} pairs
containing each of the decoded parts of the header.  \var{charset} is
\code{None} for non-encoded parts of the header, otherwise a lower
case string containing the name of the character set specified in the
encoded string.

Here's an example:

\begin{verbatim}
>>> from email.header import decode_header
>>> decode_header('=?iso-8859-1?q?p=F6stal?=')
[('p\xf6stal', 'iso-8859-1')]
\end{verbatim}
\end{funcdesc}

\begin{funcdesc}{make_header}{decoded_seq\optional{, maxlinelen\optional{,
    header_name\optional{, continuation_ws}}}}
Create a \class{Header} instance from a sequence of pairs as returned
by \function{decode_header()}.

\function{decode_header()} takes a header value string and returns a
sequence of pairs of the format \code{(decoded_string, charset)} where
\var{charset} is the name of the character set.

This function takes one of those sequence of pairs and returns a
\class{Header} instance.  Optional \var{maxlinelen},
\var{header_name}, and \var{continuation_ws} are as in the
\class{Header} constructor.
\end{funcdesc}


\subsection{Representing character sets}
\declaremodule{standard}{email.charset}
\modulesynopsis{ʸ�����å�}

���Υ⥸�塼���ʸ�����åȤ�ɽ������ \class{Charset} ���饹��
�Żҥ᡼���å������ˤդ��ޤ��ʸ�����åȴ֤��Ѵ��������
ʸ�����åȤΥ쥸���ȥ�Ȥ��Υ쥸���ȥ�����뤿���
�����Ĥ����ص�Ū�ʥ᥽�åɤ��󶡤��ޤ���\class{Charset} ���󥹥��󥹤�
\module{email} �ѥå�������ˤ���ۤ��Τ����Ĥ��Υ⥸�塼��ǻ��Ѥ���ޤ���

���Υ��饹�� \module{email.charset} �⥸�塼�뤫��import���Ƥ���������

\versionadded{2.2.2}

\begin{classdesc}{Charset}{\optional{input_charset}}
ʸ�����åȤ� email �Υץ��ѥƥ��˼������롣
Map character sets to their email properties.

���Υ��饹�Ϥ��������ʸ�����åȤ��Ф����Żҥ᡼��˲ݤ��������ξ�����󶡤��ޤ���
�ޤ���Ϳ����줿Ŭ�Ѳ�ǽ�� codec ��Ĥ��äơ�ʸ�����åȴ֤��Ѵ��򤪤��ʤ�
�ص�Ū�ʥ롼������󶡤��ޤ����ޤ�����ϡ�����ʸ�����åȤ�Ϳ����줿�Ȥ��ˡ�
����ʸ�����åȤ��Żҥ᡼���å������Τʤ���
�ɤ���ä� RFC �˽�򤷤�������ǻ��Ѥ��뤫�˴ؤ��롢
�Ǥ����뤫����ξ�����󶡤��ޤ���

ʸ�����åȤˤ�äƤϡ�������ʸ�����Żҥ᡼��Υإå����뤤�ϥ�å��������ΤǻȤ�����
quoted-printable �������뤤�� base64�����ǥ��󥳡��ɤ���ɬ�פ�����ޤ���
�ޤ�����ʸ�����åȤϤभ�����Τޤ��Ѵ�����ɬ�פ����ꡢ�Żҥ᡼�����Ǥ�
���ѤǤ��ޤ���

�ʲ��Ǥϥ��ץ������� \var{input_charset} �ˤĤ����������ޤ���
�����ͤϤĤͤ˾�ʸ���˶���Ū���Ѵ�����ޤ���
������ʸ�����åȤ���̾�����������줿���ȡ������ͤ�ʸ�����åȤ�
�쥸���ȥ���򸡺������إå��Υ��󥳡��ǥ��󥰤�
��å��������ΤΥ��󥳡��ǥ��󥰡�����ӽ��ϻ����Ѵ��˻Ȥ��� codec ��ߤĤ���Τ˻Ȥ��ޤ���
���Ȥ��� \var{input_charset} �� \code{iso-8859-1} �ξ�硢�إå�����ӥ�å��������Τ�
quoted-printable �ǥ��󥳡��ɤ��졢���ϻ����Ѵ��� codec ��ɬ�פ���ޤ���
�⤷ \var{input_charset} �� \code{euc-jp} �ʤ�С��إå��� base64 �ǥ��󥳡��ɤ��졢
��å��������Τϥ��󥳡��ɤ���ޤ��󤬡����Ϥ����ƥ����Ȥ� \code{euc-jp} ʸ�����åȤ���
\code{iso-2022-jp} ʸ�����åȤ��Ѵ�����ޤ���
\end{classdesc}

\class{Charset} ���󥹥��󥹤ϰʲ��Τ褦�ʥǡ���°�����äƤ��ޤ�:

\begin{datadesc}{input_charset}
�ǽ�˻��ꤵ���ʸ�����åȤǤ���
���̤����Ѥ��Ƥ�����̾�ϡ�\emph{������} �Żҥ᡼���Ѥ�̾�����Ѵ�����ޤ�
(���Ȥ��С�\code{latin_1} �� \code{iso-8859-1} ���Ѵ�����ޤ�)��
�ǥե���Ȥ� 7-bit �� \code{us-ascii} �Ǥ���
\end{datadesc}

\begin{datadesc}{header_encoding}
����ʸ�����åȤ��Żҥ᡼��إå��˻Ȥ������˥��󥳡��ɤ����ɬ�פ������硢
����°���� \code{Charset.QP} (quoted-printable ���󥳡��ǥ���)��
\code{Charset.BASE64} (base64 ���󥳡��ǥ���)�����뤤��
��û�� QP �ޤ��� BASE64 ���󥳡��ǥ��󥰤Ǥ��� \code{Charset.SHORTEST} ��
���ꤵ��ޤ��������Ǥʤ���硢�����ͤ� \code{None} �ˤʤ�ޤ���
\end{datadesc}

\begin{datadesc}{body_encoding}
\var{header_encoding} ��Ʊ���Ǥ����������ͤϥ�å��������ΤΤ����
���󥳡��ǥ��󥰤򵭽Ҥ��ޤ�������ϥإå��ѤΥ��󥳡��ǥ��󥰤Ȥ�
�㤦���⤷��ޤ���\var{body_encoding} �Ǥϡ�\code{Charset.SHORTEST} ��
�Ȥ����ȤϤǤ��ޤ���
\end{datadesc}

\begin{datadesc}{output_charset}
ʸ�����åȤˤ�äƤϡ��Żҥ᡼��Υإå����뤤�ϥ�å��������Τ�
�Ȥ����ˤ�����Ѵ�����ɬ�פ�����ޤ����⤷ \var{input_charset} ��
������ʸ�����åȤΤɤ줫�򤵤��Ƥ����顢���� \var{output_charset} °����
���줬���ϻ����Ѵ������ʸ�����åȤ�̾���򤢤�路�Ƥ��ޤ���
����ʳ��ξ�硢�����ͤ� \code{None} �ˤʤ�ޤ���
\end{datadesc}

\begin{datadesc}{input_codec}
\var{input_charset} �� Unicode ���Ѵ����뤿��� Python �� codec ̾�Ǥ���
�Ѵ��Ѥ� codec ��ɬ�פʤ��Ȥ��ϡ������ͤ� \code{None} �ˤʤ�ޤ���
\end{datadesc}

\begin{datadesc}{output_codec}
Unicode �� \var{output_charset} ���Ѵ����뤿��� Python �� codec ̾�Ǥ���
�Ѵ��Ѥ� codec ��ɬ�פʤ��Ȥ��ϡ������ͤ� \code{None} �ˤʤ�ޤ���
����°���� \var{input_codec} ��Ʊ���ͤ��Ĥ��Ȥˤʤ�Ǥ��礦��
\end{datadesc}

\class{Charset} ���󥹥��󥹤ϡ��ʲ��Υ᥽�åɤ���äƤ��ޤ�:

\begin{methoddesc}[Charset]{get_body_encoding}{}
��å��������ΤΥ��󥳡��ɤ˻Ȥ���
content-transfer-encoding ���ͤ��֤��ޤ���

�����ͤϻ��Ѥ��Ƥ��륨�󥳡��ǥ��󥰤�ʸ���� \samp{quoted-printable} �ޤ��� \samp{base64} ����
���뤤�ϴؿ��Τɤ��餫�Ǥ�����Ԥξ�硢����ϥ��󥳡��ɤ���� Message ���֥������Ȥ�
ñ��ΰ����Ȥ��Ƽ��褦�ʴؿ��Ǥ���ɬ�פ�����ޤ������δؿ����Ѵ���
\mailheader{Content-Transfer-Encoding} �إå����Τ򡢤ʤ�Ǥ���Ŭ�ڤ��ͤ����ꤹ��ɬ�פ�����ޤ���

���Υ᥽�åɤ� \var{body_encoding} �� \code{QP} �ξ��
\samp{quoted-printable} ���֤���\var{body_encoding} �� \code{BASE64} �ξ��
\samp{base64} ���֤��ޤ�������ʳ��ξ���ʸ���� \samp{7bit} ���֤��ޤ���
\end{methoddesc}

\begin{methoddesc}{convert}{s}
ʸ���� \var{s} �� \var{input_codec} ���� \var{output_codec} ���Ѵ����ޤ���
\end{methoddesc}

\begin{methoddesc}{to_splittable}{s}
�����餯�ޥ���Х��Ȥ�ʸ����򡢰����� split �Ǥ���������Ѵ����ޤ���
\var{s} �ˤ� split ����ʸ������Ϥ��ޤ���

����� \var{input_codec} ��Ȥä�ʸ����� Unicode �ˤ��뤳�Ȥǡ�
ʸ����ʸ���ζ����� (���Ȥ����줬�ޥ���Х���ʸ���Ǥ��äƤ�) ������
split �Ǥ���褦�ˤ��ޤ���

\var{input_charset} ��ʸ���� \var{s} ��ɤ���ä� Unicode ���Ѵ�����Ф�������
�����ʾ�硢���Υ᥽�åɤ�Ϳ����줿ʸ���󤽤Τ�Τ��֤��ޤ���

Unicode ���Ѵ��Ǥ��ʤ��ä�ʸ���ϡ�Unicode �ִ�ʸ��
(Unicode replacement character) \character{U+FFFD} ���ִ�����ޤ���
\end{methoddesc}

\begin{methoddesc}{from_splittable}{ustr\optional{, to_output}}
split �Ǥ���ʸ����򥨥󥳡��ɤ��줿ʸ������Ѵ����ʤ����ޤ���
\var{ustr} �� ``��split'' ���뤿��� Unicode ʸ����Ǥ���

���Υ᥽�åɤǤϡ�ʸ����� Unicode ����٤ĤΥ��󥳡��ɷ������Ѵ����뤿���
Ŭ�ڤ� codec ����Ѥ��ޤ���Ϳ����줿ʸ���� Unicode �ǤϤʤ��ä���硢
���뤤�Ϥ����ɤ���ä� Unicode �����Ѵ����뤫�������ä����ϡ�
Ϳ����줿ʸ���󤽤Τ�Τ��֤���ޤ���

Unicode �����������Ѵ��Ǥ��ʤ��ä�ʸ���ˤĤ��Ƥϡ�
Ŭ����ʸ�� (�̾�� \character{?}) ���֤��������ޤ���

\var{to_output} �� \code{True} �ξ�� (�ǥե����)��
���Υ᥽�åɤ� \var{output_codec} �򥨥󥳡��ɤη����Ȥ���
���Ѥ��ޤ���\var{to_output} �� \code{False} �ξ�硢�����
\var{input_codec} ����Ѥ��ޤ���
\end{methoddesc}

\begin{methoddesc}{get_output_charset}{}
�����Ѥ�ʸ�����åȤ��֤��ޤ���

����� \var{output_charset} °���� \code{None} �Ǥʤ���Ф����ͤˤʤ�ޤ���
����ʳ��ξ�硢�����ͤ� \var{input_charset} ��Ʊ���Ǥ���
\end{methoddesc}

\begin{methoddesc}{encoded_header_len}{}
���󥳡��ɤ��줿�إå�ʸ�����Ĺ�����֤��ޤ���
����� quoted-printable ���󥳡��ǥ��󥰤��뤤�� base64 ���󥳡��ǥ��󥰤��Ф��Ƥ�
�������׻�����ޤ���
\end{methoddesc}

\begin{methoddesc}{header_encode}{s\optional{, convert}}
ʸ���� \var{s} ��إå��Ѥ˥��󥳡��ɤ��ޤ���

\var{convert} �� \code{True} �ξ�硢
ʸ�����������ʸ�����åȤ��������ʸ�����åȤ˼�ưŪ���Ѵ�����ޤ���
����ϹԤ�Ĺ������Τ���ޥ���Х��Ȥ�ʸ�����åȤ��Ф��Ƥ����Ω���ޤ���
(�ޥ���Х���ʸ���ϥХ��ȶ����ǤϤʤ���ʸ�����Ȥζ����� split ����ɬ�פ�����ޤ�)��
����������򰷤��ˤϡ�����Υ��饹�Ǥ��� \class{Header} ���饹��
�ȤäƤ������� (\refmodule{email.header} �򻲾�)��
\var{convert} ���ͤϥǥե���ȤǤ� \code{False} �Ǥ���

���󥳡��ǥ��󥰤η��� (base64 �ޤ��� quoted-printable) �ϡ�
\var{header_encoding} °���˴�Ť��ޤ���
\end{methoddesc}

\begin{methoddesc}{body_encode}{s\optional{, convert}}
ʸ���� \var{s} ���å����������Ѥ˥��󥳡��ɤ��ޤ���

\var{convert} �� \code{True} �ξ�� (�ǥե����)��
ʸ�����������ʸ�����åȤ��������ʸ�����åȤ˼�ưŪ���Ѵ�����ޤ���
\method{header_encode()} �Ȥϰۤʤꡢ��å��������ΤˤϤդĤ�
�Х��ȶ����������ޥ���Х���ʸ�����åȤ����꤬�ʤ��Τǡ�
����Ϥ����ư����ˤ����ʤ��ޤ���

���󥳡��ǥ��󥰤η��� (base64 �ޤ��� quoted-printable) �ϡ�
\var{body_encoding} °���˴�Ť��ޤ���
\end{methoddesc}

\class{Charset} ���饹�ˤϡ�
ɸ��Ū�ʱ黻���Ȥ߹��ߴؿ��򥵥ݡ��Ȥ���
�����Ĥ��Υ᥽�åɤ�����ޤ���

\begin{methoddesc}[Charset]{__str__}{}
\var{input_charset} ��ʸ�����Ѵ����줿ʸ���󷿤Ȥ����֤��ޤ���
\method{__repr__()} �ϡ�\method{__str__()} ����̾�ȤʤäƤ��ޤ���
\end{methoddesc}

\begin{methoddesc}[Charset]{__eq__}{other}
���Υ᥽�åɤϡ�2�Ĥ� \class{Charset} ���󥹥��󥹤�Ʊ�����ɤ���������å�����Τ˻Ȥ��ޤ���
\end{methoddesc}

\begin{methoddesc}[Header]{__ne__}{other}
���Υ᥽�åɤϡ�2�Ĥ� \class{Charset} ���󥹥��󥹤��ۤʤ뤫�ɤ���������å�����Τ˻Ȥ��ޤ���
\end{methoddesc}

�ޤ���\module{email.charset} �⥸�塼��ˤϡ�
�������Х��ʸ�����åȡ�ʸ�����åȤ���̾(�����ꥢ��) ����� codec �ѤΥ쥸���ȥ��
����������ȥ���ɲä���ʲ��δؿ���դ��ޤ�Ƥ��ޤ�:

\begin{funcdesc}{add_charset}{charset\optional{, header_enc\optional{,
    body_enc\optional{, output_charset}}}}
ʸ����°���򥰥����Х�ʥ쥸���ȥ���ɲä��ޤ���

\var{charset} �������Ѥ�ʸ�����åȤǡ�����ʸ�����åȤ�����̾�Τ���ꤹ��ɬ�פ�����ޤ���

���ץ������� \var{header_enc} ����� \var{body_enc} ��
quoted-printable ���󥳡��ǥ��󥰤򤢤�魯 \code{Charset.QP} ����
base64 ���󥳡��ǥ��󥰤򤢤�魯 \code{Charset.BASE64}��
��û�� quoted-printable �ޤ��� base64 ���󥳡��ǥ��󥰤򤢤�魯
\code{Charset.SHORTEST}�����뤤�ϥ��󥳡��ǥ��󥰤ʤ��� \code{None} ��
�ɤ줫�ˤʤ�ޤ���\code{SHORTEST} ���Ȥ���Τ� \var{header_enc} �����Ǥ���
�ǥե���Ȥ��ͤϥ��󥳡��ǥ��󥰤ʤ��� \code{None} �ˤʤäƤ��ޤ���

���ץ������� \var{output_charset} �ˤϽ����Ѥ�ʸ�����åȤ�����ޤ���
\method{Charset.convert()} ���ƤФ줿�Ȥ����Ѵ���
�ޤ������Ѥ�ʸ�����åȤ� Unicode ���Ѵ��������줫������Ѥ�ʸ�����åȤ�
�Ѵ�����ޤ����ǥե���ȤǤϡ����Ϥ����Ϥ�Ʊ��ʸ�����åȤˤʤäƤ��ޤ���

\var{input_charset} ����� \var{output_charset} ��
���Υ⥸�塼�����ʸ�����å�-codec �б�ɽ�ˤ��� Unicode codec ����ȥ�Ǥ���
ɬ�פ�����ޤ����⥸�塼�뤬�ޤ��б����Ƥ��ʤ� codec ���ɲä���ˤϡ�
\function{add_codec()} ��ȤäƤ������������ܤ�������ˤĤ��Ƥ�
\refmodule{codecs} �⥸�塼���ʸ��򻲾Ȥ��Ƥ���������

�������Х��ʸ�����å��ѤΥ쥸���ȥ�ϡ��⥸�塼��� global ����
\code{CHARSETS} ����ݻ�����Ƥ��ޤ���
\end{funcdesc}

\begin{funcdesc}{add_alias}{alias, canonical}
ʸ�����åȤ���̾ (�����ꥢ��) ���ɲä��ޤ���
\var{alias} �Ϥ�����̾�ǡ����Ȥ��� \code{latin-1} �Τ褦�˻��ꤷ�ޤ���
\var{canonical} �Ϥ���ʸ�����åȤ�����̾�Τǡ����Ȥ��� \code{iso-8859-1} �Τ褦�˻��ꤷ�ޤ���

ʸ�����åȤΥ������Х����̾�ѥ쥸���ȥ�ϡ��⥸�塼��� global ����
\code{ALIASES} ����ݻ�����Ƥ��ޤ���
\end{funcdesc}

\begin{funcdesc}{add_codec}{charset, codecname}
Ϳ����줿ʸ�����åȤ�ʸ���� Unicode �Ȥ��Ѵ��򤪤��ʤ� codec ���ɲä��ޤ���

\var{charset} �Ϥ���ʸ�����åȤ�����̾�Τǡ�
\var{codecname} �� Python �� codec ��̾���Ǥ���
������Ȥ߹��ߴؿ� \function{unicode()} ����2��������
���뤤�� Unicode ʸ���󷿤� \method{encode()} �᥽�åɤ�
Ŭ���������ˤʤäƤ��ʤ���Фʤ�ޤ���
\end{funcdesc}


\subsection{Encoders}
\declaremodule{standard}{email.encoders}
\modulesynopsis{�Żҥ᡼���å������Υڥ������ɤΤ���Υ��󥳡�����}

����ʤ��Ȥ������� \class{Message} ���������Ȥ����Ф���ɬ�פˤʤ�Τ���
�ڥ������ɤ�᡼�륵���Ф��̤�����˥��󥳡��ɤ��뤳�ȤǤ���
����ϤȤ��˥Х��ʥ�ǡ�����ޤ��
\mimetype{image/*} �� \mimetype{text/*} �����פΥ�å�������ɬ�פǤ���

\module{email} �ѥå������Ǥϡ�\module{encoders} �⥸�塼��ˤ�����
���������ص�Ū�ʥ��󥳡��ǥ��󥰤򥵥ݡ��Ȥ��Ƥ��ޤ����ºݤˤϤ�����
���󥳡����� \class{MIMEAudio} ����� \class{MIMEImage} ���饹��
���󥹥ȥ饯���ǥǥե���ȥ��󥳡����Ȥ��ƻȤ��Ƥ��ޤ���
���٤ƤΥ��󥳡��ǥ��󥰴ؿ��ϡ����󥳡��ɤ����å��������֥�������
�ҤȤĤ���������ˤȤ�ޤ��������ϤդĤ��ڥ������ɤ��������
����򥨥󥳡��ɤ��ơ��ڥ������ɤ򥨥󥳡��ɤ��줿��Τ˥��åȤ��ʤ����ޤ���
�����Ϥޤ� \mailheader{Content-Transfer-Encoding} �إå���Ŭ�ڤ��ͤ�
���ꤷ�ޤ���

�󶡤���Ƥ��륨�󥳡��ǥ��󥰴ؿ��ϰʲ��ΤȤ���Ǥ�:

\begin{funcdesc}{encode_quopri}{msg}
�ڥ������ɤ� quoted-printable �����˥��󥳡��ɤ���
\mailheader{Content-Transfer-Encoding} �إå���
\code{quoted-printable}\footnote{����: \method{encode_quopri()} ��
�Ȥäƥ��󥳡��ɤ���ȡ��ǡ�����Υ���ʸ�������ʸ����
���󥳡��ɤ���ޤ���} �����ꤷ�ޤ���
����Ϥ��Υڥ������ɤΤۤȤ�ɤ��̾�ΰ�����ǽ��ʸ������ʤäƤ��뤬��
�����Բ�ǽ��ʸ������������������Ȥ��Υ��󥳡�����ˡ�Ȥ���Ŭ���Ƥ��ޤ���
\end{funcdesc}

\begin{funcdesc}{encode_base64}{msg}
�ڥ������ɤ� base64 �����ǥ��󥳡��ɤ���
\mailheader{Content-Transfer-Encoding} �إå���
\code{base64} ���ѹ����ޤ�������ϥڥ����������
�ǡ����ΤۤȤ�ɤ������Բ�ǽ��ʸ���Ǥ������Ŭ���Ƥ��ޤ���
quoted-printable ���������̤Ȥ��Ƥϥ���ѥ��Ȥʥ������ˤʤ뤫��Ǥ���
base64 �����η����ϡ����줬�ʹ֤ˤϤޤä����ɤ�ʤ��ƥ����Ȥ�
�ʤäƤ��ޤ����ȤǤ���
\end{funcdesc}

\begin{funcdesc}{encode_7or8bit}{msg}
����ϼºݤˤϥڥ������ɤ��ѹ��Ϥ��ޤ��󤬡��ڥ������ɤη����˱�����
\mailheader{Content-Transfer-Encoding} �إå��� \code{7bit} ���뤤��
\code{8bit} ��Ŭ�����������ꤷ�ޤ���
\end{funcdesc}

\begin{funcdesc}{encode_noop}{msg}
����ϲ��⤷�ʤ����󥳡����Ǥ���
\mailheader{Content-Transfer-Encoding} �إå������ꤵ�����ޤ���
\end{funcdesc}


\subsection{Exception and Defect classes}
\declaremodule{standard}{email.errors}
\modulesynopsis{email �ѥå������ǻȤ����㳰���饹}

\module{email.errors} �⥸�塼��Ǥϡ�
�ʲ����㳰���饹���������Ƥ��ޤ�:

\begin{excclassdesc}{MessageError}{}
����� \module{email} �ѥå�������ȯ�������뤹�٤Ƥ��㳰�δ��쥯�饹�Ǥ���
�����ɸ��� \exception{Exception} ���饹�����������Ƥ��ꡢ
�ɲäΥ᥽�åɤϤޤä����������Ƥ��ޤ���
\end{excclassdesc}

\begin{excclassdesc}{MessageParseError}{}
����� \class{Parser} ���饹��ȯ���������㳰�δ��쥯�饹�Ǥ���
\exception{MessageError} �����������Ƥ��ޤ���
\end{excclassdesc}

\begin{excclassdesc}{HeaderParseError}{}
��å������� \rfc{2822} �إå�����Ϥ��Ƥ�������ˤ�����ǥ��顼���������ȯ�����ޤ���
����� \exception{MessageParseError} �����������Ƥ��ޤ���
�����㳰���������ǽ��������Τ� \method{Parser.parse()} �᥽�åɤ�
\method{Parser.parsestr()} �᥽�åɤǤ���

�����㳰��ȯ������Τϥ�å�������Ǻǽ�� \rfc{2822} �إå������줿���Ȥ�
����٥����ץإå������Ĥ��ä��Ȥ����ǽ�� \rfc{2822} �إå������������
���Υإå�����η�³�Ԥ����Ĥ��ä��Ȥ�����������ޤߤޤ���
���뤤�ϥإå��Ǥ��³�ԤǤ�ʤ��Ԥ��إå���˸��Ĥ��ä����Ǥ�
�����㳰��ȯ�����ޤ���
\end{excclassdesc}

\begin{excclassdesc}{BoundaryError}{}
��å������� \rfc{2822} �إå�����Ϥ��Ƥ�������ˤ�����ǥ��顼���������ȯ�����ޤ���
����� \exception{MessageParseError} �����������Ƥ��ޤ���
�����㳰���������ǽ��������Τ� \method{Parser.parse()} �᥽�åɤ�
\method{Parser.parsestr()} �᥽�åɤǤ���

�����㳰��ȯ������Τϡ����ʤʥѡ����������Ѥ����Ƥ���Ȥ��ˡ�
\mimetype{multipart/*} �����γ��Ϥ��뤤�Ͻ�λ��ʸ���󤬸��Ĥ���ʤ��ä����ʤɤǤ���
\end{excclassdesc}

\begin{excclassdesc}{MultipartConversionError}{}
�����㳰�ϡ�
\class{Message} ���֥������Ȥ� \method{add_payload()} �᥽�åɤ�Ȥä�
�ڥ������ɤ��ɲä���Ȥ������Υڥ������ɤ����Ǥ�ñ����ͤǤ���
(����: �ꥹ�ȤǤʤ�) �ˤ⤫����餺�����Υ�å������� \mailheader{Content-Type} 
�إå��Υᥤ�󥿥��פ����Ǥ����ꤵ��Ƥ��ơ����줬 \mimetype{multipart} �ʳ��ˤʤä�
���ޤäƤ�����ˤ����㳰��ȯ�����ޤ���
\exception{MultipartConversionError} �� \exception{MessageError} ��
�Ȥ߹��ߤ� \exception{TypeError} ��ξ���Ѿ����Ƥ��ޤ���

\method{Message.add_payload()} �Ϥ�Ϥ�侩����ʤ��᥽�åɤΤ��ᡢ
�����㳰�ϤդĤ���ä���ȯ�����ޤ��󡣤����������㳰��
\method{attach()} �᥽�åɤ� \class{MIMENonMultipart} ����
�����������饹�Υ��󥹥��� (��: \class{MIMEImage} �ʤ�) ���Ф���
�ƤФ줿�Ȥ��ˤ�ȯ�����뤳�Ȥ�����ޤ���
\end{excclassdesc}

�ʲ��� \class{FeedParser} ����å������β�����˸��Ф���㳲 (defect) �ΰ����Ǥ���
����: �����ξ㳲�ϡ����꤬���Ĥ��ä���å��������ɲä���뤿�ᡢ���Ȥ���
\mimetype{multipart/alternative} ��ˤ���ͥ��Ȥ�����å�������
�۾�ʥإå����äƤ������ˤϡ����Υͥ��Ȥ�����å��������㳲��
���äƤ��뤬�����οƥ�å������ˤϾ㳲�Ϥʤ��Ȥߤʤ���ޤ���

���٤Ƥξ㳲���饹�� \class{email.errors.MessageDefect} �Υ��֥��饹�Ǥ�����
������㳰�Ȥ�\emph{�㤤�ޤ�}�Τ����դ��Ƥ���������

\versionadded[All the defect classes were added]{2.4}

\begin{itemize}
\item \class{NoBoundaryInMultipartDefect} -- ��å������� multipart �����������Ƥ���Τˡ�
      \mimetype{boundary} �ѥ�᡼�����ʤ���

\item \class{StartBoundaryNotFoundDefect} -- \mailheader{Content-Type} �إå���������줿
      ���϶������ʤ���

\item \class{FirstHeaderLineIsContinuationDefect} -- ��å������κǽ�Υإå���
      ��³�Ԥ���ϤޤäƤ��롣

\item \class{MisplacedEnvelopeHeaderDefect} -- �إå��֥��å�������� ``Unix From'' �إå������롣

\item \class{MalformedHeaderDefect} -- ������Τʤ��إå������롢���뤤�Ϥ���ʳ��ΰ۾�ʥإå��Ǥ��롣

\item \class{MultipartInvariantViolationDefect} -- �������� \mimetype{multipart} ����
      �������Ƥ���Τˡ����֥ѡ��Ȥ�¸�ߤ��ʤ�������: ��å����������ξ㳲����äƤ���Ȥ���
      \method{is_multipart()} �᥽�åɤ� ���Ȥ����� content-type �� \mimetype{multipart} �Ǥ��äƤ�
      false ���֤����Ȥ�����ޤ���
\end{itemize}


\subsection{Miscellaneous utilities}
\declaremodule{standard}{email.utils}
\modulesynopsis{�Żҥ᡼��ѥå������λ�¿�ʥ桼�ƥ���ƥ���}

\module{email.utils} �⥸�塼��ǤϤ����Ĥ��������ʥ桼�ƥ���ƥ����󶡤��Ƥ��ޤ���

\begin{funcdesc}{quote}{str}
ʸ���� \var{str} ��ΥХå�����å���� �Хå�����å���2�� ���ִ�����
������ʸ������֤��ޤ����ޤ������֥륯�����Ȥ� �Хå�����å��� + ���֥륯�����Ȥ��ִ�����ޤ���
\end{funcdesc}

\begin{funcdesc}{unquote}{str}
ʸ���� \var{str} �� \emph{�ե�������}����������ʸ������֤��ޤ���
�⤷ \var{str} ����Ƭ���뤤�����������֥륯�����Ȥ��ä���硢
������ñ���ڤꤪ�Ȥ���ޤ���Ʊ�ͤˤ⤷ \var{str} ����Ƭ���뤤��������
�ѥ֥饱�å� (<��>) ���ä������ڤꤪ�Ȥ���ޤ���
\end{funcdesc}

\begin{funcdesc}{parseaddr}{address}
���ɥ쥹��ѡ������ޤ���\mailheader{To} �� \mailheader{Cc} �Τ褦��
���ɥ쥹��դ�����ե�����ɤ��ͤ�Ϳ����ȡ�������ʬ��
\emph{��̾} �� \emph{�Żҥ᡼�륢�ɥ쥹} ����Ф��ޤ���
�ѡ���������������硢�����ξ���򥿥ץ�
\code{(realname, email_address)} �ˤ����֤��ޤ���
���Ԥ������� 2���ǤΥ��ץ� \code{('', '')} ���֤��ޤ���
\end{funcdesc}

\begin{funcdesc}{formataddr}{pair}
\method{parseaddr()} �εդǡ���̾���Żҥ᡼�륢�ɥ쥹����ʤ�
2���ǤΥ��ץ� \code{(realname, email_address)} ������ˤȤꡢ
\mailheader{To} ���뤤�� \mailheader{Cc} �إå���Ŭ����������ʸ�����
�֤��ޤ������ץ� \var{pair} ����1���Ǥ����Ǥ����硢��2���Ǥ��ͤ�
���Τޤ��֤��ޤ���
\end{funcdesc}

\begin{funcdesc}{getaddresses}{fieldvalues}
���Υ᥽�åɤ� 2���ǥ��ץ�Υꥹ�Ȥ� \code{parseaddr()} ��Ʊ���������֤��ޤ���
\var{fieldvalues} �Ϥ��Ȥ��� \method{Message.get_all()} ���֤��褦�ʡ�
�إå��Υե�������ͤ���ʤ륷�����󥹤Ǥ����ʲ��Ϥ����Żҥ᡼���å���������
���٤Ƥμ������ͤ��������Ǥ�:

\begin{verbatim}
from email.utils import getaddresses

tos = msg.get_all('to', [])
ccs = msg.get_all('cc', [])
resent_tos = msg.get_all('resent-to', [])
resent_ccs = msg.get_all('resent-cc', [])
all_recipients = getaddresses(tos + ccs + resent_tos + resent_ccs)
\end{verbatim}
\end{funcdesc}

\begin{funcdesc}{parsedate}{date}
\rfc{2822} �˵����줿��§�ˤ�ȤŤ������դ���Ϥ��ޤ���
���������ᥤ�顼�ˤ�äƤϤ����ǻ��ꤵ�줿��§�˽��äƤ��ʤ���Τ����ꡢ
���Τ褦�ʾ�� \function{parsedate()} �Ϥʤ�٤����������դ��¬���褦�Ȥ��ޤ���
\var{date} �� \rfc{2822} ���������դ��ݻ����Ƥ���ʸ����ǡ�
\code{"Mon, 20 Nov 1995 19:12:08 -0500"} �Τ褦�ʷ��򤷤Ƥ��ޤ���
���դβ��Ϥ�����������硢\function{parsedate()} ��
�ؿ� \function{time.mktime()} ��ľ���Ϥ��������
9���Ǥ���ʤ륿�ץ���֤������Ԥ������� \code{None} ���֤��ޤ���
�֤���륿�ץ�� 6��7��8���ܤΥե�����ɤ�ͭ���ǤϤʤ��Τ����դ��Ƥ���������
\end{funcdesc}

\begin{funcdesc}{parsedate_tz}{date}
\function{parsedate()} ��Ʊ�ͤε�ǽ���󶡤��ޤ�����
\code{None} �ޤ��� 10���ǤΥ��ץ���֤��Ȥ������㤤�ޤ���
�ǽ�� 9�Ĥ����Ǥ� \function{time.mktime()} ��ľ���Ϥ�������Τ�ΤǤ��ꡢ
�Ǹ�� 10���ܤ����Ǥϡ��������դλ����Ӥ� UTC
(����˥å�ɸ����θ����ʸƤ�̾�Ǥ�) ���Ф��륪�ե��åȤǤ�
\footnote{����: ���λ����ӤΥ��ե��å��ͤ� \code{time.timezone} ���ͤ�
��礬�դǤ�������� \code{time.timezone} �� \POSIX{} ɸ��˽�򤷤Ƥ���Τ��Ф��ơ�
������� \rfc{2822} �˽�򤷤Ƥ��뤫��Ǥ���}��
���Ϥ��줿ʸ����˻����Ӥ����ꤵ��Ƥ��ʤ��ä���硢10���ܤ����Ǥˤ�
\code{None} ������ޤ���
���ץ�� 6��7��8���ܤΥե�����ɤ�ͭ���ǤϤʤ��Τ����դ��Ƥ���������
\end{funcdesc}

\begin{funcdesc}{mktime_tz}{tuple}
\function{parsedate_tz()} ���֤� 10���ǤΥ��ץ�� UTC ��
�����ॹ����פ��Ѵ����ޤ���Ϳ����줿�����Ӥ� \code{None} �Ǥ����硢
�����ӤȤ��Ƹ��ϻ��� (localtime) �����ꤵ��ޤ���
�ޥ��ʡ��ʷ���: \function{mktime_tz()} �Ϥޤ� \var{tuple} �κǽ�� 8���Ǥ�
localtime �Ȥ����Ѵ������Ĥ��˻����Ӥκ����̣���Ƥ��ޤ���
�ƻ��֤�ȤäƤ�����ˤϡ�������̾�λ��ѤˤϤ����Ĥ����ʤ���ΤΡ�
�鷺���ʸ����������뤫�⤷��ޤ���
\end{funcdesc}

\begin{funcdesc}{formatdate}{\optional{timeval\optional{, localtime}\optional{, usegmt}}}
���դ� \rfc{2822} ������ʸ������֤��ޤ�����:

\begin{verbatim}
Fri, 09 Nov 2001 01:08:47 -0000
\end{verbatim}

���ץ����Ȥ��� float �����ͤ��İ��� \var{timeval} ��Ϳ����줿��硢
����� \function{time.gmtime()} ����� \function{time.localtime()} ��
�Ϥ���ޤ�������ʳ��ξ�硢���ߤλ��郎�Ȥ��ޤ���

���ץ������� \var{localtime} �ϥե饰�Ǥ���
���줬 \code{True} �ξ�硢���δؿ��� \var{timeval} ����Ϥ�������
UTC �Τ����˸��ϻ��� (localtime) �λ����Ӥ�Ĥ��ä��Ѵ����ޤ���
�����餯�ƻ��֤��θ���������Ǥ��礦��
�ǥե���ȤǤϤ����ͤ� \code{False} �ǡ�UTC ���Ȥ��ޤ���

���ץ������� \var{usegmt} �� \code{True} �ΤȤ��ϡ������ॾ�����ɽ���Τ�
���ͤ� \code{-0000} �ǤϤʤ� asciiʸ����Ǥ��� \code{GMT} ���Ȥ��ޤ���
����� (HTTP �ʤɤ�) �����Ĥ��Υץ��ȥ����ɬ�פǤ���
���ε�ǽ�� \var{localtime} �� \code{False} �ΤȤ��Τ�Ŭ�Ѥ���ޤ���
\versionadded{2.4}
\end{funcdesc}

\begin{funcdesc}{make_msgid}{\optional{idstring}}
\rfc{2822} �������� \mailheader{Message-ID} �إå���Ŭ����
ʸ������֤��ޤ������ץ������� \var{idstring} ��ʸ����Ȥ���
Ϳ����줿��硢����ϥ�å����� ID �ΰ���������Τ����Ѥ���ޤ���
\end{funcdesc}

\begin{funcdesc}{decode_rfc2231}{s}
\rfc{2231} �˽��ä�ʸ���� \var{s} ��ǥ����ɤ��ޤ���
\end{funcdesc}

\begin{funcdesc}{encode_rfc2231}{s\optional{, charset\optional{, language}}}
\rfc{2231} �˽��ä� \var{s} �򥨥󥳡��ɤ��ޤ���
���ץ������� \var{charset} ����� \var{language} ��Ϳ����줿��硢
������ʸ�����å�̾�ȸ���̾�Ȥ��ƻȤ��ޤ���
�⤷�����Τɤ����Ϳ�����Ƥ��ʤ���硢\var{s} �Ϥ��Τޤ��֤���ޤ���
\var{charset} ��Ϳ�����Ƥ��뤬 \var{language} ��Ϳ�����Ƥ��ʤ���硢
ʸ���� \var{s} �� \var{language} �ζ�ʸ�����Ȥäƥ��󥳡��ɤ���ޤ���
\end{funcdesc}

\begin{funcdesc}{collapse_rfc2231_value}{value\optional{, errors\optional{,
    fallback_charset}}}
�إå��Υѥ�᡼���� \rfc{2231} �����ǥ��󥳡��ɤ���Ƥ����硢
\method{Message.get_param()} �� 3���Ǥ���ʤ륿�ץ���֤����Ȥ�����ޤ���
�����ˤϡ����Υѥ�᡼����ʸ�����åȡ����졢������ͤν�˳�Ǽ����Ƥ��ޤ���
\function{collapse_rfc2231_value()} �Ϥ��Υѥ�᡼����ҤȤĤ� Unicode ʸ�����
�ޤȤ�ޤ������ץ������� \var{errors} �� built-in �Ǥ��� \function{unicode()} �ؿ���
���� \var{errors} ���Ϥ���ޤ������Υǥե�����ͤ� \code{replace} �ȤʤäƤ��ޤ���
���ץ������� \var{fallback_charset} �ϡ��⤷ \rfc{2231} �إå��λ��Ѥ��Ƥ���
ʸ�����åȤ� Python ���ΤäƤ����ΤǤϤʤ��ä����������ʸ�����åȤȤ���
�Ȥ��ޤ����ǥե���ȤǤϡ������ͤ� \code{us-ascii} �Ǥ���

�ص��塢\function{collapse_rfc2231_value()} ���Ϥ��줿���� \var{value} ��
���ץ�Ǥʤ����ˤϡ������ʸ����Ǥ���ɬ�פ�����ޤ������ξ��ˤ�
unquote ���줿ʸ�����֤���ޤ���
\end{funcdesc}

\begin{funcdesc}{decode_params}{params}
\rfc{2231} �˽��äƥѥ�᡼���Υꥹ�Ȥ�ǥ����ɤ��ޤ���
\var{params} �� \code{(content-type, string-value)} �Τ褦�ʷ�����
2���Ǥ���ʤ륿�ץ�Ǥ���
\end{funcdesc}

\versionchanged[\function{dump_address_pair()} �ؿ���ű���ޤ����������� 
\function{formataddr()} �ؿ���ȤäƤ���������]{2.4}

\versionchanged[\function{decode()} �ؿ���ű���ޤ����������� 
\method{Header.decode_header()} �᥽�åɤ�ȤäƤ���������]{2.4}
 
\versionchanged[\function{encode()} �ؿ���ű���ޤ����������� 
\method{Header.encode()} �᥽�åɤ�ȤäƤ���������]{2.4}


\subsection{Iterators}
\declaremodule{standard}{email.iterators}
\modulesynopsis{Iterate over a  message object tree.}

Iterating over a message object tree is fairly easy with the
\method{Message.walk()} method.  The \module{email.iterators} module
provides some useful higher level iterations over message object
trees.

\begin{funcdesc}{body_line_iterator}{msg\optional{, decode}}
This iterates over all the payloads in all the subparts of \var{msg},
returning the string payloads line-by-line.  It skips over all the
subpart headers, and it skips over any subpart with a payload that
isn't a Python string.  This is somewhat equivalent to reading the
flat text representation of the message from a file using
\method{readline()}, skipping over all the intervening headers.

Optional \var{decode} is passed through to \method{Message.get_payload()}.
\end{funcdesc}

\begin{funcdesc}{typed_subpart_iterator}{msg\optional{,
    maintype\optional{, subtype}}}
This iterates over all the subparts of \var{msg}, returning only those
subparts that match the MIME type specified by \var{maintype} and
\var{subtype}.

Note that \var{subtype} is optional; if omitted, then subpart MIME
type matching is done only with the main type.  \var{maintype} is
optional too; it defaults to \mimetype{text}.

Thus, by default \function{typed_subpart_iterator()} returns each
subpart that has a MIME type of \mimetype{text/*}.
\end{funcdesc}

The following function has been added as a useful debugging tool.  It
should \emph{not} be considered part of the supported public interface
for the package.

\begin{funcdesc}{_structure}{msg\optional{, fp\optional{, level}}}
Prints an indented representation of the content types of the
message object structure.  For example:

\begin{verbatim}
>>> msg = email.message_from_file(somefile)
>>> _structure(msg)
multipart/mixed
    text/plain
    text/plain
    multipart/digest
        message/rfc822
            text/plain
        message/rfc822
            text/plain
        message/rfc822
            text/plain
        message/rfc822
            text/plain
        message/rfc822
            text/plain
    text/plain
\end{verbatim}

Optional \var{fp} is a file-like object to print the output to.  It
must be suitable for Python's extended print statement.  \var{level}
is used internally.
\end{funcdesc}


\subsection{Package History\label{email-pkg-history}}

This table describes the release history of the email package, corresponding
to the version of Python that the package was released with.  For purposes of
this document, when you see a note about change or added versions, these refer
to the Python version the change was made it, \emph{not} the email package
version.  This table also describes the Python compatibility of each version
of the package.

\begin{tableiii}{l|l|l}{constant}{email version}{distributed with}{compatible with}
\lineiii{1.x}{Python 2.2.0 to Python 2.2.1}{\emph{no longer supported}}
\lineiii{2.5}{Python 2.2.2+ and Python 2.3}{Python 2.1 to 2.5}
\lineiii{3.0}{Python 2.4}{Python 2.3 to 2.5}
\lineiii{4.0}{Python 2.5}{Python 2.3 to 2.5}
\end{tableiii}

Here are the major differences between \module{email} version 4 and version 3:

\begin{itemize}
\item All modules have been renamed according to \pep{8} standards.  For
      example, the version 3 module \module{email.Message} was renamed to
      \module{email.message} in version 4.

\item A new subpackage \module{email.mime} was added and all the version 3
      \module{email.MIME*} modules were renamed and situated into the
      \module{email.mime} subpackage.  For example, the version 3 module
      \module{email.MIMEText} was renamed to \module{email.mime.text}.

      \emph{Note that the version 3 names will continue to work until Python
      2.6}.

\item The \module{email.mime.application} module was added, which contains the
      \class{MIMEApplication} class.

\item Methods that were deprecated in version 3 have been removed.  These
      include \method{Generator.__call__()}, \method{Message.get_type()},
      \method{Message.get_main_type()}, \method{Message.get_subtype()}.

\item Fixes have been added for \rfc{2231} support which can change some of
      the return types for \function{Message.get_param()} and friends.  Under
      some circumstances, values which used to return a 3-tuple now return
      simple strings (specifically, if all extended parameter segments were
      unencoded, there is no language and charset designation expected, so the
      return type is now a simple string).  Also, \%-decoding used to be done
      for both encoded and unencoded segments; this decoding is now done only
      for encoded segments.
\end{itemize}

Here are the major differences between \module{email} version 3 and version 2:

\begin{itemize}
\item The \class{FeedParser} class was introduced, and the \class{Parser}
      class was implemented in terms of the \class{FeedParser}.  All parsing
      therefore is non-strict, and parsing will make a best effort never to
      raise an exception.  Problems found while parsing messages are stored in
      the message's \var{defect} attribute.

\item All aspects of the API which raised \exception{DeprecationWarning}s in
      version 2 have been removed.  These include the \var{_encoder} argument
      to the \class{MIMEText} constructor, the \method{Message.add_payload()}
      method, the \function{Utils.dump_address_pair()} function, and the
      functions \function{Utils.decode()} and \function{Utils.encode()}.

\item New \exception{DeprecationWarning}s have been added to:
      \method{Generator.__call__()}, \method{Message.get_type()},
      \method{Message.get_main_type()}, \method{Message.get_subtype()}, and
      the \var{strict} argument to the \class{Parser} class.  These are
      expected to be removed in future versions.

\item Support for Pythons earlier than 2.3 has been removed.
\end{itemize}

Here are the differences between \module{email} version 2 and version 1:

\begin{itemize}
\item The \module{email.Header} and \module{email.Charset} modules
      have been added.

\item The pickle format for \class{Message} instances has changed.
      Since this was never (and still isn't) formally defined, this
      isn't considered a backward incompatibility.  However if your
      application pickles and unpickles \class{Message} instances, be
      aware that in \module{email} version 2, \class{Message}
      instances now have private variables \var{_charset} and
      \var{_default_type}.

\item Several methods in the \class{Message} class have been
      deprecated, or their signatures changed.  Also, many new methods
      have been added.  See the documentation for the \class{Message}
      class for details.  The changes should be completely backward
      compatible.

\item The object structure has changed in the face of
      \mimetype{message/rfc822} content types.  In \module{email}
      version 1, such a type would be represented by a scalar payload,
      i.e. the container message's \method{is_multipart()} returned
      false, \method{get_payload()} was not a list object, but a single
      \class{Message} instance.

      This structure was inconsistent with the rest of the package, so
      the object representation for \mimetype{message/rfc822} content
      types was changed.  In \module{email} version 2, the container
      \emph{does} return \code{True} from \method{is_multipart()}, and
      \method{get_payload()} returns a list containing a single
      \class{Message} item.

      Note that this is one place that backward compatibility could
      not be completely maintained.  However, if you're already
      testing the return type of \method{get_payload()}, you should be
      fine.  You just need to make sure your code doesn't do a
      \method{set_payload()} with a \class{Message} instance on a
      container with a content type of \mimetype{message/rfc822}.

\item The \class{Parser} constructor's \var{strict} argument was
      added, and its \method{parse()} and \method{parsestr()} methods
      grew a \var{headersonly} argument.  The \var{strict} flag was
      also added to functions \function{email.message_from_file()}
      and \function{email.message_from_string()}.

\item \method{Generator.__call__()} is deprecated; use
      \method{Generator.flatten()} instead.  The \class{Generator}
      class has also grown the \method{clone()} method.

\item The \class{DecodedGenerator} class in the
      \module{email.Generator} module was added.

\item The intermediate base classes \class{MIMENonMultipart} and
      \class{MIMEMultipart} have been added, and interposed in the
      class hierarchy for most of the other MIME-related derived
      classes.

\item The \var{_encoder} argument to the \class{MIMEText} constructor
      has been deprecated.  Encoding  now happens implicitly based
      on the \var{_charset} argument.

\item The following functions in the \module{email.Utils} module have
      been deprecated: \function{dump_address_pairs()},
      \function{decode()}, and \function{encode()}.  The following
      functions have been added to the module:
      \function{make_msgid()}, \function{decode_rfc2231()},
      \function{encode_rfc2231()}, and \function{decode_params()}.

\item The non-public function \function{email.Iterators._structure()}
      was added.
\end{itemize}

\subsection{Differences from \module{mimelib}}

The \module{email} package was originally prototyped as a separate
library called
\ulink{\module{mimelib}}{http://mimelib.sf.net/}.
Changes have been made so that
method names are more consistent, and some methods or modules have
either been added or removed.  The semantics of some of the methods
have also changed.  For the most part, any functionality available in
\module{mimelib} is still available in the \refmodule{email} package,
albeit often in a different way.  Backward compatibility between
the \module{mimelib} package and the \module{email} package was not a
priority.

Here is a brief description of the differences between the
\module{mimelib} and the \refmodule{email} packages, along with hints on
how to port your applications.

Of course, the most visible difference between the two packages is
that the package name has been changed to \refmodule{email}.  In
addition, the top-level package has the following differences:

\begin{itemize}
\item \function{messageFromString()} has been renamed to
      \function{message_from_string()}.

\item \function{messageFromFile()} has been renamed to
      \function{message_from_file()}.

\end{itemize}

The \class{Message} class has the following differences:

\begin{itemize}
\item The method \method{asString()} was renamed to \method{as_string()}.

\item The method \method{ismultipart()} was renamed to
      \method{is_multipart()}.

\item The \method{get_payload()} method has grown a \var{decode}
      optional argument.

\item The method \method{getall()} was renamed to \method{get_all()}.

\item The method \method{addheader()} was renamed to \method{add_header()}.

\item The method \method{gettype()} was renamed to \method{get_type()}.

\item The method \method{getmaintype()} was renamed to
      \method{get_main_type()}.

\item The method \method{getsubtype()} was renamed to
      \method{get_subtype()}.

\item The method \method{getparams()} was renamed to
      \method{get_params()}.
      Also, whereas \method{getparams()} returned a list of strings,
      \method{get_params()} returns a list of 2-tuples, effectively
      the key/value pairs of the parameters, split on the \character{=}
      sign.

\item The method \method{getparam()} was renamed to \method{get_param()}.

\item The method \method{getcharsets()} was renamed to
      \method{get_charsets()}.

\item The method \method{getfilename()} was renamed to
      \method{get_filename()}.

\item The method \method{getboundary()} was renamed to
      \method{get_boundary()}.

\item The method \method{setboundary()} was renamed to
      \method{set_boundary()}.

\item The method \method{getdecodedpayload()} was removed.  To get
      similar functionality, pass the value 1 to the \var{decode} flag
      of the {get_payload()} method.

\item The method \method{getpayloadastext()} was removed.  Similar
      functionality
      is supported by the \class{DecodedGenerator} class in the
      \refmodule{email.generator} module.

\item The method \method{getbodyastext()} was removed.  You can get
      similar functionality by creating an iterator with
      \function{typed_subpart_iterator()} in the
      \refmodule{email.iterators} module.
\end{itemize}

The \class{Parser} class has no differences in its public interface.
It does have some additional smarts to recognize
\mimetype{message/delivery-status} type messages, which it represents as
a \class{Message} instance containing separate \class{Message}
subparts for each header block in the delivery status
notification\footnote{Delivery Status Notifications (DSN) are defined
in \rfc{1894}.}.

The \class{Generator} class has no differences in its public
interface.  There is a new class in the \refmodule{email.generator}
module though, called \class{DecodedGenerator} which provides most of
the functionality previously available in the
\method{Message.getpayloadastext()} method.

The following modules and classes have been changed:

\begin{itemize}
\item The \class{MIMEBase} class constructor arguments \var{_major}
      and \var{_minor} have changed to \var{_maintype} and
      \var{_subtype} respectively.

\item The \code{Image} class/module has been renamed to
      \code{MIMEImage}.  The \var{_minor} argument has been renamed to
      \var{_subtype}.

\item The \code{Text} class/module has been renamed to
      \code{MIMEText}.  The \var{_minor} argument has been renamed to
      \var{_subtype}.

\item The \code{MessageRFC822} class/module has been renamed to
      \code{MIMEMessage}.  Note that an earlier version of
      \module{mimelib} called this class/module \code{RFC822}, but
      that clashed with the Python standard library module
      \refmodule{rfc822} on some case-insensitive file systems.

      Also, the \class{MIMEMessage} class now represents any kind of
      MIME message with main type \mimetype{message}.  It takes an
      optional argument \var{_subtype} which is used to set the MIME
      subtype.  \var{_subtype} defaults to \mimetype{rfc822}.
\end{itemize}

\module{mimelib} provided some utility functions in its
\module{address} and \module{date} modules.  All of these functions
have been moved to the \refmodule{email.utils} module.

The \code{MsgReader} class/module has been removed.  Its functionality
is most closely supported in the \function{body_line_iterator()}
function in the \refmodule{email.iterators} module.

\subsection{Examples}

Here are a few examples of how to use the \module{email} package to
read, write, and send simple email messages, as well as more complex
MIME messages.

First, let's see how to create and send a simple text message:

\verbatiminput{email-simple.py}

Here's an example of how to send a MIME message containing a bunch of
family pictures that may be residing in a directory:

\verbatiminput{email-mime.py}

Here's an example of how to send the entire contents of a directory as
an email message:
\footnote{Thanks to Matthew Dixon Cowles for the original inspiration
          and examples.}

\verbatiminput{email-dir.py}

And finally, here's an example of how to unpack a MIME message like
the one above, into a directory of files:

\verbatiminput{email-unpack.py}
