\documentclass{manual}

% XXX PM explain how to add new types to Python

\title{Extending and Embedding the Python Interpreter}

\author{Guido van Rossum\\
	Fred L. Drake, Jr., editor}
\authoraddress{
	\strong{Python Software Foundation}\\
	Email: \email{docs@python.org}
}

\date{19th September, 2006}			% XXX update before final release!
% This file is generated by ../tools/getversioninfo;
% do not edit manually.

\release{2.5}
\setreleaseinfo{}
\setshortversion{2.5}
		% include Python version information


% Tell \index to actually write the .idx file
\makeindex

\begin{document}

\maketitle

\ifhtml
\chapter*{Front Matter\label{front}}
\fi

Copyright \copyright{} 2001-2006 Python Software Foundation.
All rights reserved.

Copyright \copyright{} 2000 BeOpen.com.
All rights reserved.

Copyright \copyright{} 1995-2000 Corporation for National Research Initiatives.
All rights reserved.

Copyright \copyright{} 1991-1995 Stichting Mathematisch Centrum.
All rights reserved.

See the end of this document for complete license and permissions
information.



\begin{abstract}

\noindent
Python is an interpreted, object-oriented programming language.  This
document describes how to write modules in C or \Cpp{} to extend the
Python interpreter with new modules.  Those modules can define new
functions but also new object types and their methods.  The document
also describes how to embed the Python interpreter in another
application, for use as an extension language.  Finally, it shows how
to compile and link extension modules so that they can be loaded
dynamically (at run time) into the interpreter, if the underlying
operating system supports this feature.

This document assumes basic knowledge about Python.  For an informal
introduction to the language, see the
\citetitle[../tut/tut.html]{Python Tutorial}.  The
\citetitle[../ref/ref.html]{Python Reference Manual} gives a more
formal definition of the language.  The
\citetitle[../lib/lib.html]{Python Library Reference} documents the
existing object types, functions and modules (both built-in and
written in Python) that give the language its wide application range.

For a detailed description of the whole Python/C API, see the separate
\citetitle[../api/api.html]{Python/C API Reference Manual}.

\end{abstract}

\tableofcontents


\chapter{\C{} �� \Cpp{} �ˤ�� Python �γ�ĥ \label{intro}}

C �ץ������ν������ΤäƤ���ʤ顢Python �˿������Ȥ߹���
�⥸�塼����ɲä���ΤϤ����ƴ�ñ�Ǥ���
���ο����ʥ⥸�塼�롢��ĥ�⥸�塼�� (\dfn{extention module})
��Ȥ��ȡ�Python ��ľ�ܹԤ��ʤ���ĤΤ���: �������Ȥ߹���
���֥������Ȥμ��������������Ƥ� C �饤�֥��ؿ���
�����ƥॳ������Ф���ƤӽФ������Ǥ���褦�ˤʤ�ޤ���

��ĥ�⥸�塼��򥵥ݡ��Ȥ��뤿�ᡢPython API
(Application Programmer's Interface) �Ǥϰ�Ϣ�δؿ����ޥ���
������ѿ����󶡤��Ƥ��ơ�Python ��󥿥��ॷ���ƥ��
�ۤȤ�ɤ�¦�̤ؤΥ����������ʤ��󶡤��Ƥ��ޤ���
Python API �ϡ��إå� \code{"Python.h"} �򥤥󥯥롼�ɤ���
C �������˼����ߤޤ���

��ĥ�⥸�塼��Υ���ѥ�����ˡ�ϡ��⥸�塼������Ӥ䥷���ƥ��
������ˡ�˰�¸���ޤ�; �ܺ٤ϸ�ξϤ��������ޤ���


\section{����
         \label{simpleExample}}

\samp{spam} (Monty Python �ե���ι�ʪ�Ǥ���) �Ȥ���̾�γ�ĥ�⥸�塼���
�������뤳�Ȥˤ��ơ�C �饤�֥��ؿ� \cfunction{system()} ���Ф���
Python ���󥿥ե�����������������Ȥ��ޤ���
\footnote{���δؿ��ؤΥ��󥿥ե������Ϥ��Ǥ�ɸ��⥸�塼�� \module{os}
�ˤ���ޤ� --- ���δؿ���������Τϡ�ñ���ľ��Ū����򼨤���������Ǥ���}
���δؿ��� null �ǽ�ü���줿����饯��ʸ���������ˤȤꡢ
�������֤��ޤ������δؿ���ʲ��Τ褦�ˤ��� Python ����ƤӽФ���褦��
�������Ȥ��ޤ�:

\begin{verbatim}
>>> import spam
>>> status = spam.system("ls -l")
\end{verbatim}

�ޤ��� \file{spammodule.c} ���������Ȥ�������Ϥ�ޤ���
(�����Ȥ��ơ�\samp{spam} �Ȥ���̾���Υ⥸�塼�����������硢
�⥸�塼��μ��������ä� C �ե������ \file{spammodule.c} ��
�Ƥ֤��ȤˤʤäƤ��ޤ�;  \samp{spammify} �Τ褦��Ĺ������
�⥸�塼��̾�ξ��ˤϡ�ñ��\file{spammify.c} �ˤ�Ǥ��ޤ���)

���Υե�����κǽ�ιԤϰʲ��Τ褦�ˤ��ޤ�:

\begin{verbatim}
#include <Python.h>
\end{verbatim}

����ǡ�Python API ������ߤޤ� (ɬ�פʤ顢�⥸�塼������Ӥ�
�ؤ��������䡢���ɽ�����ɲä��ޤ�)��
Python �ϡ������ƥ�ˤ�äƤ�ɸ��إå�������˱ƶ�����褦��
�ץ�ץ����å������ԤäƤ���Τǡ� \file{Python.h} ��
�������ɸ��إå��������˥��󥯥롼�ɤ��ͤФʤ�ޤ���

\file{Python.h} ���������Ƥ���桼������Ļ�Υ���ܥ�ϡ�
������Ƭ��\samp{Py} �ޤ��� \samp{PY} ���դ��Ƥ��ޤ�����������
ɸ��إå��ե������������Ͻ����ޤ���
��ñ�Τ���ȡ�Python ��ǹ��Ϥ˻Ȥ����Ȥˤʤ�Ȥ�����ͳ���顢
\code{"Python.h"} �Ϥ����Ĥ���ɸ��إå��ե�����:
\code{<stdio.h>}�� \code{<string.h>}�� \code{<errno.h>}�������
\code{<stdlib.h>} �򥤥󥯥롼�ɤ��Ƥ��ޤ���
��ԤΥإå��ե����뤬�����ƥ��ˤʤ���С�\code{"Python.h"} ��
�ؿ� \cfunction{malloc()}��\cfunction{free()} ����� 
\cfunction{realloc()} ��ľ��������ޤ���

���˥ե�������ɲä������Ƥϡ�Python �� \samp{spam.system(\var{string})}
��ɾ������ݤ˸ƤӽФ���뤳�Ȥˤʤ� C �ؿ��Ǥ�
(���δؿ���ǽ�Ū�ˤɤΤ褦�˸ƤӽФ����ϡ���Ǥ����狼��ޤ�):

\begin{verbatim}
static PyObject *
spam_system(PyObject *self, PyObject *args)
{
    const char *command;
    int sts;

    if (!PyArg_ParseTuple(args, "s", &command))
        return NULL;
    sts = system(command);
    return Py_BuildValue("i", sts);
}
\end{verbatim}

�����Ǥϡ�Python �ΰ����ꥹ�� (�㤨�С�ñ��μ� \code{"ls -l"}) 
���� C �ؿ����Ϥ������ˤ��Τޤ��Ѵ����Ƥ��ޤ���
C �ؿ��Ͼ����Ĥΰ�����������ص�Ū�� \var{self} ����� \var{args}
�ȸƤФ�ޤ���

\var{self} ������ C �ؿ��� Python �δؿ��ǤϤʤ��Ȥ߹��ߥ᥽�å�
��������Ƥ�����ˤΤ߻Ȥ��ޤ���������Ǥϥ᥽�åɤǤϤʤ�
�ؿ���������Ƥ���Τǡ� \var{self} �Ͼ�� \NULL{} �ݥ��󥿤ˤʤ�ޤ���
(����ϡ����󥿥ץ꥿����Ĥΰۤʤ������ C �ؿ������򤷤ʤ��Ƥ�褯
���뤿��Ǥ���)

\var{args} �����ϡ����������ä� Python ���ץ륪�֥������Ȥؤ�
�ݥ��󥿤ˤʤ�ޤ������ץ���γ����Ǥϡ��ƤӽФ��κݤΰ����ꥹ�Ȥ�
������ư������б����ޤ��������� Python ���֥������ȤǤ� --- 
C �ؿ��ǰ�����ȤäƲ�����Ԥ��ˤϡ����֥������Ȥ��� C ���ͤ�
�Ѵ����ͤФʤ�ޤ���Python API �δؿ� \cfunction{PyArg_ParseTuple()}
�ϰ����η�������å�����C ���ͤ��Ѵ����ޤ���
\cfunction{PyArg_ParseTuple()} �ϥƥ�ץ졼��ʸ�����Ȥäơ�
�������֥������Ȥη��ȡ��Ѵ����줿�ͤ������ C �ѿ��η���Ƚ�̤��ޤ���
����ˤĤ��Ƥϸ�Ǿܤ����������ޤ���

\cfunction{PyArg_ParseTuple()} �ϡ����Ƥΰ�����������������äƤ��ơ�
���ɥ쥹�Ϥ����줿���ѿ��˳ư������Ǥ���¸�����Ȥ��˿� (�󥼥�) ��
�֤��ޤ������δؿ��������ʰ����ꥹ�Ȥ��Ϥ��ȵ� (����) ���֤��ޤ���
��Ԥξ�硢�ؿ���Ŭ�ڤ��㳰�����Ф���Τǡ��ƤӽФ�¦��
(��ˤ⤢��褦��) ������\NULL{} ���֤��褦�ˤ��Ƥ���������


\section{��־���: ���顼���㳰
         \label{errors}}

Python ���󥿥ץ꥿���Τ��̤��ơ���Ĥν��פʼ���᤬����ޤ�:
����ϡ��ؿ��������˼��Ԥ�����硢�㳰���֤򥻥åȤ��ơ�
���顼�򼨤��� (�̾�� \NULL{} �ݥ���) ���֤��ͤФʤ�ʤ���
�Ȥ������ȤǤ���
�㳰�ϥ��󥿥ץ꥿�����Ū�ʥ������Х��ѿ�����¸����ޤ�;
�����ͤ� \NULL{} �ξ�硢�㳰�ϲ��ⵯ���Ƥ��ʤ����Ȥˤʤ�ޤ���
����Υ������Х��ѿ��ˤϡ��㳰�� ``��°�� (associated value)''
(\keyword{raise} ʸ���������) ������ޤ���
�軰���ͤˤϡ����顼��ȯ������ Python ����������ä�����
�����å��ȥ졼���Хå� (stack traceback) ������ޤ���
�����λ��Ĥ��ѿ��ϡ����줾�� Python ���ѿ�
\code{sys.exc_type}�� \code{sys.exc_value} �����
\code{sys.exc_traceback} �������� C ���ѿ��Ǥ�
(\citetitle[../lib/lib.html]{Python �饤�֥���ե����}
��\module{sys} �⥸�塼��˴ؤ�����򻲾Ȥ��Ƥ���������)
���顼���ɤΤ褦�˼����Ϥ���뤫�����򤹤�ˤϡ��������ѿ���
�Ĥ��Ƥ褯�ΤäƤ������Ȥ����פǤ���

Python API �Ǥϡ��͡��ʷ����㳰�򥻥åȤ��뤿��δؿ��򤤤��Ĥ�
������Ƥ��ޤ���

��äȤ�褯�Ѥ�����Τ�\cfunction{PyErr_SetString()} �Ǥ���
�������㳰���֥������Ȥ� C ʸ����Ǥ����㳰���֥������Ȥ�
�̾\cdata{PyExc_ZeroDivisionError} �Τ褦������ѤߤΥ��֥�������
�Ǥ���
C ʸ����ϥ��顼�θ����򼨤���Python ʸ���󥪥֥������Ȥ��Ѵ������
�㳰�� ``��°��'' ����¸����ޤ���

�⤦���ͭ�Ѥʴؿ��Ȥ���\cfunction{PyErr_SetFromErrno()} ������ޤ���
���δؿ��ϰ������㳰������Ȥꡢ��°�ͤϥ������Х��ѿ� \cdata{errno}
���鹽�ۤ��ޤ�����äȤ�����Ū�ʴؿ���\cfunction{PyErr_SetObject()} �ǡ�
��ĤΥ��֥������ȡ��㳰����°�ͤ�����ˤȤ�ޤ��������ؿ���
�Ϥ����֥������Ȥˤ�\cfunction{Py_INCREF()} ��Ȥ�ɬ�פϤ���ޤ���

�㳰�����åȤ���Ƥ��뤫�ɤ����ϡ�\cfunction{PyErr_Occurred()} 
��Ȥä����˲�Ū��Ĵ�٤��ޤ������δؿ��ϸ��ߤ��㳰���֥������Ȥ�
�֤��ޤ����㳰��ȯ�����Ƥ��ʤ����ˤ� \NULL{} ���֤��ޤ���
�̾�ϡ��ؿ�������ͤ��饨�顼��ȯ����������Ƚ�̤Ǥ���Ϥ��ʤΤǡ�
\cfunction{PyErr_Occurred()} ��ƤӽФ�ɬ�פϤ���ޤ���

�ؿ�\var{g} ��ƤӽФ�\var{f} �������Ԥδؿ��θƤӽФ��˼��Ԥ������Ȥ�
���Ф���ȡ�\var{f} ���Τϥ��顼�� (����� \NULL{} �� \code{-1})
���֤��ͤФʤ�ޤ��󡣤�������\cfunction{PyErr_*()} �ؿ�����
�����줫��ƤӽФ�ɬ�פ� \emph{����ޤ���} --- �ʤ��ʤ顢\var{g}
�����Ǥ˸ƤӽФ��Ƥ��뤫��Ǥ���������\var{f} ��ƤӽФ��������ɤ�
���顼�򼨤��ͤ�\emph{�����ƤӽФ���������} ���֤����Ȥˤʤ�ޤ�����
Ʊ�ͤ�\cfunction{PyErr_*()} ��\emph{�ƤӽФ��ޤ���}�� �ʲ�Ʊ�ͤ�
³���ޤ� --- ���顼�κǤ�ܤ��������ϡ��ǽ�˥��顼�򸡽Ф���
�ؿ������Ǥ���𤷤Ƥ��뤫��Ǥ������顼�� Python ���󥿥ץ꥿��
�ᥤ��롼�פ���ã����ȡ����߼¹���� Python �����ɤϰ�����
���� Python �ץ�����ޤ����ꤷ���㳰�ϥ�ɥ��õ���Ф����Ȥ��ޤ���

(�⥸�塼�뤬\cfunction{PyErr_*()} �ؿ���⤦���ٸƤӽФ��ơ����ܺ٤�
���顼��å��������󶡤���褦�ʾ���������ޤ������Τ褦�ʾ����Ǥ�
�������٤��Ǥ����ȤϤ���������Ū�ʵ�§�Ȥ��Ƥϡ�\cfunction{PyErr_*()} 
���٤�ƤӽФ�ɬ�פϤʤ����Ȥ⤹��Х��顼�θ����˴ؤ�������
������̤ˤʤ꤬���Ǥ�: ����ˤ�ꡢ�ۤȤ�ɤ����͡�����ͳ����
���Ԥ��뤫�⤷��ޤ���)

����ؿ��ƤӽФ��Ǥν����μ��Ԥˤ�äƥ��åȤ��줿�㳰��̵�뤹��ˤϡ�
\cfunction{PyErr_Clear()} ��ƤӽФ����㳰���֤�����Ū�˾õ�
���ʤ��ƤϤʤ�ޤ���
���顼�򥤥󥿥ץ꥿�ˤ��Ϥ������ʤ��������� (����¾�κ�Ȥ�Ԥä��ꡢ
���ⵯ����ʤ��ä����Τ褦�˸���������褦��) ���顼����������
�Ԥ����ˤΤߡ�\cfunction{PyErr_Clear()} ��ƤӽФ��褦�ˤ��٤��Ǥ���

\cfunction{malloc()} �θƤӽФ����Ԥϡ�����㳰�ˤ��ʤ��Ƥ�
�ʤ�ޤ��� --- \cfunction{malloc()} (�ޤ��� \cfunction{realloc()})
��ľ�ܸƤӽФ��Ƥ��륳���ɤϡ�\cfunction{PyErr_NoMemory()} ��
�ƤӽФ��ơ����Ԥ򼨤��ͤ��֤��ͤФʤ�ޤ��󡣥��֥������Ȥ�
�����������Ƥδؿ� (�㤨�� \cfunction{PyInt_FromLong()}) ��
\cfunction{PyErr_NoMemory()} �θƤӽФ���Ѥޤ��Ƥ��ޤ��Τǡ�
���ε�§���ط�����Τ�ľ�� \cfunction{malloc()} ��ƤӽФ�
�����ɤ����Ǥ���

�ޤ���\cfunction{PyArg_ParseTuple()} �Ȥ������פ��㳰������ơ�
�����ξ��֥����ɤ��֤��ؿ��Ϥ����Ƥ���\UNIX{} �Υ����ƥॳ����
��Ʊ���������������������ݤˤϥ����ޤ��������ͤ��֤���
���Ԥ������ˤ� \code{-1} ���֤��ޤ���

�Ǹ�ˡ����顼ɸ���ͤ��֤��ݤˡ�(���顼��ȯ������ޤǤ˴���
�������Ƥ��ޤä����֥������Ȥ��Ф���\cfunction{Py_XDECREF()} ��
\cfunction{Py_DECREF()} ��ƤӽФ���) ���߽��������տ���
�ԤäƤ�������!

�ɤ��㳰���֤���������ϡ��桼���˴����ˤ���ͤ��ޤ���
\cdata{PyExc_ZeroDivisionError} �Τ褦�ˡ����Ƥ��Ȥ߹��ߤ� Python 
�㳰�ˤ��б���������Ѥߤ� C ���֥������Ȥ����ꡢľ�����ѤǤ��ޤ���
��������㳰������ϸ����Ԥ�ͤФʤ�ޤ��� --- 
�ե����뤬�����ʤ��ä����Ȥ�ɽ���Τ�\cdata{PyExc_TypeError} 
��Ȥä���Ϥ��ʤ��Ǥ������� (���ξ��Ϥ����餯\cdata{PyExc_IOError} 
�����ˤ��٤��Ǥ��礦)��
�����ꥹ�Ȥ����꤬������ˤϡ�\cfunction{PyArg_ParseTuple()} 
�Ϥ����Ƥ� \cdata{PyExc_TypeError} �����Ф��ޤ���
�������ͤ�������ϰϤ�Ķ���Ƥ����ꡢ����¾���������٤�����������
�ʤ��ä����ˤϡ�\cdata{PyExc_ValueError} ��Ŭ�ڤǤ���

�⥸�塼���ͭ�ο������㳰������Ǥ��ޤ����������ˤϡ��̾��
�ե��������Ƭ��ʬ����Ū�ʥ��֥��������ѿ��������Ԥ��ޤ�:

\begin{verbatim}
static PyObject *SpamError;
\end{verbatim}

�����ơ��⥸�塼��ν�����ؿ� (\cfunction{initspam()})
����ǡ��㳰���֥������Ȥ�Ȥäƽ�������ޤ� (�����Ǥ�
���顼�����å����ά���Ƥ��ޤ�):

\begin{verbatim}
PyMODINIT_FUNC
initspam(void)
{
    PyObject *m;

    m = Py_InitModule("spam", SpamMethods);

    SpamError = PyErr_NewException("spam.error", NULL, NULL);
    Py_INCREF(SpamError);
    PyModule_AddObject(m, "error", SpamError);
}
\end{verbatim}

Python ��٥�Ǥ��㳰���֥������Ȥ�̾���� \exception{spam.error}
�ˤʤ뤳�Ȥ����դ��Ƥ��������� \cfunction{PyErr_NewException()} 
�ؿ��ϡ�\citetitle[../lib/lib.html]{Python �饤�֥���ե����} 
�� ``�Ȥ߹����㳰'' ����˽Ҥ٤��Ƥ���\exception{Exception} 
���饹����쥯�饹�˻����㳰���饹������Ǥ��ޤ� 
(\NULL �������¾�Υ��饹���Ϥ��������̤Ǥ�)��

\cdata{SpamError} �ѿ��ϡ��������������줿�㳰���饹�ؤλ��Ȥ�
�ݻ����뤳�Ȥˤ����դ��Ƥ�������; ����ϰտ�Ū�ʻ��ͤǤ�!
�����Υ����ɤ��㳰���֥������Ȥ�⥸�塼�뤫�����Ǥ��뤿�ᡢ
�⥸�塼�뤫�鿷���˺��������㳰���饹�������ʤ��ʤꡢ
\cdata{SpamError} ���֤鲼����ݥ��� (dangling pointer)
�ˤʤäƤ��ޤ�ʤ��褦�ˤ��뤿��ˡ����饹���Ф��뻲�Ȥ��ͭ����
�����ͤФʤ�ޤ���
�⤷\cdata{SpamError} ���֤鲼����ݥ��󥿤ˤʤäƤ��ޤ��ȡ�
C �����ɤ��㳰�����Ф��褦�Ȥ����Ȥ��˥�������פ�տޤ��ʤ������Ѥ�
�������������Ȥ�����ޤ���

������ˤ��롢�ؿ�������ͷ��� PyMODINIT_FUNC �λȤ�
��ˡ�ˤĤ��Ƥϸ�ǵ������ޤ���

\section{������
         \label{backToExample}}

��ۤɤδؿ���������ȡ����٤ϰʲ��μ¹�ʸ������Ǥ���Ϥ��Ǥ�:

\begin{verbatim}
    if (!PyArg_ParseTuple(args, "s", &command))
        return NULL;
\end{verbatim}

���μ¹�ʸ�ϡ�\cfunction{PyArg_ParseTuple()} �����åȤ����㳰
�ˤ�äơ������ꥹ�Ȥ˲��餫�Υ��顼���������Ȥ���\NULL{} 
(���֥������ȤؤΥݥ��󥿤��֤������פδؿ��ˤ����륨�顼ɸ����) 
���֤��ޤ������顼�Ǥʤ���С������Ȥ���Ϳ����ʸ�����ͤϥ�������
���ѿ� \cdata{command} �˥��ԡ�����Ƥ��ޤ���
�������ϥݥ��������Ǥ��ꡢ�ݥ��󥿤��ؤ��Ƥ���ʸ������Ф���
�ѹ����Ԥ���Ȥ����ꤵ��Ƥ��ޤ��� (���äơ�ɸ�� C �Ǥϡ�
�ѿ� \cdata{command} �� \samp{const char* command} �Ȥ���
Ŭ�ڤ�������ͤФʤ�ޤ���)��

����ʸ�Ǥϡ�\cfunction{PyArg_ParseTuple()} ������ʸ�����
�Ϥ��� \UNIX{} �ؿ� \cfunction{system()} ��ƤӽФ��Ƥ��ޤ�:

\begin{verbatim}
    sts = system(command);
\end{verbatim}

\function{spam.system()} �� \cdata{sts} �� Python ���֥�������
�Ȥ����֤��ͤФʤ�ޤ��󡣤���ˤϡ�\cfunction{PyArg_ParseTuple()}
�εդȤ⤤���٤��ؿ�\cfunction{Py_BuildValue()} ��Ȥ��ޤ�:
\cfunction{Py_BuildValue()} �Ͻ񼰲�ʸ�����Ǥ�դο��� C ���ͤ�
�����ˤȤꡢ������ Python ���֥������Ȥ��֤��ޤ���
\cfunction{Py_BuildValue()} �˴ؤ���ܤ�������ϸ�Ǽ����ޤ���

\begin{verbatim}
    return Py_BuildValue("i", sts);
\end{verbatim}

��ξ��Ǥϡ�\cfunction{Py_BuildValue()} ���������֥������Ȥ�
�֤��ޤ���(�����������Ǥ��顢 Python �ˤ����Ƥϥҡ��׾��
���֥������ȤʤΤǤ�! )

����ͭ�Ѥ��ͤ��֤��ʤ��ؿ� (\ctype{void} ���֤��ؿ�) ��
�б����� Python �δؿ���\code{None} ���֤��ͤФʤ�ޤ���
�ؿ��� \code{None} ���֤�����ˤϡ��ʲ��Τ褦�ʴ��Ѷ��Ȥ��ޤ�
(���δ��Ѷ��\csimplemacro{Py_RETURN_NONE} �ޥ����˼���
����Ƥ��ޤ�):

\begin{verbatim}
    Py_INCREF(Py_None);
    return Py_None;
\end{verbatim}

\cdata{Py_None} ���ü�� Pyhton ���֥������ȤǤ��� \code{None} ��
�б����� C �Ǥ�̾���Ǥ�������ޤǸ��Ƥ����褦�ˤۤȤ�ɤΥ���ƥ�����
�� ``���顼'' ���̣���� \NULL{} �ݥ��󥿤Ȥϰ㤤��\code{None} �Ͻ���
Python �Υ��֥������ȤǤ���


\section{�⥸�塼��Υ᥽�åɥơ��֥�Ƚ�����ؿ�\label{methodTable}}

���ơ�������«�����褦�ˡ�\cfunction{spam_system()} Python �ץ������
����ɤ���äƸƤӽФ����򤳤줫�鼨���ޤ����ޤ��ϡ��ؿ�̾�ȥ��ɥ쥹��
``�᥽�åɥơ��֥� (method table)'' ����󤹤�ɬ�פ�����ޤ�:

\begin{verbatim}
static PyMethodDef SpamMethods[] = {
    ...
    {"system",  spam_system, METH_VARARGS,
     "Execute a shell command."},
    ...
    {NULL, NULL, 0, NULL}        /* Sentinel */
};
\end{verbatim}

�ꥹ�����Ǥλ����ܤΥ���ȥ� (\samp{METH_VARARGS}) �����դ��Ƥ���������
���Υ���ȥ�ϡ�C �ؿ����Ȥ��ƤӽФ�����򥤥󥿥ץ꥿�˶����뤿���
�ե饰�Ǥ����̾盧���ͤ�\samp{METH_VARARGS} ��
\samp{METH_VARARGS | METH_KEYWORDS} �ΤϤ��Ǥ�; \code{0} 
�ϵ켰��\cfunction{PyArg_ParseTuple()} ���Ѳ������Ȥ��뤳�Ȥ�
��̣���ޤ���

\samp{METH_VARARGS} ������Ȥ���硢C �ؿ��ϡ�Python ��٥�Ǥΰ�����
\cfunction{PyArg_ParseTuple()} �������Ǥ��륿�ץ�η������Ϥ������
�����ꤷ�ʤ���Фʤ�ޤ���; ���δؿ��ˤĤ��Ƥξܺ٤ϲ����������ޤ���

�ؿ��˥�����ɰ������Ϥ���뤳�ȤˤʤäƤ���Τʤ顢
�軰�ե�����ɤ�\constant{METH_KEYWORDS} �ӥåȤ򥻥åȤǤ��ޤ���
���ξ�硢C �ؿ����軰������ \samp{PyObject *} ���������褦��
���ͤФʤ�ޤ��󡣤��Υ��֥������Ȥϡ�������ɰ����μ����
�ʤ�ޤ������������ؿ��ǰ������᤹��ˤϡ�
\cfunction{PyArg_ParseTupleAndKeywords()} ��ȤäƤ���������

�᥽�åɥơ��֥�ϡ��⥸�塼��ν�����ؿ���ǥ��󥿥ץ꥿��
�Ϥ��ͤФʤ�ޤ��󡣽�����ؿ��ϥ⥸�塼���̾���� \var{name}
�Ȥ����Ȥ��� \cfunction{init\var{name}()} �Ȥ���̾���Ǥʤ����
�ʤ餺���⥸�塼��ե���������������Ƥ����ΤΤ�����ͣ���
��\keyword{static} ���ǤǤʤ���Фʤ�ޤ���:

\begin{verbatim}
PyMODINIT_FUNC
initspam(void)
{
    (void) Py_InitModule("spam", SpamMethods);
}
\end{verbatim}

PyMODINIT_FUNC �ϴؿ�������ͤ� \code{void} �ˤʤ�褦���������
�ץ�åȥե��������ɬ�פȤ���롢��ͭ�Υ����� (linkage declaration)
��������뤳�ȡ������ \Cpp{} �ξ��ˤϴؿ��� \code{extern "C"} ��
������뤳�Ȥ����դ��Ƥ���������

Python �ץ�����ब�⥸�塼�� \module{spam} ����� import
����Ȥ���\cfunction{initspam()} ���ƤӽФ���ޤ���
(Python �������ߤ˴ؤ��륳���Ȥϲ����򻲾Ȥ��Ƥ���������)
\cfunction{initspam()} �� \cfunction{Py_InitModule()} ��ƤӽФ���
``�⥸�塼�륪�֥�������'' �������� (���֥������Ȥ�\code{"spam"} ��
�����Ȥ��Ƽ��� \code{sys.modules} ����������ޤ�)����������Ȥ���
Ϳ�����᥽�åɥơ��֥� (\ctype{PyMethodDef} ��¤�Τ�����) �ξ����
��Ť��ơ��Ȥ߹��ߴؿ����֥������Ȥ򿷤��ʥ⥸�塼����������Ƥ����ޤ���
\cfunction{Py_InitModule()} �ϡ����餬�������� (�����ʳ��ǤϤޤ�̤���Ѥ�) 
�⥸�塼�륪�֥������ȤؤΥݥ��󥿤��֤��ޤ���
\cfunction{Py_InitModule()} �ϡ��⥸�塼�����­�˽�����Ǥ��ʤ��ä���硢
��̿Ū���顼�����Ǥ��뤿�ᡢ���δؿ��θƤӽФ�¦�����顼������å�����
ɬ�פϤ���ޤ���

Python ����������ˤϡ�\cdata{_PyImport_Inittab} �ơ��֥��
����ȥ���� \cfunction{initspam()} ���ʤ��¤ꡢ\cfunction{initspam()}
�ϼ�ưŪ�ˤϸƤӽФ���ޤ��󡣤���������褹��Ǥ��ñ����ˡ�ϡ�
\cfunction{Py_Initialize()} �� \cfunction{PyMac_Initialize()} ��
�ƤӽФ������ \cfunction{initspam()} ��ľ�ܸƤӽФ���
��Ū�˥�󥯤��Ƥ������⥸�塼�����Ū�˽�������Ƥ��ޤ��Ȥ�����ΤǤ�:

\begin{verbatim}
int
main(int argc, char *argv[])
{
    /* Python ���󥿥ץ꥿�� argv[0] ���Ϥ� */
    Py_SetProgramName(argv[0]);

    /* Python ���󥿥ץ꥿���������롣ɬ��ɬ�ס� */
    Py_Initialize();

    /* ��Ū�⥸�塼����ɲä��� */
    initspam();
\end{verbatim}

Python ����������ʪ��� \file{Demo/embed/demo.c} �ե�������
���㤬����ޤ���

\note{ñ��Υץ������� (�ޤ���
\cfunction{fork()} ��� \cfunction{exec()} ���������Ƥ��ʤ�����)
�ˤ�����ʣ���Υ��󥿥ץ꥿�ˤ����ơ� \code{sys.module} ����
����ȥ�������꿷���ʥ���ѥ���Ѥߥ⥸�塼��� import 
�����ꤹ��ȡ���ĥ�⥸�塼��ˤ�äƤ�����������뤳�Ȥ�����ޤ���
��ĥ�⥸�塼��κ�Ԥϡ������ǡ�����¤����������ݤˤϤ褯�褯
�ѿ����٤��Ǥ����ޤ���\function{reload()} �ؿ����ĥ�⥸�塼���
�Ф������ѤǤ������ξ��ϥ⥸�塼�������ؿ� (\cfunction{initspam()})
�ϸƤӽФ���ޤ������⥸�塼�뤬ưŪ�˥����ɲ�ǽ�ʥ��֥������ȥե�����
(\UNIX{}�Ǥ� \file{.so}��Windows �Ǥ� \file{.dll}) �����ɤ߽Ф��줿
���ˤϥ⥸�塼��ե��������ɤ߹��ߤ��ʤ��Τ����դ��Ƥ���������}

���¼�Ū�ʥ⥸�塼����ϡ�Python ����������ʪ��
\file{Modules/xxmodule.c} �Ȥ���̾�������äƤ��ޤ���
���Υե�����ϥƥ�ץ졼�ȤȤ��Ƥ����ѤǤ��ޤ�����ñ����Ȥ��Ƥ�
�ɤ�ޤ�������������ʪ�� Windows �˥��󥹥ȡ��뤵�줿 Python �����äƤ���
\program{modulator.py} �Ǥϡ���ĥ�⥸�塼��Ǽ������ʤ���Фʤ�ʤ�
�ؿ��䥪�֥������Ȥ��������������ʬ�����ƺ������뤿��Υƥ�ץ졼��
�������Ǥ���褦�ʡ���ñ�ʥ���ե�����桼�����󥿥ե�������
�󶡤��Ƥ��ޤ���
���Υ�����ץȤ�\file{Tools/modulator/} �ǥ��쥯�ȥ�ˤ���ޤ�;
�ܤ����ϥǥ��쥯�ȥ���� \file{README} �ե�����򻲾Ȥ��Ƥ���������


\section{����ѥ���ȥ��
         \label{compilation}}

��������ĥ�⥸�塼���Ȥ���褦�ˤʤ�ޤǡ��ޤ���Ĥκ��: 
����ѥ���ȡ�Python �����ƥ�ؤΥ�󥯡����ĤäƤ��ޤ���
ưŪ�ɤ߹��� (dynamic loading) ��ȤäƤ���Τʤ顢��Ȥξܺ٤�
��ʬ�Υ����ƥब�ȤäƤ���ưŪ�ɤ߹��ߤη����ˤ�ä��Ѥ�뤫��
����ޤ���; �ܤ����ϡ���ĥ�⥸�塼��Υӥ�ɤ˴ؤ���� 
(\ref{building} ��) �䡢Windows �ˤ�����ӥ�ɤ˴ط������ɲþ���ξ�
(\ref{building-on-windows} ��) �򻲾Ȥ��Ƥ���������


ưŪ�ɤ߹��ߤ�Ȥ��ʤ��ä��ꡢ�⥸�塼����� Python ���󥿥ץ꥿��
�����ˤ��Ƥ����������ˤϡ����󥿥ץ꥿�Υӥ��������ѹ����ƺƥӥ��
���ʤ���Фʤ�ʤ��ʤ�Ǥ��礦��\UNIX{}�Ǥϡ������ʤ��Ȥˤ��κ�Ȥ�
�ȤƤ�ñ��Ǥ�: ñ�˼���Υ⥸�塼��ե����� (�㤨��
\file{spammodule.c} ) ��Ÿ����������������ʪ�� \file{Modules/} 
�ǥ��쥯�ȥ���֤��� \file{Modules/Setup.local} �˼�ʬ�Υե������
��������ʲ��ΰ��:

\begin{verbatim}
spam spammodule.o
\end{verbatim}

���ɲä��ơ��ȥåץ�٥�Υǥ��쥯�ȥ�� \program{make} ��¹Ԥ��ơ�
���󥿥ץ꥿��ƥӥ�ɤ�������Ǥ���
\file{Modules/} ���֥ǥ��쥯�ȥ�Ǥ� \program{make} ��¹ԤǤ��ޤ�����
����ä� `\program{make} Makefile' ��¹Ԥ��� \file{Makefile}
���ƥӥ�ɤ��Ƥ����ʤ���Фʤ��ޤ���(���κ�Ȥ�
\file{Setup} �ե�������ѹ����뤿�Ӥ�ɬ�פǤ���)

�⥸�塼�뤬�̤Υ饤�֥��ȥ�󥯤���Ƥ���ɬ�פ������硢
�饤�֥�������ե���������Ǥ��ޤ����㤨�аʲ��Τ褦�ˤ��ޤ�:

\begin{verbatim}
spam spammodule.o -lX11
\end{verbatim}

\section{C ���� Python �ؿ���ƤӽФ�
         \label{callingPython}}

����ޤǤϡ�Python ����� C �ؿ��θƤӽФ��˽������֤���
�Ҥ٤Ƥ��ޤ������Ȥ����Ǥ��ε�:  C ����� Python �ؿ��θƤӽФ�
��ޤ�ͭ�ѤǤ���
�Ȥ�櫓�������� ``������Хå�'' �ؿ��򥵥ݡ��Ȥ���褦��
�饤�֥����������ݤˤϤ��ε�ǽ�������Ǥ���
���� C ���󥿥ե�������������Хå������Ѥ��Ƥ����硢
Ʊ���ε�ǽ���󶡤��� Python �����ɤǤϡ����Ф��� Python �ץ�����ޤ�
������Хå��������󶡤���ɬ�פ�����ޤ�; ���ΤȤ������Ǥϡ�
C �ǽ񤫤줿������Хå��ؿ����� Python �ǽ񤫤줿������ѥå��ؿ�
��ƤӽФ��褦�ˤ���ɬ�פ�����Ǥ��礦��
�������¾�����Ӥ�ͤ����ޤ���

�����ʤ��Ȥˡ�Python ���󥿥ץ꥿�ϴ�ñ�˺Ƶ��ƤӽФ��Ǥ���
Python �ؿ���ƤӽФ������ɸ�।�󥿥ե������⤢��ޤ���
(Python �ѡ��������������ʸ����ȤäƸƤӽФ���ˡ�ˤĤ���
���⤹��Ĥ��Ϥ���ޤ��� --- ������ˡ�˶�̣������ʤ顢
Python �����������ɤ� \file{Python/pythonmain.c} �ˤ��롢
���ޥ�ɥ饤�󥪥ץ����\programopt{-c} �μ����򸫤Ƥ�������)

Python �ؿ��θƤӽФ��ϴ�ñ�Ǥ����ޤ���C �Υ����ɤ��Ф���
������Хå�����Ͽ���褦�Ȥ��� Python �ץ������ϡ����餫����ˡ��
Python �δؿ����֥������Ȥ��Ϥ��ͤФʤ�ޤ��󡣤��Τ���ˡ�
������Хå���Ͽ�ؿ� (�ޤ��Ϥ���¾�Υ��󥿥ե�����) ����
���ͤФʤ�ޤ��󡣤��Υ�����Хå���Ͽ�ؿ����ƤӽФ��줿�ݤˡ�
�����Ϥ��줿 Python �ؿ����֥������ȤؤΥݥ��󥿤򥰥����Х��ѿ��� --- 
���뤤�ϡ��ɤ���Ŭ�ڤʾ��� --- ��¸���ޤ�
(�ؿ����֥������Ȥ�\cfunction{Py_INCREF()} ����褦�褯���դ���
��������!)���㤨�С��ʲ��Τ褦�ʴؿ����⥸�塼��ΰ����ˤʤä�
���뤳�ȤǤ��礦:

\begin{verbatim}
static PyObject *my_callback = NULL;

static PyObject *
my_set_callback(PyObject *dummy, PyObject *args)
{
    PyObject *result = NULL;
    PyObject *temp;

    if (PyArg_ParseTuple(args, "O:set_callback", &temp)) {
        if (!PyCallable_Check(temp)) {
            PyErr_SetString(PyExc_TypeError, "parameter must be callable");
            return NULL;
        }
        Py_XINCREF(temp);         /* �����ʥ�����Хå��ؤλ��Ȥ��ɲ� */
        Py_XDECREF(my_callback);  /* �����Υ�����Хå���ΤƤ� */
        my_callback = temp;       /* �����ʥ�����Хå��򵭲� */
        /* "None" ���֤��ݤ��귿�� */
        Py_INCREF(Py_None);
        result = Py_None;
    }
    return result;
}
\end{verbatim}

���δؿ���\constant{METH_VARARGS} �ե饰��Ȥäƥ��󥿥ץ꥿��
��Ͽ���ͤФʤ�ޤ���; \constant{METH_VARARGS} �ե饰�ˤĤ��Ƥϡ�
\ref{methodTable} �ᡢ ``�⥸�塼��Υ᥽�åɥơ��֥�Ƚ�����ؿ�''
���������Ƥ��ޤ���
\cfunction{PyArg_ParseTuple()} �ؿ��Ȥ��ΰ����ˤĤ��Ƥϡ�
\ref{parseTuple} �ᡢ ``��ĥ�⥸�塼��ؿ��ǤΥѥ�᥿Ÿ��''
�˵��Ҥ��Ƥ��ޤ���

\cfunction{Py_XINCREF()} �����\cfunction{Py_XDECREF()} �ϡ�
���֥������Ȥ��Ф��뻲�ȥ�����Ȥ򥤥󥯥����/�ǥ�����Ȥ���
����Υޥ����ǡ�\NULL{} �ݥ��󥿤��Ϥ���Ƥ���������Ǥ���
�����Ǥ� (�ȤϤ��������ή��Ǥ�\var{temp} ��\NULL{} �ˤʤ뤳�Ȥ�
����ޤ���)��
�����Υޥ����Ȼ��ȥ�����ȤˤĤ��Ƥϡ�\ref{refcounts}
�ᡢ ``���ȥ������'' ���������Ƥ��ޤ���

���θ塢������Хå��ؿ���ƤӽФ������褿�顢C �ؿ�
\cfunction{PyEval_CallObject()}\ttindex{PyEval_CallObject()} 
��ƤӽФ��ޤ���
���δؿ��ˤ���Ĥΰ���: Python �ؿ��� Python �ؿ��ΰ����ꥹ�Ȥ����ꡢ
�������Ǥ�դ� Python ���֥������Ȥ�ɽ���ݥ��󥿷��Ǥ���
�����ꥹ�ȤϾ�˥��ץ륪�֥������ȤǤʤ���Фʤ餺������Ĺ����
�����ο��ˤʤ�ޤ���Python �ؿ�������ʤ��ǸƤӽФ��Τʤ�
���Υ��ץ���Ϥ��ޤ�; ñ��ΰ����Ǵؿ���ƤӽФ��Τʤ顢
ñ���� (singleton) �Υ��ץ���Ϥ��ޤ���
\cfunction{Py_BuildValue()} �ν񼰲�ʸ������ˡ������Ĥޤ���
��İʾ�ν񼰲������ɤ����ä��ݳ�̤������硢���δؿ���
���ץ���֤��ޤ����ʲ�����򼨤��ޤ�:

\begin{verbatim}
    int arg;
    PyObject *arglist;
    PyObject *result;
    ...
    arg = 123;
    ...
    /* �����ǥ�����Хå���Ƥ� */
    arglist = Py_BuildValue("(i)", arg);
    result = PyEval_CallObject(my_callback, arglist);
    Py_DECREF(arglist);
\end{verbatim}

\cfunction{PyEval_CallObject()} �� Python ���֥������ȤؤΥݥ��󥿤�
�֤��ޤ�; ����� Python �ؿ����������ͤˤʤ�ޤ���
\cfunction{PyEval_CallObject()} �ϡ��������Ф���
``���ȥ��������Ω (reference-count-neutral)'' �Ǥ���
�����Ǥϥ��ץ���������ư����ꥹ�ȤȤ����󶡤��Ƥ��ꡢ����
���ץ�ϸƤӽФ�ľ��� \cfunction{Py_DECREF()} ���Ƥ��ޤ���

\cfunction{PyEval_CallObject()} �� ``������'' ����ͤ��֤��ޤ�: 
����ͤ�ɽ�����֥������ȤϿ����ʥ��֥������Ȥ�����¸�Υ��֥������Ȥ�
���ȥ�����Ȥ򥤥󥯥���Ȥ�����ΤǤ������äơ����Υ��֥������Ȥ�
�������Х��ѿ�����¸�������ΤǤʤ������ꡢ
���Ȥ���������ͤ˶�̣���ʤ��Ƥ� (�ष���������Ǥ���Фʤ�����!)
������������ˡ������ͥ��֥������Ȥ� \cfunction{Py_DECREF()} 
���ʤ���Фʤ�ޤ���

�ȤϤ���������ͤ�\cfunction{Py_DECREF()} �������ˤϡ��ͤ� \NULL
�Ǥʤ��������å����Ƥ������Ȥ����פǤ����⤷ \NULL �ʤ顢�ƤӽФ���
Python �ؿ����㳰�����Ф��ƽ�λ�������Ƥ��ޤ���
\cfunction{PyEval_CallObject()} ��ƤӽФ��Ƥ��륳���ɼ��Τ�ޤ�
Python ����ƤӽФ���Ƥ���ΤǤ���С����٤� C �����ɤ���ʬ��
�ƤӽФ��Ƥ��� Python �����ɤ˥��顼ɸ���ͤ��֤��ͤФʤ�ޤ���
����ˤ�ꡢ���󥿥ץ꥿�ϥ����å��ȥ졼������Ϥ����ꡢ�㳰��
�������뤿��� Python �����ɤ�ƤӽФ�����Ǥ��ޤ���
�㳰�����Ф��Բ�ǽ���ä��ꡢ�������ʤ��Τʤ顢
\cfunction{PyErr_Clear()} ��Ƥ���㳰��õ�Ƥ����ͤФʤ�ޤ���
�㤨�аʲ��Τ褦�ˤ��ޤ�:

\begin{verbatim}
    if (result == NULL)
        return NULL; /* ���顼���֤� */
    ...use result...
    Py_DECREF(result); 
\end{verbatim}

Python ������Хå��ؿ���ɤ�ʥ��󥿥ե������ˤ��������ˤ�äƤϡ�
�����ꥹ�Ȥ�\cfunction{PyEval_CallObject()} ��Ϳ���ʤ����
�ʤ�ʤ����⤢��ޤ���
���륱�����Ǥϡ�������Хå��ؿ�����ꤷ���Τ�Ʊ�����󥿥ե�����
��𤷤ơ������ꥹ�Ȥ��Ϥ���Ƥ��뤫�⤷��ޤ���
�ޤ��̤Υ������Ǥϡ����������ץ���ۤ��ư����ꥹ�Ȥ��Ϥ��ͤ�
�ʤ�ʤ����⤷��ޤ��󡣤��ξ��Ǥ��ñ�ʤΤ�
\cfunction{Py_BuildValue()} ��Ƥ֤�����Ǥ���
�㤨�С������Υ��٥�ȥ����ɤ��Ϥ�������С��ʲ��Τ褦�ʥ����ɤ�
�Ȥ����Ȥˤʤ�Ǥ��礦:

\begin{verbatim}
    PyObject *arglist;
    ...
    arglist = Py_BuildValue("(l)", eventcode);
    result = PyEval_CallObject(my_callback, arglist);
    Py_DECREF(arglist);
    if (result == NULL)
        return NULL; /* ���顼���֤� */
    /* ���ˤ�äƤϤ����Ƿ�̤�Ȥ������ */
    Py_DECREF(result);
\end{verbatim}

\samp{Py_DECREF(arglist)} ���ƤӽФ���ľ�塢���顼�����å���������
�֤���Ƥ��뤳�Ȥ����դ��Ƥ�������! �ޤ�����̩�˸����С����Υ����ɤ�
�����ǤϤ���ޤ���: \cfunction{Py_BuildValue()} �ϥ�����­��
�������뤫�⤷�줺�������å����Ƥ����٤��Ǥ���


\section{��ĥ�⥸�塼��ؿ��ǤΥѥ�᥿Ÿ��
         \label{parseTuple}}

\ttindex{PyArg_ParseTuple()}

\cfunction{PyArg_ParseTuple()} �ϡ��ʲ��Τ褦���������Ƥ��ޤ�:

\begin{verbatim}
int PyArg_ParseTuple(PyObject *arg, char *format, ...);
\end{verbatim}

����\var{arg} �� C �ؿ����� Python ���Ϥ��������ꥹ�Ȥ����ä�
���ץ륪�֥������ȤǤʤ���Фʤ�ޤ���
\var{format} �����Ͻ񼰲�ʸ����ǡ�
\citetitle[../api/api.html]{Python/C API ��ե���󥹥ޥ˥奢��}
�� ``\ulink{�����β����ͤι���}{../api/arg-parsing.html}'' 
�Dz��⤵��Ƥ����ˡ�˽���ͤФʤ�ޤ���
�Ĥ�ΰ����ϡ����줾����ѿ��Υ��ɥ쥹�ǡ��񼰲�ʸ���󤫤�
��ޤ뷿�ˤʤäƤ��ʤ���Фʤ�ޤ���

\cfunction{PyArg_ParseTuple()} �� Python ¦����Ϳ����줿������
ɬ�פʷ��ˤʤäƤ��뤫Ĵ�٤�Τ��Ф���\cfunction{PyArg_ParseTuple()} 
�ϸƤӽФ��κݤ��Ϥ��줿 C �ѿ��Υ��ɥ쥹��ͭ�����ͤ���Ĥ�Ĵ��
���ʤ����Ȥ����դ��Ƥ�������: �����Ǵְ㤤���Ȥ��ȡ������ɤ�
����å��夹�뤫�⤷��ޤ��󤷡����ʤ��Ȥ�Ǥ����ʥӥåȤ�
����˾�񤭤��Ƥ��ޤ��ޤ������Ť�! 

�ƤӽФ�¦���󶡤���륪�֥������Ȥؤλ��ȤϤ��٤� \emph{����}
���� (borrowed reference) �ˤʤ�ޤ�; �����Υ��֥������Ȥλ���
������Ȥ�ǥ�����Ȥ��ƤϤʤ�ޤ���!

�ʲ��ˤ����Ĥ��θƤӽФ���򼨤��ޤ�:

\begin{verbatim}
    int ok;
    int i, j;
    long k, l;
    const char *s;
    int size;

    ok = PyArg_ParseTuple(args, ""); /* �����ʤ� */
        /* Python �ǤθƤӽФ�: f() */
\end{verbatim}

\begin{verbatim}
    ok = PyArg_ParseTuple(args, "s", &s); /* ʸ���� */
        /* Python �ǤθƤӽФ���: f('whoops!') */
\end{verbatim}

\begin{verbatim}
    ok = PyArg_ParseTuple(args, "lls", &k, &l, &s); 
        /* ��Ĥ� long ��ʸ���� */
        /* Python �ǤθƤӽФ���: f(1, 2, 'three') */
\end{verbatim}

\begin{verbatim}
    ok = PyArg_ParseTuple(args, "(ii)s#", &i, &j, &s, &size);
        /* ��Ĥ� int ��ʸ����ʸ����Υ��������֤� */
        /* Python �ǤθƤӽФ���: f((1, 2), 'three') */
\end{verbatim}

\begin{verbatim}
    {
        const char *file;
        const char *mode = "r";
        int bufsize = 0;
        ok = PyArg_ParseTuple(args, "s|si", &file, &mode, &bufsize);
        /* ʸ���󡢥��ץ����Ȥ���ʸ���󤬤⤦��Ĥ���������� */
        /* Python �ǤθƤӽФ���:
           f('spam')
           f('spam', 'w')
           f('spam', 'wb', 100000) */
    }
\end{verbatim}

\begin{verbatim}
    {
        int left, top, right, bottom, h, v;
        ok = PyArg_ParseTuple(args, "((ii)(ii))(ii)",
                 &left, &top, &right, &bottom, &h, &v);
        /* ���������ɽ������ǡ��� */
        /* Python �ǤθƤӽФ���:
           f(((0, 0), (400, 300)), (10, 10)) */
    }
\end{verbatim}

\begin{verbatim}
    {
        Py_complex c;
        ok = PyArg_ParseTuple(args, "D:myfunction", &c);
        /* ʣ�ǿ������顼ȯ�����Ѥ˴ؿ�̾����� */
        /* Python �ǤθƤӽФ���: myfunction(1+2j) */
    }
\end{verbatim}


\section{��ĥ�⥸�塼��ؿ��Υ�����ɥѥ�᥿
         \label{parseTupleAndKeywords}}

\ttindex{PyArg_ParseTupleAndKeywords()}

\cfunction{PyArg_ParseTupleAndKeywords()} 
�ϡ��ʲ��Τ褦���������Ƥ��ޤ�:

\begin{verbatim}
int PyArg_ParseTupleAndKeywords(PyObject *arg, PyObject *kwdict,
                                char *format, char *kwlist[], ...);
\end{verbatim}

\var{arg} �� \var{format} �ѥ�᥿��\cfunction{PyArg_ParseTuple()} 
�Τ�Τ�Ʊ���Ǥ���\var{kwdict} �ѥ�᥿�ϥ�����ɰ��������ä�
����ǡ� Python ��󥿥��ॷ���ƥफ���軰�ѥ�᥿�Ȥ��Ƽ������ޤ���
\var{kwlist} �ѥ�᥿�ϳƥѥ�᥿���̤��뤿���ʸ���󤫤�ʤ롢
\NULL ��ü���줿�ꥹ�ȤǤ�; �ƥѥ�᥿̾�� \var{format} ���
��������Ф��ƺ����鱦�ν�˾ȹ礵��ޤ���

���������\cfunction{PyArg_ParseTupleAndKeywords()} �Ͽ����֤���
����ʳ��ξ��ˤ�Ŭ�ڤ��㳰�����Ф��Ƶ����֤��ޤ���

\note{������ɰ�����ȤäƤ����硢���ץ������Ҥˤ��ƻȤ��ޤ���!
\var{kwlist} ���¸�ߤ��ʤ�������ɥѥ�᥿���Ϥ��줿��硢
\exception{TypeError} �����Ф�����������ޤ���}

�ʲ��˥�����ɤ�Ȥä��⥸�塼����򼨤��ޤ��������
Geoff Philbrick (\email{philbrick@hks.com}) �ˤ��ץ���������
��Ȥˤ��Ƥ��ޤ�:%
\index{Philbrick, Geoff}

\begin{verbatim}
#include "Python.h"

static PyObject *
keywdarg_parrot(PyObject *self, PyObject *args, PyObject *keywds)
{  
    int voltage;
    char *state = "a stiff";
    char *action = "voom";
    char *type = "Norwegian Blue";

    static char *kwlist[] = {"voltage", "state", "action", "type", NULL};

    if (!PyArg_ParseTupleAndKeywords(args, keywds, "i|sss", kwlist, 
                                     &voltage, &state, &action, &type))
        return NULL; 
  
    printf("-- This parrot wouldn't %s if you put %i Volts through it.\n", 
           action, voltage);
    printf("-- Lovely plumage, the %s -- It's %s!\n", type, state);

    Py_INCREF(Py_None);

    return Py_None;
}

static PyMethodDef keywdarg_methods[] = {
    /* PyCFunction ���ͤ� PyObject* �ѥ�᥿����Ĥ�������������
     * ���ʤ����� keywordarg_parrot() �ϻ��ĤȤ�Τǡ����㥹�Ȥ�
     * ɬ�ס�
     */
    {"parrot", (PyCFunction)keywdarg_parrot, METH_VARARGS | METH_KEYWORDS,
     "Print a lovely skit to standard output."},
    {NULL, NULL, 0, NULL}   /* ����ƥ��ͥ��� */
};
\end{verbatim}

\begin{verbatim}
void
initkeywdarg(void)
{
  /* �⥸�塼���������ƴؿ����ɲä��� */
  Py_InitModule("keywdarg", keywdarg_methods);
}
\end{verbatim}


\section{Ǥ�դ��ͤ��ۤ���
         \label{buildValue}}

\cfunction{Py_BuildValue()} ��\cfunction{PyArg_ParseTuple()} ��
�жˤ˰��֤����ΤǤ������δؿ��ϰʲ��Τ褦���������Ƥ��ޤ�:

\begin{verbatim}
PyObject *Py_BuildValue(char *format, ...);
\end{verbatim}

\cfunction{Py_BuildValue()} �ϡ�\cfunction{PyArg_ParseTuple()}
��ǧ�������Ϣ�ν񼰲�ñ�̤˻����񼰲�ñ�̤�ǧ�����ޤ���������
(�ؿ��ؤν��ϤǤϤʤ������Ϥ˻Ȥ���) �����ϥݥ��󥿤ǤϤʤ���
�������ͤǤʤ���Фʤ�ޤ���
Python ����ƤӽФ��줿 C �ؿ����֤��ͤȤ���Ŭ�ڤʡ������� Python 
���֥������Ȥ��֤��ޤ���

\cfunction{PyArg_ParseTuple()} �Ȥϰ�İ㤦��������ޤ�: 
\cfunction{PyArg_ParseTuple()} ���������򥿥ץ�ˤ���ɬ�פ�����ޤ�
(Python �ΰ����ꥹ�Ȥ�����Ū�ˤϾ�˥��ץ�Ȥ���ɽ������뤫��Ǥ�)
����\cfunction{Py_BuildValue()} �ϥ��ץ����������Ȥϸ¤�ޤ���
\cfunction{Py_BuildValue()} �Ͻ񼰲�ʸ������˽񼰲�ñ�̤�
��Ĥ�����ʾ����äƤ�����ˤΤߥ��ץ���ۤ��ޤ���
�񼰲�ʸ���󤬶��ʤ顢\code{None} ���֤��ޤ������ä����Ĥ�
�񼰲�ñ�̤ʤ顢���ν񼰲�ñ�̤����Ҥ��Ƥ��벿�餫�Υ��֥�������
�ˤʤ�ޤ����������� 0 �� 1 �Υ��ץ��֤��������Τʤ顢�񼰲�
ʸ�����ݳ�̤ǰϤ��ޤ���

�ʲ�����򼨤��ޤ� (���˸ƤӽФ���򡢱��˹��ۤ���� Python �ͤ򼨤��ޤ�):

\begin{verbatim}
    Py_BuildValue("")                        None
    Py_BuildValue("i", 123)                  123
    Py_BuildValue("iii", 123, 456, 789)      (123, 456, 789)
    Py_BuildValue("s", "hello")              'hello'
    Py_BuildValue("ss", "hello", "world")    ('hello', 'world')
    Py_BuildValue("s#", "hello", 4)          'hell'
    Py_BuildValue("()")                      ()
    Py_BuildValue("(i)", 123)                (123,)
    Py_BuildValue("(ii)", 123, 456)          (123, 456)
    Py_BuildValue("(i,i)", 123, 456)         (123, 456)
    Py_BuildValue("[i,i]", 123, 456)         [123, 456]
    Py_BuildValue("{s:i,s:i}",
                  "abc", 123, "def", 456)    {'abc': 123, 'def': 456}
    Py_BuildValue("((ii)(ii)) (ii)",
                  1, 2, 3, 4, 5, 6)          (((1, 2), (3, 4)), (5, 6))
\end{verbatim}


\section{���ȥ������ˡ
         \label{refcounts}}

C �� \Cpp �Τ褦�ʸ���Ǥϡ��ץ�����ޤϥҡ��׾�Υ����
ưŪ�˳��ݤ�������������ꤹ����Ǥ������ޤ���
����������Ȥ� C �Ǥϴؿ�\cfunction{malloc()} ��\cfunction{free()} ��
�Ԥ��ޤ���\Cpp �Ǥ��ܼ�Ū��Ʊ����̣�DZ黻��\keyword{new} ��
\keyword{delete} ���Ȥ��ޤ��������ǡ��ʲ��ε����� C �ξ��˸���
���ƹԤ��ޤ���

\cfunction{malloc()} �����ݤ������ƤΥ���֥��å��ϡ��ǽ�Ū�ˤ�
\cfunction{free()} ��̩�˰��٤����ƤӽФ������Ѳ�ǽ����Υס����
�ᤵ�ͤФʤ�ޤ��󡣤����ǡ�Ŭ�ڤʻ���\cfunction{free()} ��ƤӽФ�
���Ȥ����פˤʤ�ޤ���
�������֥��å����Ф��ơ�\cfunction{free()} ��ƤФʤ��ä��ˤ�
������餺���Υ��ɥ쥹��˺�Ѥ��Ƥ��ޤ��ȡ��֥��å�����ͭ���Ƥ������
�ϥץ�����ब��λ����ޤǺ����ѤǤ��ʤ��ʤ�ޤ���
����ϥ���꡼��(\dfn{memory leak}) �ȸƤФ�Ƥ��ޤ���
�դˡ��ץ�����ब�������֥��å����Ф���\cfunction{free()} ��
�Ƥ�Ǥ����ʤ��顢���Υ֥��å���Ȥ�³���褦�Ȥ���ȡ�
�̤� \cfunction{malloc()} �ƤӽФ��ˤ�äƹԤ���֥��å��κ�����
�Ⱦ��ͤ򵯤����ޤ�������ϲ����Ѥߥ���λ��� (\dfn{using freed memory})
�ȸƤФ�ޤ�������Ͻ��������Ƥ��ʤ��ǡ������Ф��뻲�Ȥ�Ʊ�ͤ�
�褯�ʤ���� --- ��������ס����ä����ȡ��ԲIJ�ʥ���å��� ---
������������ޤ���

�褯�������꡼���θ����ϥ�����������̤Ǥʤ�������ϩ�Ǥ���
�㤨�С�����ؿ����������֥��å�����ݤ������餫�η׻���Ԥäơ�
���٥֥��å����������Ȥ��ޤ������ơ��ؿ����׵���ͤ��ѹ����ơ�
�׻����Ф���ƥ��Ȥ��ɲä���ȡ����顼���򸡽Ф����ؿ��������
�������᤹�褦�ˤʤ뤫�⤷��ޤ���
��������Ǥν�λ��������Ȥ������ݤ��줿����֥��å��ϲ�����˺��
�䤹���ΤǤ��������ɤ�����ɲä��줿���ˤ��äˤ����Ǥ���
���Τ褦�ʥ���꡼������öʶ�����Ǥ��ޤ��ȡ�Ĺ���ָ���
����ʤ��ޤޤˤʤ뤳�Ȥ��褯����ޤ�: ���顼�ˤ��ؿ��ν�λ�ϡ�
���Ƥδؿ��ƤӽФ��Τ��Ф��Ƥۤ�Τ鷺���ʳ�礷�������������ΰ�����
�ۤȤ�ɤζ���Ū�ʷ׻����������̤β��۵�������äƤ��뤿�ᡢ
����꡼�������餫�ˤʤ�Τϡ�Ĺ����ư��Ƥ����ץ�������
�꡼���򵯤����ؿ����٤�Ȥä����˸¤��뤫��Ǥ���
���äơ����μ�Υ��顼��Ǿ��¤ˤȤɤ��褦�ʥ����ǥ��󥰵������ά��
�ߤ��ơ���θ�Υ���꡼�����򤱤뤳�Ȥ����פʤΤǤ���

Python ��\cfunction{malloc()} ��\cfunction{free()} �����ˤ褯����
���뤿�ᡢ����꡼�����ɻߤ˲ä����������줿����λ��Ѥ�
�ɻߤ�����ά��ɬ�פǤ������Τ�������Ф줿�Τ�
���ȥ������ˡ (\dfn{reference counting}) �ȸƤФ���ˡ�Ǥ���
���ȥ������ˡ�θ����ϴ�ñ�Ǥ�: ���ƤΥ��֥������Ȥˤ�
�����󥿤����ꡢ���֥������Ȥ��Ф��뻲�Ȥ��ɤ�������¸���줿��
�����󥿤򥤥󥯥���Ȥ������֥������Ȥ��Ф��뻲�Ȥ�������줿��
�ǥ�����Ȥ��ޤ��������󥿤������ˤʤä��顢���֥������Ȥؤ�
�Ǹ�λ��Ȥ�������줿���Ȥˤʤꡢ���֥������Ȥϲ�������ޤ���

�⤦��Ĥ���ά�ϼ�ư���١������쥯����� 
(\dfn{automatic garbage collection}) �ȸƤФ�Ƥ��ޤ���
(���ȥ������ˡ�ϥ��١������쥯�������ά�ΰ�ĤȤ��Ƶ󤲤��뤳�Ȥ�
����Τǡ���Ĥ���̤��뤿���ɮ�Ԥ� ``��ư (automatic)'' ��ȤäƤ��ޤ���)
��ư���١������쥯�������礭�������ϡ��桼����\cfunction{free()} 
������Ū�ˤ�Фʤ��Ƥ褤���Ȥˤ���ޤ���
(®�٤�����ͭ���������������Ȥ��Ƽ�ĥ����Ƥ��ޤ� --- ����
����ϳΤ�����¤ǤϤ���ޤ���)
C �ˤ����뼫ư���١������쥯�����η����ϡ����˲������Τ���
���١������쥯����¸�ߤ��ʤ��Ȥ������ȤǤ���������Ф���
���ȥ������ˡ�ϲ������Τ���������Ǥ��ޤ� (\cfunction{malloc()} 
��\cfunction{free()} �����ѤǤ���Τ�����Ǥ� --- C ɸ���
������ݾڤ��Ƥ��ޤ�)��
���Ĥ���������ʬ�������Τ��륬�١������쥯���� C �ǻȤ���褦��
�ʤ뤫�⤷��ޤ��󤬡�����ޤǤϻ��ȥ������ˡ�Ǥ�äƤ����ʳ��ˤ�
�ʤ��ΤǤ���

Python �Ǥϡ�����Ū�ʻ��ȥ������ˡ�μ�����ԤäƤ�������ǡ�
���Ȥν۴Ĥ򸡽Ф��뤿���Ư���۴Ļ��ȸ��е��� (cycle detector)
���󶡤��Ƥ��ޤ����۴Ļ��ȸ��е����Τ������ǡ�ľ�ܡ����ܤ�
������餺�۴Ļ��Ȥ������򵤤ˤ����˥��ץꥱ���������ۤǤ��ޤ�;
�Ȥ����Τ⡢���ȥ������ˡ������Ȥä����١������쥯����������
�Ȥäƽ۴Ļ��Ȥϼ���������Ǥ���
�۴Ļ��Ȥϡ�(���ܻ��Ȥξ���ޤ��) ��ߤؤλ��Ȥ����ä����֥�������
�����������뤿�ᡢ�۴���Υ��֥������Ȥϳơ��󥼥��λ��ȥ������
������ޤ���ŵ��Ū�ʻ��ȥ������ˡ�μ����Ǥϡ����Ȥ��۴Ļ��Ȥ��������
���֥������Ȥ��Ф���¾���������Ȥ��ʤ��Ȥ��Ƥ⡢
�۴Ļ�����ΤɤΥ��֥������Ȥ�°������������ѤǤ��ޤ���

�۴Ļ��ȸ��е����ϡ����ߤȤʤä��۴Ļ��Ȥ򸡽Ф���Python �Ǽ���
���줿������ؿ� (finalizer��\method{__del__()} �᥽�å�) �����
����Ƥ��ʤ������ꡢ�����Υ��������ѤǤ��ޤ���
������ؿ��������硢���е����ϸ��Ф����۴Ļ��Ȥ� \ulink{\module{gc}
�⥸�塼��}{../lib/module-gc.html} �� (����Ū�ˤϤ��Υ⥸�塼���
\code{garbage} �ѿ���) �˸������ޤ���\module{gc} �⥸�塼��ǤϤޤ���
���е��� (\function{collect()} �ؿ�) ��¹Ԥ�����ˡ�������Ѥ�
���󥿥ե��������¹Ի��˸��е�����̵�������뵡ǽ��������Ƥ��ޤ���
�۴Ļ��ȸ��е����ϥ��ץ����ε����Ȥߤʤ���Ƥ��ޤ�;
�ǥե���Ȥ����äƤϤ��ޤ�����\UNIX{} �ץ�åȥե�����
(Mac OS X ��ޤߤޤ�) �Ǥϥӥ�ɻ���\program{configure} ������ץȤ�
\longprogramopt{without-cycle-gc} ���ץ�����Ȥäơ�
¾�Υץ�åȥե�����Ǥ�\file{pyconfig.h} �إå���\code{WITH_CYCLE_GC}
�����Ϥ�����̵���ˤǤ��ޤ���
�������ƽ۴Ļ��ȸ��е�����̵��������ȡ�\module{gc} �⥸�塼���
���ѤǤ��ʤ��ʤ�ޤ���


\subsection{Python �ˤ����뻲�ȥ������ˡ
            \label{refcountsInPython}}

Python �ˤϡ����ȥ�����ȤΥ��󥯥���Ȥ�ǥ�����Ȥ����������Ĥ�
�ޥ�����\code{Py_INCREF(x)} �� \code{Py_DECREF(x)} ������ޤ���
\cfunction{Py_DECREF()} �ϡ����ȥ�����Ȥ���������ã�����ݤˡ�
���֥������ȤΥ��������Ԥ��ޤ���
��������������뤿��ˡ�\cfunction{free()} ��ľ�ܸƤӽФ��ޤ��� --- 
��������˥��֥������Ȥη����֥������� (\dfn{type object})
��𤷤ޤ������Τ���� (¾����Ū�⤢��ޤ���)�����ƤΥ��֥������Ȥˤ�
���Ȥη����֥������Ȥ��Ф���ݥ��󥿤����äƤ��ޤ���

���ơ��ޤ�����ʵ��䤬�ĤäƤ��ޤ�: ���� \code{Py_INCREF(x)} ��
\code{Py_DECREF(x)} ��Ȥ��Ф褤�ΤǤ��礦��?  
�ޤ��������Ĥ����Ѹ���������Ϥᤵ���Ƥ���������
�ޤ������֥������Ȥ� ``��ͭ (own)'' ����뤳�ȤϤ���ޤ���;
�����������륪�֥������Ȥ��Ф��뻲�Ȥν�ͭ \dfn{own a reference} 
�ϤǤ��ޤ������֥������Ȥλ��ȥ�����Ȥϡ����Υ��֥������Ȥ�
���Ȥ��ͭ������Ƥ��������������Ƥ��ޤ���
���Ȥν�ͭ�Ԥϡ����Ȥ�ɬ�פʤ��ʤä��ݤ�\cfunction{Py_DECREF()} 
��ƤӽФ�����ô���ޤ������Ȥν�ͭ���ϰѾ� (transfer) �Ǥ��ޤ���
��ͭ���� (owned reference) �������ˤϡ��Ϥ�����¸���롢
\cfunction{Py_DECREF()} ��ƤӽФ����Ȥ������Ĥ���ˡ������ޤ���
��ͭ���Ȥ������˺���ȡ�����꡼��������������ޤ���

���֥������Ȥ��Ф��뻲�Ȥϡ����� (\dfn{borrow}) ���ǽ�Ǥ���
\footnote{���Ȥ� ``���Ѥ���'' �Ȥ����᥿�ե��ϸ�̩�ˤ�����������ޤ���:
�ʤ��ʤ顢���Ȥν�ͭ�Ԥϰ����Ȥ��ƻ��ȤΥ��ԡ�����äƤ���
����Ǥ���}
���Ȥμ��ѼԤϡ�\cfunction{Py_DECREF()} ��Ƥ�ǤϤʤ�ޤ���
���ѼԤϡ����Ȥν�ͭ�Ԥ�����Ѥ������֤�Ķ���ƻ��Ȥ��ݻ���³���Ƥ�
�ʤ�ޤ��󡣽�ͭ�Ԥ����Ȥ�����������Ǽ��ѻ��Ȥ�Ȥ��ȡ�
�����Ѥߥ������Ѥ��Ƥ��ޤ�����������Τǡ����Ф��򤱤ͤФʤ�ޤ���
\footnote{���ȥ�����Ȥ� 1 �ʾ夫�ɤ���Ĵ�٤���ˡ��
\strong{���ޤ������ޤ���} --- ���ȥ�����ȼ��Τ�������줿������
���뤿�ᡢ�����ΰ褬¾�Υ��֥������Ȥ˻Ȥ��Ƥ����ǽ��������ޤ�!}

���Ȥμ��Ѥ����Ȥν�ͭ����ͥ��Ƥ������ϡ������ɤ��Ȥꤦ��
�����������ϩ�ǻ��Ȥ��Ѵ����Ƥ����褦���դ��ʤ��ƺѤळ�ȤǤ�
--- �̤θ������򤹤�С����ѻ��Ȥξ��ˤϡ�����������Ǵؿ���
��λ���Ƥ����꡼���δ������������Ȥ��ʤ����Ȥ������ȤǤ���
�դˡ�����꡼���δ����������������������ϡ������ޤȤ��
�����륳���ɤ����ºݤˤϻ��Ȥμ��Ѹ�����������Ƥ��ޤä����
���λ��Ȥ�Ȥ����⤷��ʤ��褦����̯�ʾ���������Ȥ������ȤǤ���

\cfunction{Py_INCREF()} ��ƤӽФ��ȡ����ѻ��Ȥ��ͭ���� 
���ѹ��Ǥ��ޤ����������ϻ��Ȥμ��Ѹ��ξ��֤ˤϱƶ����ޤ��� --- 
\cfunction{Py_INCREF()} �Ͽ����ʽ�ͭ���Ȥ������������Ȥν�ͭ�Ԥ�
ô���٤����Ƥ���Ǥ��ݤ��ޤ� (�Ĥޤꡢ�����ʻ��Ȥν�ͭ�Ԥϡ�������
��ͭ�Ԥ�Ʊ�͡����Ȥ�������Ŭ�ڤ˹Ԥ�ͤФʤ�ޤ���)��


\subsection{��ͭ���ˤޤĤ�뵬§
            \label{ownershipRules}}

���֥������Ȥؤλ��Ȥ�ؿ����⳰���Ϥ����ˤϡ����֥������Ȥ�
��ͭ�������Ȥȶ����Ϥ���뤫�ݤ�����˴ؿ����󥿥ե��������ͤΰ�����
�ʤ�ޤ���

���֥������Ȥؤλ��Ȥ��֤��ۤȤ�ɤδؿ��ϡ����ȤȤȤ�˽�ͭ����
�Ϥ��ޤ����äˡ�\cfunction{PyInt_FromLong()} ��
\cfunction{Py_BuildValue()} �Τ褦�ˡ����������֥������Ȥ���������
�ؿ������ƽ�ͭ���������Ϥ��ޤ������֥������Ȥ��ºݤˤϿ�����
���֥������ȤǤʤ��Ƥ⡢���Υ��֥������Ȥ��Ф��뿷���ʻ��Ȥ�
��ͭ�������ޤ����㤨�С�\cfunction{PyInt_FromLong()}
�Ϥ褯�Ȥ��ͤ򥭥�å��夷�Ƥ��ꡢ����å��夵�줿�ͤؤλ��Ȥ�
�֤����Ȥ�����ޤ���

\cfunction{PyObject_GetAttrString()} �Τ褦�ˡ����륪�֥������Ȥ���
�̤Υ��֥������Ȥ���Ф���褦�ʴؿ���ޤ������ȤȤȤ�˽�ͭ����
�Ѿ����ޤ�������������Ϥ�����򤷤ˤ������⤷��ޤ��󡣤Ȥ����Τ�
�褯�Ȥ���롼����Τ����Ĥ����㳰�ȤʤäƤ��뤫��Ǥ�:
\cfunction{PyTuple_GetItem()}�� \cfunction{PyList_GetItem()}��
\cfunction{PyDict_GetItem()}������� \cfunction{PyDict_GetItemString()}
�����ơ����ץ롢�ꥹ�ȡ��ޤ��ϼ��񤫤���ѻ��Ȥ��֤��ޤ���

\cfunction{PyImport_AddModule()} �ϡ��ºݤˤϥ��֥������Ȥ���������
�֤����Ȥ�����ˤ⤫����餺�����ѻ��Ȥ��֤��ޤ�: ���줬��ǽ�ʤΤϡ�
�������줿���֥������Ȥ��Ф����ͭ���Ȥ�\code{sys.modules} ��
�ݻ�����뤫��Ǥ���

���֥������Ȥؤλ��Ȥ��̤δؿ����Ϥ���硢����Ū�ˤϡ��ؿ�¦��
�ƤӽФ��꤫�黲�Ȥ���Ѥ��ޤ� --- ���Ȥ���¸����ɬ�פ�����ʤ顢
�ؿ�¦��\cfunction{Py_INCREF()} ��ƤӽФ�����Ω������ͭ�Ԥ�
�ʤ�ޤ����ȤϤ��������ε�§�ˤ���Ĥν��פ��㳰:
\cfunction{PyTuple_SetItem()} ��\cfunction{PyList_SetItem()}
������ޤ��������δؿ��ϡ��Ϥ��줿�������Ǥ��Ф��ƽ�ͭ����
��ü�� (take over) �ޤ� --- ���Ȥ����Ԥ��Ƥ�Ǥ�!
(\cfunction{PyDict_SetItem()} �Ȥ�����֤Ͻ�ͭ�����ü��ޤ��� ---
�����Ϥ���� ``���̤�'' �ؿ��Ǥ���)

Python ���� C �ؿ����ƤӽФ����ݤˤϡ�C �ؿ��ϸƤӽФ�¦����
�����ؤλ��Ȥ���Ѥ��ޤ���C �ؿ��θƤӽФ�¦�ϥ��֥������Ȥؤλ��Ȥ�
��ͭ���Ƥ���Τǡ����ѻ��Ȥ���¸���֤��ݾڤ����Τϴؿ���������
�֤��ޤǤǤ������Τ褦�ˤ��Ƽ��ѻ��Ȥ���¸������¾���Ϥ����ꤷ����
���ˤΤߡ�\cfunction{Py_INCREF()} ��Ȥäƽ�ͭ���Ȥˤ���ɬ�פ�
����ޤ���

Python ����ƤӽФ��줿 C �ؿ����֤����ȤϽ�ͭ���ȤǤʤ����
�ʤ�ޤ��� --- ��ͭ���ϴؿ�����ƤӽФ�¦�ؤȰѾ�����ޤ���


\subsection{��ɹ
            \label{thinIce}}

�����ʤ������ˤ����ơ��츫̵���˸�������ѻ��Ȥ����Ѥ������Ҥ�������
���Ȥ�����ޤ�����������Ϥ��٤ơ����󥿥ץ꥿��������Ū�˸ƤӽФ��졢
���󥿥ץ꥿�����Ȥν�ͭ�Ԥ˻��Ȥ����������Ƥ��ޤ������ȴط����Ƥ��ޤ���

�ΤäƤ����٤��������Τ����ǽ�Ρ������ƺǤ���פʤ�Τϡ�
�ꥹ�����Ǥ��Ф��뻲�Ȥ�ڤ�Ƥ���ݤ˵����롢
�ط��ʤ����֥������Ȥ��Ф���\cfunction{Py_DECREF()} �λ��ѤǤ���
�㤨��:

\begin{verbatim}
void
bug(PyObject *list)
{
    PyObject *item = PyList_GetItem(list, 0);

    PyList_SetItem(list, 1, PyInt_FromLong(0L));
    PyObject_Print(item, stdout, 0); /* BUG! */
}
\end{verbatim}

��δؿ��Ϥޤ���\code{list[0]} �ؤλ��Ȥ���Ѥ�������\code{list[1]} 
���� \code{0} ���֤��������Ǹ�ˤ����ۤɼ��Ѥ������Ȥ����
���Ƥ��ޤ�����������ʤ��褦�˸����ޤ���? �Ǥ⤽���ǤϤʤ��ΤǤ�!

\cfunction{PyList_SetItem()} �ν�����ή������פ��Ƥߤޤ��礦��
�ꥹ�Ȥ����Ƥ����Ǥ��Ф��ƻ��Ȥ��ͭ���Ƥ���Τǡ����� 1 ��
�֤�������ȡ����������� 1 ���������ޤ��������ǡ����������� 1 
���桼��������饹�Υ��󥹥��󥹤Ǥ��ꡢ����ˤ��Υ��饹��
\method{__del__()} �᥽�åɤ�������Ƥ���Ȳ��ꤷ�ޤ��礦��
���Υ��饹���󥹥��󥹤λ��ȥ�����Ȥ� 1 ���ä���硢
�ꥹ�Ȥ����Ȥ���������ȡ����󥹥��󥹤� \method{__del__()}
�᥽�åɤ��ƤӽФ���ޤ���

���饹�� Python �ǽ񤫤�Ƥ���Τǡ�\method{__del__()}
��Ǥ�դ� Python �����ɤ�¹ԤǤ��ޤ������� \method{__del__()}
�� \cfunction{bug()} �ˤ����� \code{item} �˲��������ʤ��Ȥ򤷤�
����ΤǤ��礦��? �����̤�! \cfunction{buf()} ���Ϥ����ꥹ�Ȥ�
\method{__del__()} �᥽�åɤ������Ǥ���Ȥ���ȡ�\samp{del list[0]}
�θ��̤���Ĥ褦��ʸ��¹ԤǤ��Ƥ��ޤ��ޤ����⤷��������
\code{list[0]} ���Ф���Ǹ�λ��Ȥ���������Ƥ��ޤ��ȡ�
\code{list[0]} �˴�Ϣ�դ����Ƥ�������ϲ������졢
���Ū�� \code{item} ��̵�����ͤˤʤäƤ��ޤ��ޤ���

����θ�����ʬ����С����ϴ�ñ�Ǥ���
���Ū�˻��Ȳ�������䤻�Ф褤�ΤǤ���
������ư���С������ϰʲ��Τ褦�ˤʤ�ޤ�:

\begin{verbatim}
void
no_bug(PyObject *list)
{
    PyObject *item = PyList_GetItem(list, 0);

    Py_INCREF(item);
    PyList_SetItem(list, 1, PyInt_FromLong(0L));
    PyObject_Print(item, stdout, 0);
    Py_DECREF(item);
}
\end{verbatim}

����ϼºݤˤ��ä��äǤ��������ΥС������� Python �ˤϡ�
���ΥХ��ΰ�郎����Ǥ��ơ�\method{__del__()} �᥽�åɤ�
�ɤ����Ƥ��ޤ�ư���ʤ��Τ���Ĵ�٤뤿��� C �ǥХå�������
���֤���䤷���ͤ����ޤ���...

����ܤϡ����ѻ��Ȥ�����åɤ˴ط����Ƥ��륱�����Ǥ���
�̾�ϡ� Python ���󥿥ץ꥿�ˤ�����ʣ���Υ���åɤϡ�
�������Х륤�󥿥ץ꥿���å������֥������ȶ������Τ��ݸ�Ƥ���
���ᡢ�ߤ��˼��⤷�礦���ȤϤ���ޤ��󡣤ȤϤ��������å���
\csimplemacro{Py_BEGIN_ALLOW_THREADS} �ޥ����ǰ��Ū�˲�������ꡢ
\csimplemacro{Py_END_ALLOW_THREADS} �ǺƳ���������Ǥ��ޤ���
�����Υޥ����ϥ֥��å��ε����� I/O �ƤӽФ��μ��Ϥˤ褯�֤��졢
I/O ����λ����ޤǤδ֤�¾�Υ���åɤ��ץ����å������ѤǤ���褦��
���ޤ������餫�ˡ��ʲ��δؿ��Ͼ����Ȼ��������Ϥ��Ǥ��ޤ�:

\begin{verbatim}
void
bug(PyObject *list)
{
    PyObject *item = PyList_GetItem(list, 0);
    Py_BEGIN_ALLOW_THREADS
    ...�֥��å��������벿�餫�� I/O �ƤӽФ�...
    Py_END_ALLOW_THREADS
    PyObject_Print(item, stdout, 0); /* BUG! */
}
\end{verbatim}


\subsection{NULL �ݥ���
            \label{nullPointers}}

�������Ȥ��ơ����֥������Ȥؤλ��Ȥ�����ˤȤ�ؿ��ϥ桼����
\NULL{} �ݥ��󥿤��Ϥ��Ȥ�ͽ�ۤ��Ƥ��餺���Ϥ����Ȥ����
��������פˤʤ� (�������Ȥǥ�������פ����������) ���ȤǤ��礦��
���������֥������Ȥؤλ��Ȥ��֤��褦�ʴؿ��ϰ��̤ˡ��㳰��ȯ����
�������ˤΤ� \NULL{} ���֤��ޤ����������Ф��� \NULL{} �ƥ��Ȥ�
�Ԥ�ʤ���ͳ�ϡ����������ؿ����Ϥ��Ф��м�����ä��ؿ���¾�δؿ��ؤ�
�����Ϥ�����Ǥ� --- �ơ��δؿ��� \NULL �ƥ��Ȥ�Ԥ��С�
��Ĺ�ʥƥ��Ȥ����̤˹Ԥ�졢�����ɤϤ����®��ư�����Ȥˤʤ�ޤ���

���äơ�\NULL{} �Υƥ��Ȥϥ��֥������Ȥ� ``ȯ����''�����ʤ��
�ͤ� \NULL{} �ˤʤ뤫�⤷��ʤ��ݥ��󥿤������ä��Ȥ�������
���ޤ��礦��\cfunction{malloc()} �䡢�㳰�����Ф����ǽ����
����ؿ���������Ǥ���

�ޥ���\cfunction{Py_INCREF()} ����� \cfunction{Py_DECREF()}
�� \NULL{} �ݥ��󥿤Υ����å���Ԥ��ޤ��� --- ��������������
�ޥ������Ѳ����Ǥ���
\cfunction{Py_XINCREF()} ����� \cfunction{Py_XDECREF()} ��
�����å���Ԥ��ޤ���

����Υ��֥������ȷ��ˤĤ���Ĵ�٤�ޥ��� (\code{Py\var{type}_Check()}) 
�� \NULL{} �ݥ��󥿤Υ����å���Ԥ��ޤ��� --- �����֤��ޤ�����
�͡��ʰۤʤ뷿�����ꤷ�ƥ��֥������Ȥη���Ĵ�٤�ݤˤϡ���������
�ޥ�����³���ƸƤӽФ�ɬ�פ�����Τǡ����̤� \NULL{} �ݥ��󥿤�
�����å��򤹤�Ⱦ�Ĺ�ʥƥ��ȤˤʤäƤ��ޤ��ΤǤ���
����Ĵ�٤�ޥ����ˤϡ�\NULL{} �����å���Ԥ��Ѳ����Ϥ���ޤ���

Python ���� C �ؿ���ƤӽФ������ϡ� C �ؿ����Ϥ��������ꥹ��
(��Ǥ����Ȥ����� \code{args}) ���褷�� \NULL{} �ˤʤ�ʤ��褦
�ݾڤ��Ƥ��ޤ� --- �ºݤˤϡ���˥��ץ뷿�ˤʤ�褦�ݾڤ��Ƥ��ޤ���
\footnote{``�켰��'' �ƤӽФ������ȤäƤ�����ˤϡ������ݾڤ�
Ŭ�Ѥ���ޤ��� --- ��¸�Υ����ɤˤϤ��ޤ��˵켰�θƤӽФ�����
¿������ޤ�}

\NULL{} �ݥ��󥿤� Python �桼����٥�� ``ƨ����'' �Ƥ��ޤ��ȡ�
����ʥ��顼������������ޤ���

% Frank Stajano:
% A pedagogically buggy example, along the lines of the previous listing, 
% would be helpful here -- showing in more concrete terms what sort of 
% actions could cause the problem. I can't very well imagine it from the 
% description.


\section{\Cpp �Ǥγ�ĥ�⥸�塼�����
         \label{cplusplus}}

\Cpp �Ǥ��ĥ�⥸�塼��Ϻ����Ǥ��ޤ��������������Ĥ����¤�����ޤ���
�ᥤ��ץ������ (Python ���󥿥ץ꥿) �� C ����ѥ���ǥ���ѥ��뤵��
��󥯤���Ƥ���Τǡ��������Х��ѿ�����Ū���֥������Ȥ򥳥󥹥ȥ饯��
�Ǻ����Ǥ��ޤ��󡣥ᥤ��ץ�����ब \Cpp{} ����ѥ���ǥ�󥯤����
����ʤ餳�������ǤϤ���ޤ���
Python ���󥿥ץ꥿����ƤӽФ����ؿ� (�ä˥⥸�塼�������ؿ�)
�ϡ�\code{extern "C"} ��Ȥä�������ʤ���Фʤ�ޤ���
�ޤ���Python �إå��ե������\code{extern "C" \{...\}} �������ɬ��
�Ϥ���ޤ���--- ����ܥ�\samp{__cplusplus} (�Ƕ�� \Cpp{} ����ѥ����
���Ƥ��Υ���ܥ��������Ƥ��ޤ�) ���������Ƥ���Ȥ���
\code{extern "C" \{...\}} ���Ԥ���褦�ˡ��إå��ե��������
���Ǥ˽񤫤�Ƥ��뤫��Ǥ���


\section{��ĥ�⥸�塼��� C API ���󶡤���
         \label{using-cobjects}}
\sectionauthor{Konrad Hinsen}{hinsen@cnrs-orleans.fr}

¿���γ�ĥ�⥸�塼���ñ�� Python ����Ȥ��뿷���ʴؿ��䷿��
�󶡤�������Ǥ��������˳�ĥ�⥸�塼����Υ����ɤ�¾�γ�ĥ
�⥸�塼��Ǥ������ʤ��Ȥ�����ޤ����㤨�С�����⥸�塼��Ǥ�
�����ǰ�Τʤ��ꥹ�ȤΤ褦��ư��� ``���쥯����� (collection)'' 
���饹��������Ƥ��뤫�⤷��ޤ���
���礦�ɥꥹ�Ȥ�����������������Ǥ��� C API ��������ɸ���
Python �ꥹ�ȷ��Τ褦�ˡ����ο����ʥ��쥯����󷿤�¾��
��ĥ�⥸�塼�뤫��ľ�����Ǥ���褦�ˤ���ˤϰ�Ϣ�� C �ؿ���
���äƤ��ʤ���Фʤ�ޤ���

�츫����Ȥ���ϴ�ñ�ʤ���: ñ�˴ؿ��� (�������\keyword{static} 
�ʤɤȤ����������) �񤤤ơ�Ŭ�ڤʥإå��ե�������󶡤���C API
��񤱤Ф褤�������˻פ��ޤ��������ƼºݤΤȤ��������Ƥ�
��ĥ�⥸�塼�뤬 Python ���󥿥ץ꥿�˾����Ū�˥�󥯤���Ƥ���
���ˤϤ��ޤ�ư��ޤ���
�Ȥ������⥸�塼�뤬��ͭ�饤�֥��ξ��ˤϡ���ĤΥ⥸�塼���
�������Ƥ��륷��ܥ뤬¾�Υ⥸�塼�뤫���ԲĻ�ʤ��Ȥ�����ޤ���
�Ļ����ξܺ٤ϥ��ڥ졼�ƥ��󥰥����ƥ�ˤ��ޤ�; ���륷���ƥ��
Python ���󥿥ץ꥿�����Ƥγ�ĥ�⥸�塼���Ѥ�ñ��Υ������Х��
̾�����֤��Ѱդ��Ƥ��ޤ� (�㤨�� Windows)���̤Υ����ƥ�ϥ⥸�塼���
��󥯻��˼����ޤ�륷��ܥ������Ū�˻��ꤹ��ɬ�פ�����ޤ� 
(AIX �����ΰ���Ǥ�)���ޤ��̤Υ����ƥ� (�ۤȤ�ɤ� \UNIX{}) �Ǥϡ�
��ä���ά�������Ȥ����󶡤��Ƥ��ޤ���
�����ơ����Ȥ�����ܥ뤬�������Х��ѿ��Ȥ��ƲĻ�Ǥ��äƤ⡢
�ƤӽФ������ؿ������ä��⥸�塼�뤬�ޤ������ɤ���Ƥ��ʤ�����
���äƤ���ޤ�!

���äơ��������������饷��ܥ�βĻ����ˤϲ��鲾��򤷤ƤϤʤ�ʤ�
���Ȥˤʤ�ޤ����Ĥޤ��ĥ�⥸�塼��������ƤΥ���ܥ��
\keyword{static} ��������ͤФʤ�ޤ����㳰�ϥ⥸�塼��ν�����ؿ�
�ǡ������ (\ref{methodTable} �ǽҤ٤��褦��) ¾�γ�ĥ�⥸�塼��Ȥδ֤�
̾�������ͤ���Τ��򤱤뤿��Ǥ���
�ޤ���¾�γ�ĥ�⥸�塼�뤫�饢��������\emph{������٤��ǤϤʤ�} 
����ܥ���̤Τ�����Ǹ������ͤФʤ�ޤ���

Python �Ϥ����ĥ�⥸�塼��� C ��٥�ξ��� (�ݥ���) ���̤�
�⥸�塼����Ϥ�������ü�ʵ���: CObject ���󶡤��Ƥ��ޤ���
CObject �ϥݥ��� (\ctype{void*}) �򵭲����� Python �Υǡ������Ǥ���
CObject �� C API ��𤷤ƤΤ����������ꥢ������������Ǥ��ޤ�����
¾�� Python ���֥������Ȥ�Ʊ���褦�˼����Ϥ��Ǥ��ޤ���
�Ȥ�櫓��CObject �ϳ�ĥ�⥸�塼���̾��������ˤ���̾��������
�Ǥ��ޤ���¾�γ�ĥ�⥸�塼��Ϥ��Υ⥸�塼��� import �Ǥ�������̾����
���������Ǹ��CObject �ؤΥݥ��󥿤�������ޤ���

��ĥ�⥸�塼��� C API ��������뤿��ˡ��͡�����ˡ�� CObject ��
�Ȥ��ޤ����������ݡ��Ȥ���Ƥ��뤽�줾���̾����Ȥ��ȡ�CObject
���Τ䡢CObject ����ɽ���Ƥ��륢�ɥ쥹�Ǽ������������˼����줿
���Ƥ� C API �ݥ��󥿤������ޤ���
�����ơ��ݥ��󥿤��Ф�����¸������Ȥ��ä��͡��ʺ�Ȥϡ������ɤ�
�󶡤��Ƥ���⥸�塼��ȥ��饤����ȥ⥸�塼��Ȥδ֤Ǥϰۤʤ�
��ˡ��ʬ���Ǥ��ޤ���

�ʲ�����Ǥϡ�̾�����������⥸�塼��κ�ԤˤۤȤ�ɤ���٤�
�ݤ���ޤ������褯�Ȥ���饤�֥�����ݤ�Ŭ�ڤʥ��ץ�������
�±餷�ޤ���
���Υ��ץ������Ǥϡ����Ƥ� C API �ݥ��� (����Ǥϰ�Ĥ����Ǥ���!) ��
CObject ���ͤȤʤ�\ctype{void} �ݥ��󥿤��������¸���ޤ���
��ĥ�⥸�塼����б�����إå��ե�����ϡ��⥸�塼��� import 
�� C API �ݥ��󥿤��������褦���ۤ���ޥ������󶡤��ޤ�;
���饤����ȥ⥸�塼��ϡ�C API �˥��������������ˤ���
�ޥ�����Ƥ֤����Ǥ���

̾�����������¦�Υ⥸�塼��ϡ�\ref{simpleExample} ���\module{spam} 
�⥸�塼�����������ΤǤ����ؿ�\function{spam.system()} ��
C �饤�֥��ؿ�\cfunction{system()} ��ľ�ܸƤӽФ�����
\cfunction{PySpam_System()} ��ƤӽФ��ޤ������δؿ��Ϥ������
�ºݤˤ� (���ƤΥ��ޥ�ɤ� ``spam'' ���դ���Ȥ��ä��褦��) 
���������ä�������Ԥ��ޤ���
���δؿ� \cfunction{PySpam_System()} �Ϥޤ���¾�γ�ĥ�⥸�塼��
�ˤ��������ޤ���

�ؿ�\cfunction{PySpam_System()} �ϡ�¾�����Ƥδؿ���Ʊ�ͤ�
\keyword{static} ��������줿�̾�� C �ؿ��Ǥ���

\begin{verbatim}
static int
PySpam_System(const char *command)
{
    return system(command);
}
\end{verbatim}

\cfunction{spam_system()} �ˤϼ���­��ʤ��ѹ����ܤ���Ƥ��ޤ�:

\begin{verbatim}
static PyObject *
spam_system(PyObject *self, PyObject *args)
{
    const char *command;
    int sts;

    if (!PyArg_ParseTuple(args, "s", &command))
        return NULL;
    sts = PySpam_System(command);
    return Py_BuildValue("i", sts);
}
\end{verbatim}

�⥸�塼�����Ƭ�ˤ���ʲ��ι�

\begin{verbatim}
#include "Python.h"
\end{verbatim}

��ľ��ˡ��ʲ������:

\begin{verbatim}
#define SPAM_MODULE
#include "spammodule.h"
\end{verbatim}

��ɬ���ɲä��Ƥ���������

\code{\#define} �ϡ��ե�����\file{spammodule.h} �򥤥󥯥롼�ɤ���
����Τ�̾�����������¦�Υ⥸�塼��Ǥ��äơ����饤����ȥ⥸�塼��
�ǤϤʤ����Ȥ�إå��ե�����˶����뤿��˻Ȥ��ޤ���
�Ǹ�ˡ��⥸�塼��ν�����ؿ��� C API �Υݥ����������������褦
���ۤ��ʤ���Фʤ�ޤ���:

\begin{verbatim}
PyMODINIT_FUNC
initspam(void)
{
    PyObject *m;
    static void *PySpam_API[PySpam_API_pointers];
    PyObject *c_api_object;

    m = Py_InitModule("spam", SpamMethods);

    /* C API �ݥ���������������� */
    PySpam_API[PySpam_System_NUM] = (void *)PySpam_System;

    /* API �ݥ�������Υ��ɥ쥹�����ä� CObject ���������� */
    c_api_object = PyCObject_FromVoidPtr((void *)PySpam_API, NULL);

    if (c_api_object != NULL)
        PyModule_AddObject(m, "_C_API", c_api_object);
}
\end{verbatim}

\code{PySpam_API} ��\keyword{static} ���������Ƥ��뤳�Ȥ����դ���
��������; �������ʤ���С�\function{initspam()} ����λ�����Ȥ���
�ݥ��󥿥��쥤�Ͼ��Ǥ��Ƥ��ޤ��ޤ�!

���餯�������ʬ�ϥإå��ե����� \file{spammodule.h} ��ˤ��ꡢ
�ʲ��Τ褦�ˤʤäƤ��ޤ�:

\begin{verbatim}
#ifndef Py_SPAMMODULE_H
#define Py_SPAMMODULE_H
#ifdef __cplusplus
extern "C" {
#endif

/* spammodule �Υإå��ե����� */

/* C API �ؿ� */
#define PySpam_System_NUM 0
#define PySpam_System_RETURN int
#define PySpam_System_PROTO (const char *command)

/* C API �ݥ��󥿤����� */
#define PySpam_API_pointers 1


#ifdef SPAM_MODULE
/* ������ʬ�� spammodule.c �򥳥�ѥ��뤹��ݤ˻Ȥ��� */

static PySpam_System_RETURN PySpam_System PySpam_System_PROTO;

#else
/* ������ʬ�� spammodule �� API ��Ȥ��⥸�塼��¦�ǻȤ��� */

static void **PySpam_API;

#define PySpam_System \
 (*(PySpam_System_RETURN (*)PySpam_System_PROTO) PySpam_API[PySpam_System_NUM])

/* ���顼�ˤ���㳰�ξ��ˤ� -1 ����������� 0 ���֤� */
static int
import_spam(void)
{
    PyObject *module = PyImport_ImportModule("spam");

    if (module != NULL) {
        PyObject *c_api_object = PyObject_GetAttrString(module, "_C_API");
        if (c_api_object == NULL)
            return -1;
        if (PyCObject_Check(c_api_object))
            PySpam_API = (void **)PyCObject_AsVoidPtr(c_api_object);
        Py_DECREF(c_api_object);
    }
    return 0;
}

#endif

#ifdef __cplusplus
}
#endif

#endif /* !defined(Py_SPAMMODULE_H) */
\end{verbatim}

\cfunction{PySpam_System()} �ؤΥ����������ʤ����뤿���
���饤����ȥ⥸�塼��¦�����ʤ���Фʤ�ʤ����Ȥϡ�������ؿ���
�Ǥ�\cfunction{import_spam()} �ؿ� (�ޤ��ϥޥ���) �θƤӽФ��Ǥ�:

\begin{verbatim}
PyMODINIT_FUNC
initclient(void)
{
    PyObject *m;

    Py_InitModule("client", ClientMethods);
    if (import_spam() < 0)
        return;
    /* ����ʤ����������Ϥ������֤��� */
}
\end{verbatim}

���Υ��ץ������μ��פʷ����ϡ�\file{spammodule.h} ���������
�ʤ�Ȥ������ȤǤ����ȤϤ������ƴؿ��δ���Ū�ʹ����ϸ��������
��Τ�Ʊ���ʤΤǡ���������٤����ؤ٤Ф��ߤޤ���

�Ǹ�ˡ�CObject �ϡ����Ȥ���¸����Ƥ���ݥ��󥿤������ݤ�����
���������ꤹ��ݤ��ä������ʡ��⤦��Ĥε�ǽ���󶡤��Ƥ���Ȥ���
���Ȥ˿���Ƥ����ͤФʤ�ޤ��󡣾ܺ٤�
\citetitle[../api/api.html]{Python/C API ��ե���󥹥ޥ˥奢��} 
�� ``\ulink{CObjects}{../api/cObjects.html} '' ���ᡢ�����
CObjects �μ�����ʬ (Python ����������������ʪ��Υե����� 
\file{Include/cobject.h} �����\file{Objects/cobject.c} 
�˽Ҥ٤��Ƥ��ޤ���


\chapter{Object Implementation Support \label{newTypes}}


This chapter describes the functions, types, and macros used when
defining new object types.


\section{Allocating Objects on the Heap
         \label{allocating-objects}}

\begin{cfuncdesc}{PyObject*}{_PyObject_New}{PyTypeObject *type}
\end{cfuncdesc}

\begin{cfuncdesc}{PyVarObject*}{_PyObject_NewVar}{PyTypeObject *type, Py_ssize_t size}
\end{cfuncdesc}

\begin{cfuncdesc}{void}{_PyObject_Del}{PyObject *op}
\end{cfuncdesc}

\begin{cfuncdesc}{PyObject*}{PyObject_Init}{PyObject *op,
					    PyTypeObject *type}
  Initialize a newly-allocated object \var{op} with its type and
  initial reference.  Returns the initialized object.  If \var{type}
  indicates that the object participates in the cyclic garbage
  detector, it is added to the detector's set of observed objects.
  Other fields of the object are not affected.
\end{cfuncdesc}

\begin{cfuncdesc}{PyVarObject*}{PyObject_InitVar}{PyVarObject *op,
						  PyTypeObject *type, Py_ssize_t size}
  This does everything \cfunction{PyObject_Init()} does, and also
  initializes the length information for a variable-size object.
\end{cfuncdesc}

\begin{cfuncdesc}{\var{TYPE}*}{PyObject_New}{TYPE, PyTypeObject *type}
  Allocate a new Python object using the C structure type \var{TYPE}
  and the Python type object \var{type}.  Fields not defined by the
  Python object header are not initialized; the object's reference
  count will be one.  The size of the memory
  allocation is determined from the \member{tp_basicsize} field of the
  type object.
\end{cfuncdesc}

\begin{cfuncdesc}{\var{TYPE}*}{PyObject_NewVar}{TYPE, PyTypeObject *type,
                                                Py_ssize_t size}
  Allocate a new Python object using the C structure type \var{TYPE}
  and the Python type object \var{type}.  Fields not defined by the
  Python object header are not initialized.  The allocated memory
  allows for the \var{TYPE} structure plus \var{size} fields of the
  size given by the \member{tp_itemsize} field of \var{type}.  This is
  useful for implementing objects like tuples, which are able to
  determine their size at construction time.  Embedding the array of
  fields into the same allocation decreases the number of allocations,
  improving the memory management efficiency.
\end{cfuncdesc}

\begin{cfuncdesc}{void}{PyObject_Del}{PyObject *op}
  Releases memory allocated to an object using
  \cfunction{PyObject_New()} or \cfunction{PyObject_NewVar()}.  This
  is normally called from the \member{tp_dealloc} handler specified in
  the object's type.  The fields of the object should not be accessed
  after this call as the memory is no longer a valid Python object.
\end{cfuncdesc}

\begin{cfuncdesc}{PyObject*}{Py_InitModule}{char *name,
                                            PyMethodDef *methods}
  Create a new module object based on a name and table of functions,
  returning the new module object.

  \versionchanged[Older versions of Python did not support \NULL{} as
                  the value for the \var{methods} argument]{2.3}
\end{cfuncdesc}

\begin{cfuncdesc}{PyObject*}{Py_InitModule3}{char *name,
                                             PyMethodDef *methods,
                                             char *doc}
  Create a new module object based on a name and table of functions,
  returning the new module object.  If \var{doc} is non-\NULL, it will
  be used to define the docstring for the module.

  \versionchanged[Older versions of Python did not support \NULL{} as
                  the value for the \var{methods} argument]{2.3}
\end{cfuncdesc}

\begin{cfuncdesc}{PyObject*}{Py_InitModule4}{char *name,
                                             PyMethodDef *methods,
                                             char *doc, PyObject *self,
                                             int apiver}
  Create a new module object based on a name and table of functions,
  returning the new module object.  If \var{doc} is non-\NULL, it will
  be used to define the docstring for the module.  If \var{self} is
  non-\NULL, it will passed to the functions of the module as their
  (otherwise \NULL) first parameter.  (This was added as an
  experimental feature, and there are no known uses in the current
  version of Python.)  For \var{apiver}, the only value which should
  be passed is defined by the constant \constant{PYTHON_API_VERSION}.

  \note{Most uses of this function should probably be using
  the \cfunction{Py_InitModule3()} instead; only use this if you are
  sure you need it.}

  \versionchanged[Older versions of Python did not support \NULL{} as
                  the value for the \var{methods} argument]{2.3}
\end{cfuncdesc}

DL_IMPORT

\begin{cvardesc}{PyObject}{_Py_NoneStruct}
  Object which is visible in Python as \code{None}.  This should only
  be accessed using the \code{Py_None} macro, which evaluates to a
  pointer to this object.
\end{cvardesc}


\section{Common Object Structures \label{common-structs}}

There are a large number of structures which are used in the
definition of object types for Python.  This section describes these
structures and how they are used.

All Python objects ultimately share a small number of fields at the
beginning of the object's representation in memory.  These are
represented by the \ctype{PyObject} and \ctype{PyVarObject} types,
which are defined, in turn, by the expansions of some macros also
used, whether directly or indirectly, in the definition of all other
Python objects.

\begin{ctypedesc}{PyObject}
  All object types are extensions of this type.  This is a type which
  contains the information Python needs to treat a pointer to an
  object as an object.  In a normal ``release'' build, it contains
  only the objects reference count and a pointer to the corresponding
  type object.  It corresponds to the fields defined by the
  expansion of the \code{PyObject_HEAD} macro.
\end{ctypedesc}

\begin{ctypedesc}{PyVarObject}
  This is an extension of \ctype{PyObject} that adds the
  \member{ob_size} field.  This is only used for objects that have
  some notion of \emph{length}.  This type does not often appear in
  the Python/C API.  It corresponds to the fields defined by the
  expansion of the \code{PyObject_VAR_HEAD} macro.
\end{ctypedesc}

These macros are used in the definition of \ctype{PyObject} and
\ctype{PyVarObject}:

\begin{csimplemacrodesc}{PyObject_HEAD}
  This is a macro which expands to the declarations of the fields of
  the \ctype{PyObject} type; it is used when declaring new types which
  represent objects without a varying length.  The specific fields it
  expands to depend on the definition of
  \csimplemacro{Py_TRACE_REFS}.  By default, that macro is not
  defined, and \csimplemacro{PyObject_HEAD} expands to:
  \begin{verbatim}
    Py_ssize_t ob_refcnt;
    PyTypeObject *ob_type;
  \end{verbatim}
  When \csimplemacro{Py_TRACE_REFS} is defined, it expands to:
  \begin{verbatim}
    PyObject *_ob_next, *_ob_prev;
    Py_ssize_t ob_refcnt;
    PyTypeObject *ob_type;
  \end{verbatim}
\end{csimplemacrodesc}

\begin{csimplemacrodesc}{PyObject_VAR_HEAD}
  This is a macro which expands to the declarations of the fields of
  the \ctype{PyVarObject} type; it is used when declaring new types which
  represent objects with a length that varies from instance to
  instance.  This macro always expands to:
  \begin{verbatim}
    PyObject_HEAD
    Py_ssize_t ob_size;
  \end{verbatim}
  Note that \csimplemacro{PyObject_HEAD} is part of the expansion, and
  that its own expansion varies depending on the definition of
  \csimplemacro{Py_TRACE_REFS}.
\end{csimplemacrodesc}

PyObject_HEAD_INIT

\begin{ctypedesc}{PyCFunction}
  Type of the functions used to implement most Python callables in C.
  Functions of this type take two \ctype{PyObject*} parameters and
  return one such value.  If the return value is \NULL, an exception
  shall have been set.  If not \NULL, the return value is interpreted
  as the return value of the function as exposed in Python.  The
  function must return a new reference.
\end{ctypedesc}

\begin{ctypedesc}{PyMethodDef}
  Structure used to describe a method of an extension type.  This
  structure has four fields:

  \begin{tableiii}{l|l|l}{member}{Field}{C Type}{Meaning}
    \lineiii{ml_name}{char *}{name of the method}
    \lineiii{ml_meth}{PyCFunction}{pointer to the C implementation}
    \lineiii{ml_flags}{int}{flag bits indicating how the call should be
                            constructed}
    \lineiii{ml_doc}{char *}{points to the contents of the docstring}
  \end{tableiii}
\end{ctypedesc}

The \member{ml_meth} is a C function pointer.  The functions may be of
different types, but they always return \ctype{PyObject*}.  If the
function is not of the \ctype{PyCFunction}, the compiler will require
a cast in the method table.  Even though \ctype{PyCFunction} defines
the first parameter as \ctype{PyObject*}, it is common that the method
implementation uses a the specific C type of the \var{self} object.

The \member{ml_flags} field is a bitfield which can include the
following flags.  The individual flags indicate either a calling
convention or a binding convention.  Of the calling convention flags,
only \constant{METH_VARARGS} and \constant{METH_KEYWORDS} can be
combined (but note that \constant{METH_KEYWORDS} alone is equivalent
to \code{\constant{METH_VARARGS} | \constant{METH_KEYWORDS}}).
Any of the calling convention flags can be combined with a
binding flag.

\begin{datadesc}{METH_VARARGS}
  This is the typical calling convention, where the methods have the
  type \ctype{PyCFunction}. The function expects two
  \ctype{PyObject*} values.  The first one is the \var{self} object for
  methods; for module functions, it has the value given to
  \cfunction{Py_InitModule4()} (or \NULL{} if
  \cfunction{Py_InitModule()} was used).  The second parameter
  (often called \var{args}) is a tuple object representing all
  arguments. This parameter is typically processed using
  \cfunction{PyArg_ParseTuple()} or \cfunction{PyArg_UnpackTuple}.
\end{datadesc}

\begin{datadesc}{METH_KEYWORDS}
  Methods with these flags must be of type
  \ctype{PyCFunctionWithKeywords}.  The function expects three
  parameters: \var{self}, \var{args}, and a dictionary of all the
  keyword arguments.  The flag is typically combined with
  \constant{METH_VARARGS}, and the parameters are typically processed
  using \cfunction{PyArg_ParseTupleAndKeywords()}.
\end{datadesc}

\begin{datadesc}{METH_NOARGS}
  Methods without parameters don't need to check whether arguments are
  given if they are listed with the \constant{METH_NOARGS} flag.  They
  need to be of type \ctype{PyCFunction}.  When used with object
  methods, the first parameter is typically named \code{self} and will
  hold a reference to the object instance.  In all cases the second
  parameter will be \NULL.
\end{datadesc}

\begin{datadesc}{METH_O}
  Methods with a single object argument can be listed with the
  \constant{METH_O} flag, instead of invoking
  \cfunction{PyArg_ParseTuple()} with a \code{"O"} argument. They have
  the type \ctype{PyCFunction}, with the \var{self} parameter, and a
  \ctype{PyObject*} parameter representing the single argument.
\end{datadesc}

\begin{datadesc}{METH_OLDARGS}
  This calling convention is deprecated.  The method must be of type
  \ctype{PyCFunction}.  The second argument is \NULL{} if no arguments
  are given, a single object if exactly one argument is given, and a
  tuple of objects if more than one argument is given.  There is no
  way for a function using this convention to distinguish between a
  call with multiple arguments and a call with a tuple as the only
  argument.
\end{datadesc}

These two constants are not used to indicate the calling convention
but the binding when use with methods of classes.  These may not be
used for functions defined for modules.  At most one of these flags
may be set for any given method.

\begin{datadesc}{METH_CLASS}
  The method will be passed the type object as the first parameter
  rather than an instance of the type.  This is used to create
  \emph{class methods}, similar to what is created when using the
  \function{classmethod()}\bifuncindex{classmethod} built-in
  function.
  \versionadded{2.3}
\end{datadesc}

\begin{datadesc}{METH_STATIC}
  The method will be passed \NULL{} as the first parameter rather than
  an instance of the type.  This is used to create \emph{static
  methods}, similar to what is created when using the
  \function{staticmethod()}\bifuncindex{staticmethod} built-in
  function.
  \versionadded{2.3}
\end{datadesc}

One other constant controls whether a method is loaded in place of
another definition with the same method name.

\begin{datadesc}{METH_COEXIST}
  The method will be loaded in place of existing definitions.  Without
  \var{METH_COEXIST}, the default is to skip repeated definitions.  Since
  slot wrappers are loaded before the method table, the existence of a
  \var{sq_contains} slot, for example, would generate a wrapped method
  named \method{__contains__()} and preclude the loading of a
  corresponding PyCFunction with the same name.  With the flag defined,
  the PyCFunction will be loaded in place of the wrapper object and will
  co-exist with the slot.  This is helpful because calls to PyCFunctions
  are optimized more than wrapper object calls.
  \versionadded{2.4}
\end{datadesc}

\begin{cfuncdesc}{PyObject*}{Py_FindMethod}{PyMethodDef table[],
                                            PyObject *ob, char *name}
  Return a bound method object for an extension type implemented in
  C.  This can be useful in the implementation of a
  \member{tp_getattro} or \member{tp_getattr} handler that does not
  use the \cfunction{PyObject_GenericGetAttr()} function.
\end{cfuncdesc}


\section{Type Objects \label{type-structs}}

Perhaps one of the most important structures of the Python object
system is the structure that defines a new type: the
\ctype{PyTypeObject} structure.  Type objects can be handled using any
of the \cfunction{PyObject_*()} or \cfunction{PyType_*()} functions,
but do not offer much that's interesting to most Python applications.
These objects are fundamental to how objects behave, so they are very
important to the interpreter itself and to any extension module that
implements new types.

Type objects are fairly large compared to most of the standard types.
The reason for the size is that each type object stores a large number
of values, mostly C function pointers, each of which implements a
small part of the type's functionality.  The fields of the type object
are examined in detail in this section.  The fields will be described
in the order in which they occur in the structure.

Typedefs:
unaryfunc, binaryfunc, ternaryfunc, inquiry, coercion, intargfunc,
intintargfunc, intobjargproc, intintobjargproc, objobjargproc,
destructor, freefunc, printfunc, getattrfunc, getattrofunc, setattrfunc,
setattrofunc, cmpfunc, reprfunc, hashfunc

The structure definition for \ctype{PyTypeObject} can be found in
\file{Include/object.h}.  For convenience of reference, this repeats
the definition found there:

\verbatiminput{typestruct.h}

The type object structure extends the \ctype{PyVarObject} structure.
The \member{ob_size} field is used for dynamic types (created
by  \function{type_new()}, usually called from a class statement).
Note that \cdata{PyType_Type} (the metatype) initializes
\member{tp_itemsize}, which means that its instances (i.e. type
objects) \emph{must} have the \member{ob_size} field.

\begin{cmemberdesc}{PyObject}{PyObject*}{_ob_next}
\cmemberline{PyObject}{PyObject*}{_ob_prev}
  These fields are only present when the macro \code{Py_TRACE_REFS} is
  defined.  Their initialization to \NULL{} is taken care of by the
  \code{PyObject_HEAD_INIT} macro.  For statically allocated objects,
  these fields always remain \NULL.  For dynamically allocated
  objects, these two fields are used to link the object into a
  doubly-linked list of \emph{all} live objects on the heap.  This
  could be used for various debugging purposes; currently the only use
  is to print the objects that are still alive at the end of a run
  when the environment variable \envvar{PYTHONDUMPREFS} is set.

  These fields are not inherited by subtypes.
\end{cmemberdesc}

\begin{cmemberdesc}{PyObject}{Py_ssize_t}{ob_refcnt}
  This is the type object's reference count, initialized to \code{1}
  by the \code{PyObject_HEAD_INIT} macro.  Note that for statically
  allocated type objects, the type's instances (objects whose
  \member{ob_type} points back to the type) do \emph{not} count as
  references.  But for dynamically allocated type objects, the
  instances \emph{do} count as references.

  This field is not inherited by subtypes.
\end{cmemberdesc}

\begin{cmemberdesc}{PyObject}{PyTypeObject*}{ob_type}
  This is the type's type, in other words its metatype.  It is
  initialized by the argument to the \code{PyObject_HEAD_INIT} macro,
  and its value should normally be \code{\&PyType_Type}.  However, for
  dynamically loadable extension modules that must be usable on
  Windows (at least), the compiler complains that this is not a valid
  initializer.  Therefore, the convention is to pass \NULL{} to the
  \code{PyObject_HEAD_INIT} macro and to initialize this field
  explicitly at the start of the module's initialization function,
  before doing anything else.  This is typically done like this:

\begin{verbatim}
Foo_Type.ob_type = &PyType_Type;
\end{verbatim}

  This should be done before any instances of the type are created.
  \cfunction{PyType_Ready()} checks if \member{ob_type} is \NULL, and
  if so, initializes it: in Python 2.2, it is set to
  \code{\&PyType_Type}; in Python 2.2.1 and later it is
  initialized to the \member{ob_type} field of the base class.
  \cfunction{PyType_Ready()} will not change this field if it is
  non-zero.

  In Python 2.2, this field is not inherited by subtypes.  In 2.2.1,
  and in 2.3 and beyond, it is inherited by subtypes.
\end{cmemberdesc}

\begin{cmemberdesc}{PyVarObject}{Py_ssize_t}{ob_size}
  For statically allocated type objects, this should be initialized
  to zero.  For dynamically allocated type objects, this field has a
  special internal meaning.

  This field is not inherited by subtypes.
\end{cmemberdesc}

\begin{cmemberdesc}{PyTypeObject}{char*}{tp_name}
  Pointer to a NUL-terminated string containing the name of the type.
  For types that are accessible as module globals, the string should
  be the full module name, followed by a dot, followed by the type
  name; for built-in types, it should be just the type name.  If the
  module is a submodule of a package, the full package name is part of
  the full module name.  For example, a type named \class{T} defined
  in module \module{M} in subpackage \module{Q} in package \module{P}
  should have the \member{tp_name} initializer \code{"P.Q.M.T"}.

  For dynamically allocated type objects, this should just be the type
  name, and the module name explicitly stored in the type dict as the
  value for key \code{'__module__'}.

  For statically allocated type objects, the tp_name field should
  contain a dot.  Everything before the last dot is made accessible as
  the \member{__module__} attribute, and everything after the last dot
  is made accessible as the \member{__name__} attribute.

  If no dot is present, the entire \member{tp_name} field is made
  accessible as the \member{__name__} attribute, and the
  \member{__module__} attribute is undefined (unless explicitly set in
  the dictionary, as explained above).  This means your type will be
  impossible to pickle.

  This field is not inherited by subtypes.
\end{cmemberdesc}

\begin{cmemberdesc}{PyTypeObject}{Py_ssize_t}{tp_basicsize}
\cmemberline{PyTypeObject}{Py_ssize_t}{tp_itemsize}
  These fields allow calculating the size in bytes of instances of
  the type.

  There are two kinds of types: types with fixed-length instances have
  a zero \member{tp_itemsize} field, types with variable-length
  instances have a non-zero \member{tp_itemsize} field.  For a type
  with fixed-length instances, all instances have the same size,
  given in \member{tp_basicsize}.

  For a type with variable-length instances, the instances must have
  an \member{ob_size} field, and the instance size is
  \member{tp_basicsize} plus N times \member{tp_itemsize}, where N is
  the ``length'' of the object.  The value of N is typically stored in
  the instance's \member{ob_size} field.  There are exceptions:  for
  example, long ints use a negative \member{ob_size} to indicate a
  negative number, and N is \code{abs(\member{ob_size})} there.  Also,
  the presence of an \member{ob_size} field in the instance layout
  doesn't mean that the instance structure is variable-length (for
  example, the structure for the list type has fixed-length instances,
  yet those instances have a meaningful \member{ob_size} field).

  The basic size includes the fields in the instance declared by the
  macro \csimplemacro{PyObject_HEAD} or
  \csimplemacro{PyObject_VAR_HEAD} (whichever is used to declare the
  instance struct) and this in turn includes the \member{_ob_prev} and
  \member{_ob_next} fields if they are present.  This means that the
  only correct way to get an initializer for the \member{tp_basicsize}
  is to use the \keyword{sizeof} operator on the struct used to
  declare the instance layout.  The basic size does not include the GC
  header size (this is new in Python 2.2; in 2.1 and 2.0, the GC
  header size was included in \member{tp_basicsize}).

  These fields are inherited separately by subtypes.  If the base type
  has a non-zero \member{tp_itemsize}, it is generally not safe to set
  \member{tp_itemsize} to a different non-zero value in a subtype
  (though this depends on the implementation of the base type).

  A note about alignment: if the variable items require a particular
  alignment, this should be taken care of by the value of
  \member{tp_basicsize}.  Example: suppose a type implements an array
  of \code{double}. \member{tp_itemsize} is \code{sizeof(double)}.
  It is the programmer's responsibility that \member{tp_basicsize} is
  a multiple of \code{sizeof(double)} (assuming this is the alignment
  requirement for \code{double}).
\end{cmemberdesc}

\begin{cmemberdesc}{PyTypeObject}{destructor}{tp_dealloc}
  A pointer to the instance destructor function.  This function must
  be defined unless the type guarantees that its instances will never
  be deallocated (as is the case for the singletons \code{None} and
  \code{Ellipsis}).

  The destructor function is called by the \cfunction{Py_DECREF()} and
  \cfunction{Py_XDECREF()} macros when the new reference count is
  zero.  At this point, the instance is still in existence, but there
  are no references to it.  The destructor function should free all
  references which the instance owns, free all memory buffers owned by
  the instance (using the freeing function corresponding to the
  allocation function used to allocate the buffer), and finally (as
  its last action) call the type's \member{tp_free} function.  If the
  type is not subtypable (doesn't have the
  \constant{Py_TPFLAGS_BASETYPE} flag bit set), it is permissible to
  call the object deallocator directly instead of via
  \member{tp_free}.  The object deallocator should be the one used to
  allocate the instance; this is normally \cfunction{PyObject_Del()}
  if the instance was allocated using \cfunction{PyObject_New()} or
  \cfunction{PyObject_VarNew()}, or \cfunction{PyObject_GC_Del()} if
  the instance was allocated using \cfunction{PyObject_GC_New()} or
  \cfunction{PyObject_GC_VarNew()}.

  This field is inherited by subtypes.
\end{cmemberdesc}

\begin{cmemberdesc}{PyTypeObject}{printfunc}{tp_print}
  An optional pointer to the instance print function.

  The print function is only called when the instance is printed to a
  \emph{real} file; when it is printed to a pseudo-file (like a
  \class{StringIO} instance), the instance's \member{tp_repr} or
  \member{tp_str} function is called to convert it to a string.  These
  are also called when the type's \member{tp_print} field is \NULL.  A
  type should never implement \member{tp_print} in a way that produces
  different output than \member{tp_repr} or \member{tp_str} would.

  The print function is called with the same signature as
  \cfunction{PyObject_Print()}: \code{int tp_print(PyObject *self, FILE
  *file, int flags)}.  The \var{self} argument is the instance to be
  printed.  The \var{file} argument is the stdio file to which it is
  to be printed.  The \var{flags} argument is composed of flag bits.
  The only flag bit currently defined is \constant{Py_PRINT_RAW}.
  When the \constant{Py_PRINT_RAW} flag bit is set, the instance
  should be printed the same way as \member{tp_str} would format it;
  when the \constant{Py_PRINT_RAW} flag bit is clear, the instance
  should be printed the same was as \member{tp_repr} would format it.
  It should return \code{-1} and set an exception condition when an
  error occurred during the comparison.

  It is possible that the \member{tp_print} field will be deprecated.
  In any case, it is recommended not to define \member{tp_print}, but
  instead to rely on \member{tp_repr} and \member{tp_str} for
  printing.

  This field is inherited by subtypes.
\end{cmemberdesc}

\begin{cmemberdesc}{PyTypeObject}{getattrfunc}{tp_getattr}
  An optional pointer to the get-attribute-string function.

  This field is deprecated.  When it is defined, it should point to a
  function that acts the same as the \member{tp_getattro} function,
  but taking a C string instead of a Python string object to give the
  attribute name.  The signature is the same as for
  \cfunction{PyObject_GetAttrString()}.

  This field is inherited by subtypes together with
  \member{tp_getattro}: a subtype inherits both \member{tp_getattr}
  and \member{tp_getattro} from its base type when the subtype's
  \member{tp_getattr} and \member{tp_getattro} are both \NULL.
\end{cmemberdesc}

\begin{cmemberdesc}{PyTypeObject}{setattrfunc}{tp_setattr}
  An optional pointer to the set-attribute-string function.

  This field is deprecated.  When it is defined, it should point to a
  function that acts the same as the \member{tp_setattro} function,
  but taking a C string instead of a Python string object to give the
  attribute name.  The signature is the same as for
  \cfunction{PyObject_SetAttrString()}.

  This field is inherited by subtypes together with
  \member{tp_setattro}: a subtype inherits both \member{tp_setattr}
  and \member{tp_setattro} from its base type when the subtype's
  \member{tp_setattr} and \member{tp_setattro} are both \NULL.
\end{cmemberdesc}

\begin{cmemberdesc}{PyTypeObject}{cmpfunc}{tp_compare}
  An optional pointer to the three-way comparison function.

  The signature is the same as for \cfunction{PyObject_Compare()}.
  The function should return \code{1} if \var{self} greater than
  \var{other}, \code{0} if \var{self} is equal to \var{other}, and
  \code{-1} if \var{self} less than \var{other}.  It should return
  \code{-1} and set an exception condition when an error occurred
  during the comparison.

  This field is inherited by subtypes together with
  \member{tp_richcompare} and \member{tp_hash}: a subtypes inherits
  all three of \member{tp_compare}, \member{tp_richcompare}, and
  \member{tp_hash} when the subtype's \member{tp_compare},
  \member{tp_richcompare}, and \member{tp_hash} are all \NULL.
\end{cmemberdesc}

\begin{cmemberdesc}{PyTypeObject}{reprfunc}{tp_repr}
  An optional pointer to a function that implements the built-in
  function \function{repr()}.\bifuncindex{repr}

  The signature is the same as for \cfunction{PyObject_Repr()}; it
  must return a string or a Unicode object.  Ideally, this function
  should return a string that, when passed to \function{eval()}, given
  a suitable environment, returns an object with the same value.  If
  this is not feasible, it should return a string starting with
  \character{\textless} and ending with \character{\textgreater} from
  which both the type and the value of the object can be deduced.

  When this field is not set, a string of the form \samp{<\%s object
  at \%p>} is returned, where \code{\%s} is replaced by the type name,
  and \code{\%p} by the object's memory address.

  This field is inherited by subtypes.
\end{cmemberdesc}

PyNumberMethods *tp_as_number;

    XXX

PySequenceMethods *tp_as_sequence;

    XXX

PyMappingMethods *tp_as_mapping;

    XXX

\begin{cmemberdesc}{PyTypeObject}{hashfunc}{tp_hash}
  An optional pointer to a function that implements the built-in
  function \function{hash()}.\bifuncindex{hash}

  The signature is the same as for \cfunction{PyObject_Hash()}; it
  must return a C long.  The value \code{-1} should not be returned as
  a normal return value; when an error occurs during the computation
  of the hash value, the function should set an exception and return
  \code{-1}.

  When this field is not set, two possibilities exist: if the
  \member{tp_compare} and \member{tp_richcompare} fields are both
  \NULL, a default hash value based on the object's address is
  returned; otherwise, a \exception{TypeError} is raised.

  This field is inherited by subtypes together with
  \member{tp_richcompare} and \member{tp_compare}: a subtypes inherits
  all three of \member{tp_compare}, \member{tp_richcompare}, and
  \member{tp_hash}, when the subtype's \member{tp_compare},
  \member{tp_richcompare} and \member{tp_hash} are all \NULL.
\end{cmemberdesc}

\begin{cmemberdesc}{PyTypeObject}{ternaryfunc}{tp_call}
  An optional pointer to a function that implements calling the
  object.  This should be \NULL{} if the object is not callable.  The
  signature is the same as for \cfunction{PyObject_Call()}.

  This field is inherited by subtypes.
\end{cmemberdesc}

\begin{cmemberdesc}{PyTypeObject}{reprfunc}{tp_str}
  An optional pointer to a function that implements the built-in
  operation \function{str()}.  (Note that \class{str} is a type now,
  and \function{str()} calls the constructor for that type.  This
  constructor calls \cfunction{PyObject_Str()} to do the actual work,
  and \cfunction{PyObject_Str()} will call this handler.)

  The signature is the same as for \cfunction{PyObject_Str()}; it must
  return a string or a Unicode object.  This function should return a
  ``friendly'' string representation of the object, as this is the
  representation that will be used by the print statement.

  When this field is not set, \cfunction{PyObject_Repr()} is called to
  return a string representation.

  This field is inherited by subtypes.
\end{cmemberdesc}

\begin{cmemberdesc}{PyTypeObject}{getattrofunc}{tp_getattro}
  An optional pointer to the get-attribute function.

  The signature is the same as for \cfunction{PyObject_GetAttr()}.  It
  is usually convenient to set this field to
  \cfunction{PyObject_GenericGetAttr()}, which implements the normal
  way of looking for object attributes.

  This field is inherited by subtypes together with
  \member{tp_getattr}: a subtype inherits both \member{tp_getattr} and
  \member{tp_getattro} from its base type when the subtype's
  \member{tp_getattr} and \member{tp_getattro} are both \NULL.
\end{cmemberdesc}

\begin{cmemberdesc}{PyTypeObject}{setattrofunc}{tp_setattro}
  An optional pointer to the set-attribute function.

  The signature is the same as for \cfunction{PyObject_SetAttr()}.  It
  is usually convenient to set this field to
  \cfunction{PyObject_GenericSetAttr()}, which implements the normal
  way of setting object attributes.

  This field is inherited by subtypes together with
  \member{tp_setattr}: a subtype inherits both \member{tp_setattr} and
  \member{tp_setattro} from its base type when the subtype's
  \member{tp_setattr} and \member{tp_setattro} are both \NULL.
\end{cmemberdesc}

\begin{cmemberdesc}{PyTypeObject}{PyBufferProcs*}{tp_as_buffer}
  Pointer to an additional structure that contains fields relevant only to
  objects which implement the buffer interface.  These fields are
  documented in ``Buffer Object Structures'' (section
  \ref{buffer-structs}).

  The \member{tp_as_buffer} field is not inherited, but the contained
  fields are inherited individually.
\end{cmemberdesc}

\begin{cmemberdesc}{PyTypeObject}{long}{tp_flags}
  This field is a bit mask of various flags.  Some flags indicate
  variant semantics for certain situations; others are used to
  indicate that certain fields in the type object (or in the extension
  structures referenced via \member{tp_as_number},
  \member{tp_as_sequence}, \member{tp_as_mapping}, and
  \member{tp_as_buffer}) that were historically not always present are
  valid; if such a flag bit is clear, the type fields it guards must
  not be accessed and must be considered to have a zero or \NULL{}
  value instead.

  Inheritance of this field is complicated.  Most flag bits are
  inherited individually, i.e. if the base type has a flag bit set,
  the subtype inherits this flag bit.  The flag bits that pertain to
  extension structures are strictly inherited if the extension
  structure is inherited, i.e. the base type's value of the flag bit
  is copied into the subtype together with a pointer to the extension
  structure.  The \constant{Py_TPFLAGS_HAVE_GC} flag bit is inherited
  together with the \member{tp_traverse} and \member{tp_clear} fields,
  i.e. if the \constant{Py_TPFLAGS_HAVE_GC} flag bit is clear in the
  subtype and the \member{tp_traverse} and \member{tp_clear} fields in
  the subtype exist (as indicated by the
  \constant{Py_TPFLAGS_HAVE_RICHCOMPARE} flag bit) and have \NULL{}
  values.

  The following bit masks are currently defined; these can be or-ed
  together using the \code{|} operator to form the value of the
  \member{tp_flags} field.  The macro \cfunction{PyType_HasFeature()}
  takes a type and a flags value, \var{tp} and \var{f}, and checks
  whether \code{\var{tp}->tp_flags \& \var{f}} is non-zero.

  \begin{datadesc}{Py_TPFLAGS_HAVE_GETCHARBUFFER}
    If this bit is set, the \ctype{PyBufferProcs} struct referenced by
    \member{tp_as_buffer} has the \member{bf_getcharbuffer} field.
  \end{datadesc}

  \begin{datadesc}{Py_TPFLAGS_HAVE_SEQUENCE_IN}
    If this bit is set, the \ctype{PySequenceMethods} struct
    referenced by \member{tp_as_sequence} has the \member{sq_contains}
    field.
  \end{datadesc}

  \begin{datadesc}{Py_TPFLAGS_GC}
    This bit is obsolete.  The bit it used to name is no longer in
    use.  The symbol is now defined as zero.
  \end{datadesc}

  \begin{datadesc}{Py_TPFLAGS_HAVE_INPLACEOPS}
    If this bit is set, the \ctype{PySequenceMethods} struct
    referenced by \member{tp_as_sequence} and the
    \ctype{PyNumberMethods} structure referenced by
    \member{tp_as_number} contain the fields for in-place operators.
    In particular, this means that the \ctype{PyNumberMethods}
    structure has the fields \member{nb_inplace_add},
    \member{nb_inplace_subtract}, \member{nb_inplace_multiply},
    \member{nb_inplace_divide}, \member{nb_inplace_remainder},
    \member{nb_inplace_power}, \member{nb_inplace_lshift},
    \member{nb_inplace_rshift}, \member{nb_inplace_and},
    \member{nb_inplace_xor}, and \member{nb_inplace_or}; and the
    \ctype{PySequenceMethods} struct has the fields
    \member{sq_inplace_concat} and \member{sq_inplace_repeat}.
  \end{datadesc}

  \begin{datadesc}{Py_TPFLAGS_CHECKTYPES}
    If this bit is set, the binary and ternary operations in the
    \ctype{PyNumberMethods} structure referenced by
    \member{tp_as_number} accept arguments of arbitrary object types,
    and do their own type conversions if needed.  If this bit is
    clear, those operations require that all arguments have the
    current type as their type, and the caller is supposed to perform
    a coercion operation first.  This applies to \member{nb_add},
    \member{nb_subtract}, \member{nb_multiply}, \member{nb_divide},
    \member{nb_remainder}, \member{nb_divmod}, \member{nb_power},
    \member{nb_lshift}, \member{nb_rshift}, \member{nb_and},
    \member{nb_xor}, and \member{nb_or}.
  \end{datadesc}

  \begin{datadesc}{Py_TPFLAGS_HAVE_RICHCOMPARE}
    If this bit is set, the type object has the
    \member{tp_richcompare} field, as well as the \member{tp_traverse}
    and the \member{tp_clear} fields.
  \end{datadesc}

  \begin{datadesc}{Py_TPFLAGS_HAVE_WEAKREFS}
    If this bit is set, the \member{tp_weaklistoffset} field is
    defined.  Instances of a type are weakly referenceable if the
    type's \member{tp_weaklistoffset} field has a value greater than
    zero.
  \end{datadesc}

  \begin{datadesc}{Py_TPFLAGS_HAVE_ITER}
    If this bit is set, the type object has the \member{tp_iter} and
    \member{tp_iternext} fields.
  \end{datadesc}

  \begin{datadesc}{Py_TPFLAGS_HAVE_CLASS}
    If this bit is set, the type object has several new fields defined
    starting in Python 2.2: \member{tp_methods}, \member{tp_members},
    \member{tp_getset}, \member{tp_base}, \member{tp_dict},
    \member{tp_descr_get}, \member{tp_descr_set},
    \member{tp_dictoffset}, \member{tp_init}, \member{tp_alloc},
    \member{tp_new}, \member{tp_free}, \member{tp_is_gc},
    \member{tp_bases}, \member{tp_mro}, \member{tp_cache},
    \member{tp_subclasses}, and \member{tp_weaklist}.
  \end{datadesc}

  \begin{datadesc}{Py_TPFLAGS_HEAPTYPE}
    This bit is set when the type object itself is allocated on the
    heap.  In this case, the \member{ob_type} field of its instances
    is considered a reference to the type, and the type object is
    INCREF'ed when a new instance is created, and DECREF'ed when an
    instance is destroyed (this does not apply to instances of
    subtypes; only the type referenced by the instance's ob_type gets
    INCREF'ed or DECREF'ed).
  \end{datadesc}

  \begin{datadesc}{Py_TPFLAGS_BASETYPE}
    This bit is set when the type can be used as the base type of
    another type.  If this bit is clear, the type cannot be subtyped
    (similar to a "final" class in Java).
  \end{datadesc}

  \begin{datadesc}{Py_TPFLAGS_READY}
    This bit is set when the type object has been fully initialized by
    \cfunction{PyType_Ready()}.
  \end{datadesc}

  \begin{datadesc}{Py_TPFLAGS_READYING}
    This bit is set while \cfunction{PyType_Ready()} is in the process
    of initializing the type object.
  \end{datadesc}

  \begin{datadesc}{Py_TPFLAGS_HAVE_GC}
    This bit is set when the object supports garbage collection.  If
    this bit is set, instances must be created using
    \cfunction{PyObject_GC_New()} and destroyed using
    \cfunction{PyObject_GC_Del()}.  More information in section XXX
    about garbage collection.  This bit also implies that the
    GC-related fields \member{tp_traverse} and \member{tp_clear} are
    present in the type object; but those fields also exist when
    \constant{Py_TPFLAGS_HAVE_GC} is clear but
    \constant{Py_TPFLAGS_HAVE_RICHCOMPARE} is set.
  \end{datadesc}

  \begin{datadesc}{Py_TPFLAGS_DEFAULT}
    This is a bitmask of all the bits that pertain to the existence of
    certain fields in the type object and its extension structures.
    Currently, it includes the following bits:
    \constant{Py_TPFLAGS_HAVE_GETCHARBUFFER},
    \constant{Py_TPFLAGS_HAVE_SEQUENCE_IN},
    \constant{Py_TPFLAGS_HAVE_INPLACEOPS},
    \constant{Py_TPFLAGS_HAVE_RICHCOMPARE},
    \constant{Py_TPFLAGS_HAVE_WEAKREFS},
    \constant{Py_TPFLAGS_HAVE_ITER}, and
    \constant{Py_TPFLAGS_HAVE_CLASS}.
  \end{datadesc}
\end{cmemberdesc}

\begin{cmemberdesc}{PyTypeObject}{char*}{tp_doc}
  An optional pointer to a NUL-terminated C string giving the
  docstring for this type object.  This is exposed as the
  \member{__doc__} attribute on the type and instances of the type.

  This field is \emph{not} inherited by subtypes.
\end{cmemberdesc}

The following three fields only exist if the
\constant{Py_TPFLAGS_HAVE_RICHCOMPARE} flag bit is set.

\begin{cmemberdesc}{PyTypeObject}{traverseproc}{tp_traverse}
  An optional pointer to a traversal function for the garbage
  collector.  This is only used if the \constant{Py_TPFLAGS_HAVE_GC}
  flag bit is set.  More information about Python's garbage collection
  scheme can be found in section \ref{supporting-cycle-detection}.

  The \member{tp_traverse} pointer is used by the garbage collector
  to detect reference cycles. A typical implementation of a
  \member{tp_traverse} function simply calls \cfunction{Py_VISIT()} on
  each of the instance's members that are Python objects.  For exampe, this
  is function \cfunction{local_traverse} from the \module{thread} extension
  module:

  \begin{verbatim}
  static int
  local_traverse(localobject *self, visitproc visit, void *arg)
  {
      Py_VISIT(self->args);
      Py_VISIT(self->kw);
      Py_VISIT(self->dict);
      return 0;
  }
  \end{verbatim}

  Note that \cfunction{Py_VISIT()} is called only on those members that can
  participate in reference cycles.  Although there is also a
  \samp{self->key} member, it can only be \NULL{} or a Python string and
  therefore cannot be part of a reference cycle.

  On the other hand, even if you know a member can never be part of a cycle,
  as a debugging aid you may want to visit it anyway just so the
  \module{gc} module's \function{get_referents()} function will include it.

  Note that \cfunction{Py_VISIT()} requires the \var{visit} and \var{arg}
  parameters to \cfunction{local_traverse} to have these specific names;
  don't name them just anything.

  This field is inherited by subtypes together with \member{tp_clear}
  and the \constant{Py_TPFLAGS_HAVE_GC} flag bit: the flag bit,
  \member{tp_traverse}, and \member{tp_clear} are all inherited from
  the base type if they are all zero in the subtype \emph{and} the
  subtype has the \constant{Py_TPFLAGS_HAVE_RICHCOMPARE} flag bit set.
\end{cmemberdesc}

\begin{cmemberdesc}{PyTypeObject}{inquiry}{tp_clear}
  An optional pointer to a clear function for the garbage collector.
  This is only used if the \constant{Py_TPFLAGS_HAVE_GC} flag bit is
  set.

  The \member{tp_clear} member function is used to break reference
  cycles in cyclic garbage detected by the garbage collector.  Taken
  together, all \member{tp_clear} functions in the system must combine to
  break all reference cycles.  This is subtle, and if in any doubt supply a
  \member{tp_clear} function.  For example, the tuple type does not
  implement a \member{tp_clear} function, because it's possible to prove
  that no reference cycle can be composed entirely of tuples.  Therefore
  the \member{tp_clear} functions of other types must be sufficient to
  break any cycle containing a tuple.  This isn't immediately obvious, and
  there's rarely a good reason to avoid implementing \member{tp_clear}.

  Implementations of \member{tp_clear} should drop the instance's
  references to those of its members that may be Python objects, and set
  its pointers to those members to \NULL{}, as in the following example:

  \begin{verbatim}
  static int
  local_clear(localobject *self)
  {
      Py_CLEAR(self->key);
      Py_CLEAR(self->args);
      Py_CLEAR(self->kw);
      Py_CLEAR(self->dict);
      return 0;
  }
  \end{verbatim}

  The \cfunction{Py_CLEAR()} macro should be used, because clearing
  references is delicate:  the reference to the contained object must not be
  decremented until after the pointer to the contained object is set to
  \NULL{}.  This is because decrementing the reference count may cause
  the contained object to become trash, triggering a chain of reclamation
  activity that may include invoking arbitrary Python code (due to
  finalizers, or weakref callbacks, associated with the contained object).
  If it's possible for such code to reference \var{self} again, it's
  important that the pointer to the contained object be \NULL{} at that
  time, so that \var{self} knows the contained object can no longer be
  used.  The \cfunction{Py_CLEAR()} macro performs the operations in a
  safe order.

  Because the goal of \member{tp_clear} functions is to break reference
  cycles, it's not necessary to clear contained objects like Python strings
  or Python integers, which can't participate in reference cycles.
  On the other hand, it may be convenient to clear all contained Python
  objects, and write the type's \member{tp_dealloc} function to
  invoke \member{tp_clear}.

  More information about Python's garbage collection
  scheme can be found in section \ref{supporting-cycle-detection}.

  This field is inherited by subtypes together with \member{tp_traverse}
  and the \constant{Py_TPFLAGS_HAVE_GC} flag bit: the flag bit,
  \member{tp_traverse}, and \member{tp_clear} are all inherited from
  the base type if they are all zero in the subtype \emph{and} the
  subtype has the \constant{Py_TPFLAGS_HAVE_RICHCOMPARE} flag bit set.
\end{cmemberdesc}

\begin{cmemberdesc}{PyTypeObject}{richcmpfunc}{tp_richcompare}
  An optional pointer to the rich comparison function.

  The signature is the same as for \cfunction{PyObject_RichCompare()}.
  The function should return the result of the comparison (usually
  \code{Py_True} or \code{Py_False}).  If the comparison is undefined,
  it must return \code{Py_NotImplemented}, if another error occurred
  it must return \code{NULL} and set an exception condition.

  This field is inherited by subtypes together with
  \member{tp_compare} and \member{tp_hash}: a subtype inherits all
  three of \member{tp_compare}, \member{tp_richcompare}, and
  \member{tp_hash}, when the subtype's \member{tp_compare},
  \member{tp_richcompare}, and \member{tp_hash} are all \NULL.

  The following constants are defined to be used as the third argument
  for \member{tp_richcompare} and for \cfunction{PyObject_RichCompare()}:

  \begin{tableii}{l|c}{constant}{Constant}{Comparison}
    \lineii{Py_LT}{\code{<}}
    \lineii{Py_LE}{\code{<=}}
    \lineii{Py_EQ}{\code{==}}
    \lineii{Py_NE}{\code{!=}}
    \lineii{Py_GT}{\code{>}}
    \lineii{Py_GE}{\code{>=}}
  \end{tableii}
\end{cmemberdesc}

The next field only exists if the \constant{Py_TPFLAGS_HAVE_WEAKREFS}
flag bit is set.

\begin{cmemberdesc}{PyTypeObject}{long}{tp_weaklistoffset}
  If the instances of this type are weakly referenceable, this field
  is greater than zero and contains the offset in the instance
  structure of the weak reference list head (ignoring the GC header,
  if present); this offset is used by
  \cfunction{PyObject_ClearWeakRefs()} and the
  \cfunction{PyWeakref_*()} functions.  The instance structure needs
  to include a field of type \ctype{PyObject*} which is initialized to
  \NULL.

  Do not confuse this field with \member{tp_weaklist}; that is the
  list head for weak references to the type object itself.

  This field is inherited by subtypes, but see the rules listed below.
  A subtype may override this offset; this means that the subtype uses
  a different weak reference list head than the base type.  Since the
  list head is always found via \member{tp_weaklistoffset}, this
  should not be a problem.

  When a type defined by a class statement has no \member{__slots__}
  declaration, and none of its base types are weakly referenceable,
  the type is made weakly referenceable by adding a weak reference
  list head slot to the instance layout and setting the
  \member{tp_weaklistoffset} of that slot's offset.

  When a type's \member{__slots__} declaration contains a slot named
  \member{__weakref__}, that slot becomes the weak reference list head
  for instances of the type, and the slot's offset is stored in the
  type's \member{tp_weaklistoffset}.

  When a type's \member{__slots__} declaration does not contain a slot
  named \member{__weakref__}, the type inherits its
  \member{tp_weaklistoffset} from its base type.
\end{cmemberdesc}

The next two fields only exist if the
\constant{Py_TPFLAGS_HAVE_CLASS} flag bit is set.

\begin{cmemberdesc}{PyTypeObject}{getiterfunc}{tp_iter}
  An optional pointer to a function that returns an iterator for the
  object.  Its presence normally signals that the instances of this
  type are iterable (although sequences may be iterable without this
  function, and classic instances always have this function, even if
  they don't define an \method{__iter__()} method).

  This function has the same signature as
  \cfunction{PyObject_GetIter()}.

  This field is inherited by subtypes.
\end{cmemberdesc}

\begin{cmemberdesc}{PyTypeObject}{iternextfunc}{tp_iternext}
  An optional pointer to a function that returns the next item in an
  iterator, or raises \exception{StopIteration} when the iterator is
  exhausted.  Its presence normally signals that the instances of this
  type are iterators (although classic instances always have this
  function, even if they don't define a \method{next()} method).

  Iterator types should also define the \member{tp_iter} function, and
  that function should return the iterator instance itself (not a new
  iterator instance).

  This function has the same signature as \cfunction{PyIter_Next()}.

  This field is inherited by subtypes.
\end{cmemberdesc}

The next fields, up to and including \member{tp_weaklist}, only exist
if the \constant{Py_TPFLAGS_HAVE_CLASS} flag bit is set.

\begin{cmemberdesc}{PyTypeObject}{struct PyMethodDef*}{tp_methods}
  An optional pointer to a static \NULL-terminated array of
  \ctype{PyMethodDef} structures, declaring regular methods of this
  type.

  For each entry in the array, an entry is added to the type's
  dictionary (see \member{tp_dict} below) containing a method
  descriptor.

  This field is not inherited by subtypes (methods are
  inherited through a different mechanism).
\end{cmemberdesc}

\begin{cmemberdesc}{PyTypeObject}{struct PyMemberDef*}{tp_members}
  An optional pointer to a static \NULL-terminated array of
  \ctype{PyMemberDef} structures, declaring regular data members
  (fields or slots) of instances of this type.

  For each entry in the array, an entry is added to the type's
  dictionary (see \member{tp_dict} below) containing a member
  descriptor.

  This field is not inherited by subtypes (members are inherited
  through a different mechanism).
\end{cmemberdesc}

\begin{cmemberdesc}{PyTypeObject}{struct PyGetSetDef*}{tp_getset}
  An optional pointer to a static \NULL-terminated array of
  \ctype{PyGetSetDef} structures, declaring computed attributes of
  instances of this type.

  For each entry in the array, an entry is added to the type's
  dictionary (see \member{tp_dict} below) containing a getset
  descriptor.

  This field is not inherited by subtypes (computed attributes are
  inherited through a different mechanism).

  Docs for PyGetSetDef (XXX belong elsewhere):

\begin{verbatim}
typedef PyObject *(*getter)(PyObject *, void *);
typedef int (*setter)(PyObject *, PyObject *, void *);

typedef struct PyGetSetDef {
    char *name;    /* attribute name */
    getter get;    /* C function to get the attribute */
    setter set;    /* C function to set the attribute */
    char *doc;     /* optional doc string */
    void *closure; /* optional additional data for getter and setter */
} PyGetSetDef;
\end{verbatim}
\end{cmemberdesc}

\begin{cmemberdesc}{PyTypeObject}{PyTypeObject*}{tp_base}
  An optional pointer to a base type from which type properties are
  inherited.  At this level, only single inheritance is supported;
  multiple inheritance require dynamically creating a type object by
  calling the metatype.

  This field is not inherited by subtypes (obviously), but it defaults
  to \code{\&PyBaseObject_Type} (which to Python programmers is known
  as the type \class{object}).
\end{cmemberdesc}

\begin{cmemberdesc}{PyTypeObject}{PyObject*}{tp_dict}
  The type's dictionary is stored here by \cfunction{PyType_Ready()}.

  This field should normally be initialized to \NULL{} before
  PyType_Ready is called; it may also be initialized to a dictionary
  containing initial attributes for the type.  Once
  \cfunction{PyType_Ready()} has initialized the type, extra
  attributes for the type may be added to this dictionary only if they
  don't correspond to overloaded operations (like \method{__add__()}).

  This field is not inherited by subtypes (though the attributes
  defined in here are inherited through a different mechanism).
\end{cmemberdesc}

\begin{cmemberdesc}{PyTypeObject}{descrgetfunc}{tp_descr_get}
  An optional pointer to a "descriptor get" function.


  The function signature is

\begin{verbatim}
PyObject * tp_descr_get(PyObject *self, PyObject *obj, PyObject *type);
\end{verbatim}

  XXX blah, blah.

  This field is inherited by subtypes.
\end{cmemberdesc}

\begin{cmemberdesc}{PyTypeObject}{descrsetfunc}{tp_descr_set}
  An optional pointer to a "descriptor set" function.

  The function signature is

\begin{verbatim}
int tp_descr_set(PyObject *self, PyObject *obj, PyObject *value);
\end{verbatim}

  This field is inherited by subtypes.

  XXX blah, blah.

\end{cmemberdesc}

\begin{cmemberdesc}{PyTypeObject}{long}{tp_dictoffset}
  If the instances of this type have a dictionary containing instance
  variables, this field is non-zero and contains the offset in the
  instances of the type of the instance variable dictionary; this
  offset is used by \cfunction{PyObject_GenericGetAttr()}.

  Do not confuse this field with \member{tp_dict}; that is the
  dictionary for attributes of the type object itself.

  If the value of this field is greater than zero, it specifies the
  offset from the start of the instance structure.  If the value is
  less than zero, it specifies the offset from the \emph{end} of the
  instance structure.  A negative offset is more expensive to use, and
  should only be used when the instance structure contains a
  variable-length part.  This is used for example to add an instance
  variable dictionary to subtypes of \class{str} or \class{tuple}.
  Note that the \member{tp_basicsize} field should account for the
  dictionary added to the end in that case, even though the dictionary
  is not included in the basic object layout.  On a system with a
  pointer size of 4 bytes, \member{tp_dictoffset} should be set to
  \code{-4} to indicate that the dictionary is at the very end of the
  structure.

  The real dictionary offset in an instance can be computed from a
  negative \member{tp_dictoffset} as follows:

\begin{verbatim}
dictoffset = tp_basicsize + abs(ob_size)*tp_itemsize + tp_dictoffset
if dictoffset is not aligned on sizeof(void*):
    round up to sizeof(void*)
\end{verbatim}

  where \member{tp_basicsize}, \member{tp_itemsize} and
  \member{tp_dictoffset} are taken from the type object, and
  \member{ob_size} is taken from the instance.  The absolute value is
  taken because long ints use the sign of \member{ob_size} to store
  the sign of the number.  (There's never a need to do this
  calculation yourself; it is done for you by
  \cfunction{_PyObject_GetDictPtr()}.)

  This field is inherited by subtypes, but see the rules listed below.
  A subtype may override this offset; this means that the subtype
  instances store the dictionary at a difference offset than the base
  type.  Since the dictionary is always found via
  \member{tp_dictoffset}, this should not be a problem.

  When a type defined by a class statement has no \member{__slots__}
  declaration, and none of its base types has an instance variable
  dictionary, a dictionary slot is added to the instance layout and
  the \member{tp_dictoffset} is set to that slot's offset.

  When a type defined by a class statement has a \member{__slots__}
  declaration, the type inherits its \member{tp_dictoffset} from its
  base type.

  (Adding a slot named \member{__dict__} to the \member{__slots__}
  declaration does not have the expected effect, it just causes
  confusion.  Maybe this should be added as a feature just like
  \member{__weakref__} though.)
\end{cmemberdesc}

\begin{cmemberdesc}{PyTypeObject}{initproc}{tp_init}
  An optional pointer to an instance initialization function.

  This function corresponds to the \method{__init__()} method of
  classes.  Like \method{__init__()}, it is possible to create an
  instance without calling \method{__init__()}, and it is possible to
  reinitialize an instance by calling its \method{__init__()} method
  again.

  The function signature is

\begin{verbatim}
int tp_init(PyObject *self, PyObject *args, PyObject *kwds)
\end{verbatim}

  The self argument is the instance to be initialized; the \var{args}
  and \var{kwds} arguments represent positional and keyword arguments
  of the call to \method{__init__()}.

  The \member{tp_init} function, if not \NULL, is called when an
  instance is created normally by calling its type, after the type's
  \member{tp_new} function has returned an instance of the type.  If
  the \member{tp_new} function returns an instance of some other type
  that is not a subtype of the original type, no \member{tp_init}
  function is called; if \member{tp_new} returns an instance of a
  subtype of the original type, the subtype's \member{tp_init} is
  called.  (VERSION NOTE: described here is what is implemented in
  Python 2.2.1 and later.  In Python 2.2, the \member{tp_init} of the
  type of the object returned by \member{tp_new} was always called, if
  not \NULL.)

  This field is inherited by subtypes.
\end{cmemberdesc}

\begin{cmemberdesc}{PyTypeObject}{allocfunc}{tp_alloc}
  An optional pointer to an instance allocation function.

  The function signature is

\begin{verbatim}
PyObject *tp_alloc(PyTypeObject *self, Py_ssize_t nitems)
\end{verbatim}

  The purpose of this function is to separate memory allocation from
  memory initialization.  It should return a pointer to a block of
  memory of adequate length for the instance, suitably aligned, and
  initialized to zeros, but with \member{ob_refcnt} set to \code{1}
  and \member{ob_type} set to the type argument.  If the type's
  \member{tp_itemsize} is non-zero, the object's \member{ob_size} field
  should be initialized to \var{nitems} and the length of the
  allocated memory block should be \code{tp_basicsize +
  \var{nitems}*tp_itemsize}, rounded up to a multiple of
  \code{sizeof(void*)}; otherwise, \var{nitems} is not used and the
  length of the block should be \member{tp_basicsize}.

  Do not use this function to do any other instance initialization,
  not even to allocate additional memory; that should be done by
  \member{tp_new}.

  This field is inherited by static subtypes, but not by dynamic
  subtypes (subtypes created by a class statement); in the latter,
  this field is always set to \cfunction{PyType_GenericAlloc()}, to
  force a standard heap allocation strategy.  That is also the
  recommended value for statically defined types.
\end{cmemberdesc}

\begin{cmemberdesc}{PyTypeObject}{newfunc}{tp_new}
  An optional pointer to an instance creation function.

  If this function is \NULL{} for a particular type, that type cannot
  be called to create new instances; presumably there is some other
  way to create instances, like a factory function.

  The function signature is

\begin{verbatim}
PyObject *tp_new(PyTypeObject *subtype, PyObject *args, PyObject *kwds)
\end{verbatim}

  The subtype argument is the type of the object being created; the
  \var{args} and \var{kwds} arguments represent positional and keyword
  arguments of the call to the type.  Note that subtype doesn't have
  to equal the type whose \member{tp_new} function is called; it may
  be a subtype of that type (but not an unrelated type).

  The \member{tp_new} function should call
  \code{\var{subtype}->tp_alloc(\var{subtype}, \var{nitems})} to
  allocate space for the object, and then do only as much further
  initialization as is absolutely necessary.  Initialization that can
  safely be ignored or repeated should be placed in the
  \member{tp_init} handler.  A good rule of thumb is that for
  immutable types, all initialization should take place in
  \member{tp_new}, while for mutable types, most initialization should
  be deferred to \member{tp_init}.

  This field is inherited by subtypes, except it is not inherited by
  static types whose \member{tp_base} is \NULL{} or
  \code{\&PyBaseObject_Type}.  The latter exception is a precaution so
  that old extension types don't become callable simply by being
  linked with Python 2.2.
\end{cmemberdesc}

\begin{cmemberdesc}{PyTypeObject}{destructor}{tp_free}
  An optional pointer to an instance deallocation function.

  The signature of this function has changed slightly: in Python
  2.2 and 2.2.1, its signature is \ctype{destructor}:

\begin{verbatim}
void tp_free(PyObject *)
\end{verbatim}

  In Python 2.3 and beyond, its signature is \ctype{freefunc}:

\begin{verbatim}
void tp_free(void *)
\end{verbatim}

  The only initializer that is compatible with both versions is
  \code{_PyObject_Del}, whose definition has suitably adapted in
  Python 2.3.

  This field is inherited by static subtypes, but not by dynamic
  subtypes (subtypes created by a class statement); in the latter,
  this field is set to a deallocator suitable to match
  \cfunction{PyType_GenericAlloc()} and the value of the
  \constant{Py_TPFLAGS_HAVE_GC} flag bit.
\end{cmemberdesc}

\begin{cmemberdesc}{PyTypeObject}{inquiry}{tp_is_gc}
  An optional pointer to a function called by the garbage collector.

  The garbage collector needs to know whether a particular object is
  collectible or not.  Normally, it is sufficient to look at the
  object's type's \member{tp_flags} field, and check the
  \constant{Py_TPFLAGS_HAVE_GC} flag bit.  But some types have a
  mixture of statically and dynamically allocated instances, and the
  statically allocated instances are not collectible.  Such types
  should define this function; it should return \code{1} for a
  collectible instance, and \code{0} for a non-collectible instance.
  The signature is

\begin{verbatim}
int tp_is_gc(PyObject *self)
\end{verbatim}

  (The only example of this are types themselves.  The metatype,
  \cdata{PyType_Type}, defines this function to distinguish between
  statically and dynamically allocated types.)

  This field is inherited by subtypes.  (VERSION NOTE: in Python
  2.2, it was not inherited.  It is inherited in 2.2.1 and later
  versions.)
\end{cmemberdesc}

\begin{cmemberdesc}{PyTypeObject}{PyObject*}{tp_bases}
  Tuple of base types.

  This is set for types created by a class statement.  It should be
  \NULL{} for statically defined types.

  This field is not inherited.
\end{cmemberdesc}

\begin{cmemberdesc}{PyTypeObject}{PyObject*}{tp_mro}
  Tuple containing the expanded set of base types, starting with the
  type itself and ending with \class{object}, in Method Resolution
  Order.

  This field is not inherited; it is calculated fresh by
  \cfunction{PyType_Ready()}.
\end{cmemberdesc}

\begin{cmemberdesc}{PyTypeObject}{PyObject*}{tp_cache}
  Unused.  Not inherited.  Internal use only.
\end{cmemberdesc}

\begin{cmemberdesc}{PyTypeObject}{PyObject*}{tp_subclasses}
  List of weak references to subclasses.  Not inherited.  Internal
  use only.
\end{cmemberdesc}

\begin{cmemberdesc}{PyTypeObject}{PyObject*}{tp_weaklist}
  Weak reference list head, for weak references to this type
  object.  Not inherited.  Internal use only.
\end{cmemberdesc}

The remaining fields are only defined if the feature test macro
\constant{COUNT_ALLOCS} is defined, and are for internal use only.
They are documented here for completeness.  None of these fields are
inherited by subtypes.

\begin{cmemberdesc}{PyTypeObject}{Py_ssize_t}{tp_allocs}
  Number of allocations.
\end{cmemberdesc}

\begin{cmemberdesc}{PyTypeObject}{Py_ssize_t}{tp_frees}
  Number of frees.
\end{cmemberdesc}

\begin{cmemberdesc}{PyTypeObject}{Py_ssize_t}{tp_maxalloc}
  Maximum simultaneously allocated objects.
\end{cmemberdesc}

\begin{cmemberdesc}{PyTypeObject}{PyTypeObject*}{tp_next}
  Pointer to the next type object with a non-zero \member{tp_allocs}
  field.
\end{cmemberdesc}

Also, note that, in a garbage collected Python, tp_dealloc may be
called from any Python thread, not just the thread which created the
object (if the object becomes part of a refcount cycle, that cycle
might be collected by a garbage collection on any thread).  This is
not a problem for Python API calls, since the thread on which
tp_dealloc is called will own the Global Interpreter Lock (GIL).
However, if the object being destroyed in turn destroys objects from
some other C or \Cpp{} library, care should be taken to ensure that
destroying those objects on the thread which called tp_dealloc will
not violate any assumptions of the library.

\section{Mapping Object Structures \label{mapping-structs}}

\begin{ctypedesc}{PyMappingMethods}
  Structure used to hold pointers to the functions used to implement
  the mapping protocol for an extension type.
\end{ctypedesc}


\section{Number Object Structures \label{number-structs}}

\begin{ctypedesc}{PyNumberMethods}
  Structure used to hold pointers to the functions an extension type
  uses to implement the number protocol.
\end{ctypedesc}


\section{Sequence Object Structures \label{sequence-structs}}

\begin{ctypedesc}{PySequenceMethods}
  Structure used to hold pointers to the functions which an object
  uses to implement the sequence protocol.
\end{ctypedesc}


\section{Buffer Object Structures \label{buffer-structs}}
\sectionauthor{Greg J. Stein}{greg@lyra.org}

The buffer interface exports a model where an object can expose its
internal data as a set of chunks of data, where each chunk is
specified as a pointer/length pair.  These chunks are called
\dfn{segments} and are presumed to be non-contiguous in memory.

If an object does not export the buffer interface, then its
\member{tp_as_buffer} member in the \ctype{PyTypeObject} structure
should be \NULL.  Otherwise, the \member{tp_as_buffer} will point to
a \ctype{PyBufferProcs} structure.

\note{It is very important that your \ctype{PyTypeObject} structure
uses \constant{Py_TPFLAGS_DEFAULT} for the value of the
\member{tp_flags} member rather than \code{0}.  This tells the Python
runtime that your \ctype{PyBufferProcs} structure contains the
\member{bf_getcharbuffer} slot. Older versions of Python did not have
this member, so a new Python interpreter using an old extension needs
to be able to test for its presence before using it.}

\begin{ctypedesc}{PyBufferProcs}
  Structure used to hold the function pointers which define an
  implementation of the buffer protocol.

  The first slot is \member{bf_getreadbuffer}, of type
  \ctype{getreadbufferproc}.  If this slot is \NULL, then the object
  does not support reading from the internal data.  This is
  non-sensical, so implementors should fill this in, but callers
  should test that the slot contains a non-\NULL{} value.

  The next slot is \member{bf_getwritebuffer} having type
  \ctype{getwritebufferproc}.  This slot may be \NULL{} if the object
  does not allow writing into its returned buffers.

  The third slot is \member{bf_getsegcount}, with type
  \ctype{getsegcountproc}.  This slot must not be \NULL{} and is used
  to inform the caller how many segments the object contains.  Simple
  objects such as \ctype{PyString_Type} and \ctype{PyBuffer_Type}
  objects contain a single segment.

  The last slot is \member{bf_getcharbuffer}, of type
  \ctype{getcharbufferproc}.  This slot will only be present if the
  \constant{Py_TPFLAGS_HAVE_GETCHARBUFFER} flag is present in the
  \member{tp_flags} field of the object's \ctype{PyTypeObject}.
  Before using this slot, the caller should test whether it is present
  by using the
  \cfunction{PyType_HasFeature()}\ttindex{PyType_HasFeature()}
  function.  If the flag is present, \member{bf_getcharbuffer} may be
  \NULL,
  indicating that the object's
  contents cannot be used as \emph{8-bit characters}.
  The slot function may also raise an error if the object's contents
  cannot be interpreted as 8-bit characters.  For example, if the
  object is an array which is configured to hold floating point
  values, an exception may be raised if a caller attempts to use
  \member{bf_getcharbuffer} to fetch a sequence of 8-bit characters.
  This notion of exporting the internal buffers as ``text'' is used to
  distinguish between objects that are binary in nature, and those
  which have character-based content.

  \note{The current policy seems to state that these characters
  may be multi-byte characters. This implies that a buffer size of
  \var{N} does not mean there are \var{N} characters present.}
\end{ctypedesc}

\begin{datadesc}{Py_TPFLAGS_HAVE_GETCHARBUFFER}
  Flag bit set in the type structure to indicate that the
  \member{bf_getcharbuffer} slot is known.  This being set does not
  indicate that the object supports the buffer interface or that the
  \member{bf_getcharbuffer} slot is non-\NULL.
\end{datadesc}

\begin{ctypedesc}[getreadbufferproc]{Py_ssize_t (*readbufferproc)
                            (PyObject *self, Py_ssize_t segment, void **ptrptr)}
  Return a pointer to a readable segment of the buffer in
  \code{*\var{ptrptr}}.  This function
  is allowed to raise an exception, in which case it must return
  \code{-1}.  The \var{segment} which is specified must be zero or
  positive, and strictly less than the number of segments returned by
  the \member{bf_getsegcount} slot function.  On success, it returns
  the length of the segment, and sets \code{*\var{ptrptr}} to a
  pointer to that memory.
\end{ctypedesc}

\begin{ctypedesc}[getwritebufferproc]{Py_ssize_t (*writebufferproc)
                            (PyObject *self, Py_ssize_t segment, void **ptrptr)}
  Return a pointer to a writable memory buffer in
  \code{*\var{ptrptr}}, and the length of that segment as the function
  return value.  The memory buffer must correspond to buffer segment
  \var{segment}.  Must return \code{-1} and set an exception on
  error.  \exception{TypeError} should be raised if the object only
  supports read-only buffers, and \exception{SystemError} should be
  raised when \var{segment} specifies a segment that doesn't exist.
% Why doesn't it raise ValueError for this one?
% GJS: because you shouldn't be calling it with an invalid
%      segment. That indicates a blatant programming error in the C
%      code.
\end{ctypedesc}

\begin{ctypedesc}[getsegcountproc]{Py_ssize_t (*segcountproc)
                            (PyObject *self, Py_ssize_t *lenp)}
  Return the number of memory segments which comprise the buffer.  If
  \var{lenp} is not \NULL, the implementation must report the sum of
  the sizes (in bytes) of all segments in \code{*\var{lenp}}.
  The function cannot fail.
\end{ctypedesc}

\begin{ctypedesc}[getcharbufferproc]{Py_ssize_t (*charbufferproc)
                            (PyObject *self, Py_ssize_t segment, const char **ptrptr)}
  Return the size of the segment \var{segment} that \var{ptrptr} 
  is set to.  \code{*\var{ptrptr}} is set to the memory buffer.
  Returns \code{-1} on error.
\end{ctypedesc}


\section{Supporting the Iterator Protocol
         \label{supporting-iteration}}


\section{Supporting Cyclic Garbage Collection
         \label{supporting-cycle-detection}}

Python's support for detecting and collecting garbage which involves
circular references requires support from object types which are
``containers'' for other objects which may also be containers.  Types
which do not store references to other objects, or which only store
references to atomic types (such as numbers or strings), do not need
to provide any explicit support for garbage collection.

An example showing the use of these interfaces can be found in
``\ulink{Supporting the Cycle
Collector}{../ext/example-cycle-support.html}'' in
\citetitle[../ext/ext.html]{Extending and Embedding the Python
Interpreter}.

To create a container type, the \member{tp_flags} field of the type
object must include the \constant{Py_TPFLAGS_HAVE_GC} and provide an
implementation of the \member{tp_traverse} handler.  If instances of the
type are mutable, a \member{tp_clear} implementation must also be
provided.

\begin{datadesc}{Py_TPFLAGS_HAVE_GC}
  Objects with a type with this flag set must conform with the rules
  documented here.  For convenience these objects will be referred to
  as container objects.
\end{datadesc}

Constructors for container types must conform to two rules:

\begin{enumerate}
\item  The memory for the object must be allocated using
       \cfunction{PyObject_GC_New()} or \cfunction{PyObject_GC_VarNew()}.

\item  Once all the fields which may contain references to other
       containers are initialized, it must call
       \cfunction{PyObject_GC_Track()}.
\end{enumerate}

\begin{cfuncdesc}{\var{TYPE}*}{PyObject_GC_New}{TYPE, PyTypeObject *type}
  Analogous to \cfunction{PyObject_New()} but for container objects with
  the \constant{Py_TPFLAGS_HAVE_GC} flag set.
\end{cfuncdesc}

\begin{cfuncdesc}{\var{TYPE}*}{PyObject_GC_NewVar}{TYPE, PyTypeObject *type,
                                                   Py_ssize_t size}
  Analogous to \cfunction{PyObject_NewVar()} but for container objects
  with the \constant{Py_TPFLAGS_HAVE_GC} flag set.
\end{cfuncdesc}

\begin{cfuncdesc}{PyVarObject *}{PyObject_GC_Resize}{PyVarObject *op, Py_ssize_t}
  Resize an object allocated by \cfunction{PyObject_NewVar()}.  Returns
  the resized object or \NULL{} on failure.
\end{cfuncdesc}

\begin{cfuncdesc}{void}{PyObject_GC_Track}{PyObject *op}
  Adds the object \var{op} to the set of container objects tracked by
  the collector.  The collector can run at unexpected times so objects
  must be valid while being tracked.  This should be called once all
  the fields followed by the \member{tp_traverse} handler become valid,
  usually near the end of the constructor.
\end{cfuncdesc}

\begin{cfuncdesc}{void}{_PyObject_GC_TRACK}{PyObject *op}
  A macro version of \cfunction{PyObject_GC_Track()}.  It should not be
  used for extension modules.
\end{cfuncdesc}

Similarly, the deallocator for the object must conform to a similar
pair of rules:

\begin{enumerate}
\item  Before fields which refer to other containers are invalidated,
       \cfunction{PyObject_GC_UnTrack()} must be called.

\item  The object's memory must be deallocated using
       \cfunction{PyObject_GC_Del()}.
\end{enumerate}

\begin{cfuncdesc}{void}{PyObject_GC_Del}{void *op}
  Releases memory allocated to an object using
  \cfunction{PyObject_GC_New()} or \cfunction{PyObject_GC_NewVar()}.
\end{cfuncdesc}

\begin{cfuncdesc}{void}{PyObject_GC_UnTrack}{void *op}
  Remove the object \var{op} from the set of container objects tracked
  by the collector.  Note that \cfunction{PyObject_GC_Track()} can be
  called again on this object to add it back to the set of tracked
  objects.  The deallocator (\member{tp_dealloc} handler) should call
  this for the object before any of the fields used by the
  \member{tp_traverse} handler become invalid.
\end{cfuncdesc}

\begin{cfuncdesc}{void}{_PyObject_GC_UNTRACK}{PyObject *op}
  A macro version of \cfunction{PyObject_GC_UnTrack()}.  It should not be
  used for extension modules.
\end{cfuncdesc}

The \member{tp_traverse} handler accepts a function parameter of this
type:

\begin{ctypedesc}[visitproc]{int (*visitproc)(PyObject *object, void *arg)}
  Type of the visitor function passed to the \member{tp_traverse}
  handler.  The function should be called with an object to traverse
  as \var{object} and the third parameter to the \member{tp_traverse}
  handler as \var{arg}.  The Python core uses several visitor functions
  to implement cyclic garbage detection; it's not expected that users will
  need to write their own visitor functions.
\end{ctypedesc}

The \member{tp_traverse} handler must have the following type:

\begin{ctypedesc}[traverseproc]{int (*traverseproc)(PyObject *self,
                                visitproc visit, void *arg)}
  Traversal function for a container object.  Implementations must
  call the \var{visit} function for each object directly contained by
  \var{self}, with the parameters to \var{visit} being the contained
  object and the \var{arg} value passed to the handler.  The \var{visit}
  function must not be called with a \NULL{} object argument.  If
  \var{visit} returns a non-zero value
  that value should be returned immediately.
\end{ctypedesc}

To simplify writing \member{tp_traverse} handlers, a
\cfunction{Py_VISIT()} macro is provided.  In order to use this macro,
the \member{tp_traverse} implementation must name its arguments
exactly \var{visit} and \var{arg}:

\begin{cfuncdesc}{void}{Py_VISIT}{PyObject *o}
  Call the \var{visit} callback, with arguments \var{o} and \var{arg}.
  If \var{visit} returns a non-zero value, then return it.  Using this
  macro, \member{tp_traverse} handlers look like:

\begin{verbatim}
static int
my_traverse(Noddy *self, visitproc visit, void *arg)
{
    Py_VISIT(self->foo);
    Py_VISIT(self->bar);
    return 0;
}
\end{verbatim}

\versionadded{2.4}
\end{cfuncdesc}


The \member{tp_clear} handler must be of the \ctype{inquiry} type, or
\NULL{} if the object is immutable.

\begin{ctypedesc}[inquiry]{int (*inquiry)(PyObject *self)}
  Drop references that may have created reference cycles.  Immutable
  objects do not have to define this method since they can never
  directly create reference cycles.  Note that the object must still
  be valid after calling this method (don't just call
  \cfunction{Py_DECREF()} on a reference).  The collector will call
  this method if it detects that this object is involved in a
  reference cycle.
\end{ctypedesc}

\chapter{distutils �ˤ�� C ����� \Cpp{} ��ĥ�⥸�塼��Υӥ��
     \label{building}}

\sectionauthor{Martin v. L\"owis}{martin@v.loewis.de}

Python 1.4 �ˤʤäƤ��顢ưŪ�˥�󥯤����褦�ʳ�ĥ�⥸�塼��
��ӥ�ɤ��뤿��Υᥤ���ե�������������褦�ʡ��ü�ʥᥤ���ե�����
��\UNIX{} �������󶡤���褦�ˤʤ�ޤ�����Python 2.0 �����
���ε��� (������ Makefile.pre.in ����� Setup �ե�����δط��ե�����)
�ϥ��ݡ��Ȥ���ʤ��ʤ�ޤ��������󥿥ץ꥿���ΤΥ������ޥ�����
�ۤȤ�ɻȤ�줺�� distutils �dz�ĥ�⥸�塼���ӥ�ɤǤ���
�褦�ˤʤä�����Ǥ���

distutils ��Ȥä���ĥ�⥸�塼��Υӥ�ɤˤϡ��ӥ�ɤ�Ԥ��׻������
distutils �򥤥󥹥ȡ��뤷�Ƥ��뤳�Ȥ�ɬ�פǤ���
Python 2.x �ˤ� distutils �����äƤ��ꡢ Python 1.5 �Ѥˤϸ��̤�
�ѥå�����������ޤ���distutils �ϥХ��ʥ�ѥå������κ�����
���ݡ��Ȥ��Ƥ���Τǡ��桼������ĥ�⥸�塼��򥤥󥹥ȡ��뤹��
�ݤˡ�ɬ�����⥳��ѥ��餬ɬ�פȤ����櫓�ǤϤ���ޤ���

distutils �١����Υѥå������ˤϡ���ư������ץ� (driver script)
�Ȥʤ� \file{setup.py} �����äƤ��ޤ���
\file{setup.py} �����̤� Python �ץ������ե�����ǡ��ۤȤ�ɤξ��
�ʲ��Τ褦�ʸ������ˤʤäƤ��ޤ�:

\begin{verbatim}
from distutils.core import setup, Extension

module1 = Extension('demo',
                    sources = ['demo.c'])

setup (name = 'PackageName',
       version = '1.0',
       description = 'This is a demo package',
       ext_modules = [module1])

\end{verbatim}

���� \file{setup.py} �ȥե����� \file{demo.c} ������Ȥ����ʲ���
���ޥ��

\begin{verbatim}
python setup.py build 
\end{verbatim}

��¹Ԥ���ȡ�\file{demo.c} �򥳥�ѥ��뤷�ơ�\samp{demo} �Ȥ���̾����
��ĥ�⥸�塼���\file{build} �ǥ��쥯�ȥ�����������ޤ���
�����ƥ�ˤ�äƤϥ⥸�塼��ե������ \file{build/lib.system}
���֥ǥ��쥯�ȥ���������졢\file{demo.so} �� \file{demo.pyd}
�Ȥ��ä�̾���ˤʤ뤳�Ȥ�����ޤ���

\file{setup.py} ��Ǥϡ����ޥ�ɤμ¹ԤϤ��٤� \samp{setup} �ؿ���
�ƤӽФ��ƹԤ��ޤ������δؿ��ϲ��ѸĤΥ�����ɰ�����Ȥ�ޤ���
��ǤϤ��ΰ�����ȤäƤ���ˤ����ޤ���
��äȶ���Ū�ˤ����ȡ������Ǥϥѥå�������ӥ�ɤ��뤿���
�᥿����ȡ��ѥå����������Ƥ���ꤷ�Ƥ��ޤ���
�̾�ѥå������ˤ� Python �������⥸�塼���ɥ�����ȡ�
���֥ѥå��������Ȥ��ä��̤Υե����������ޤ���
distutils �ε�ǽ�˴ؤ���ܺ٤ϡ�
\citetitle[../dist/dist.html]{Python �⥸�塼�������} �˽񤫤�Ƥ���
distutils �Υɥ�����Ȥ򻲾Ȥ��Ƥ�������; 
������Ǥϡ���ĥ�⥸�塼��Υӥ�ɤˤĤ��ƤΤ��������ޤ���

��ư������ץȤ���褯�������뤿��ˡ�����Ǥ��ΰ�����
\function{setup} ������Ƥ������Ȥ��褯����ޤ���
�����Ǥϡ�\function{setup} �� \samp{ext_modules} ��
��ĥ�⥸�塼��Υꥹ�Ȥǡ��ꥹ�Ȥγơ������Ǥ� \class{Extension} 
���饹�Υ��󥹥��󥹤ˤʤäƤ��ޤ���
�����Ǥϡ�\samp{demo} �Ȥ���̾�γ�ĥ�⥸�塼���������Ƥ��ơ�ñ���
�������ե�����\file{demo.c} �򥳥�ѥ��뤷�ƥӥ�ɤ���褦������Ƥ��ޤ���

¿���ξ�硢��ĥ�⥸�塼��Υӥ�ɤϤ�ä�ʣ���ˤʤ�ޤ���
�Ȥ����Τϡ��ץ�ץ����å������饤�֥����ɲû��꤬ɬ�פ�
�ʤ뤳�Ȥ����뤫��Ǥ����㤨�аʲ��Υե����뤬���μ���Ǥ���

\begin{verbatim}
from distutils.core import setup, Extension

module1 = Extension('demo',
                    define_macros = [('MAJOR_VERSION', '1'),
                                     ('MINOR_VERSION', '0')],
                    include_dirs = ['/usr/local/include'],
                    libraries = ['tcl83'],
                    library_dirs = ['/usr/local/lib'],
                    sources = ['demo.c'])

setup (name = 'PackageName',
       version = '1.0',
       description = 'This is a demo package',
       author = 'Martin v. Loewis',
       author_email = 'martin@v.loewis.de',
       url = 'http://www.python.org/doc/current/ext/building.html',
       long_description = '''
This is really just a demo package.
''',
       ext_modules = [module1])

\end{verbatim}

������Ǥϡ�\function{setup} ���ɲäΥ᥿����ȶ��˸ƤӽФ���ޤ���
���ۥѥå��������ۤ���ݤˤϡ��᥿������ɲä��侩����Ƥ��ޤ���
��ĥ�⥸�塼�뼫�ΤˤĤ��Ƥϡ��ץ�ץ����å���������󥯥롼�ɥե������
�ǥ��쥯�ȥꡢ�饤�֥��Υǥ��쥯�ȥꡢ�饤�֥��Ȥ��ä����꤬
����ޤ���
distutils �Ϥ��ξ���򥳥�ѥ���˱����ưۤʤ������ǰ��Ϥ��ޤ���
�㤨�С�\UNIX{} �Ǥϡ��������ϰʲ��Τ褦�ʥ���ѥ��륳�ޥ�ɤ�
�ʤ뤫�⤷��ޤ���:

\begin{verbatim}
gcc -DNDEBUG -g -O3 -Wall -Wstrict-prototypes -fPIC -DMAJOR_VERSION=1 -DMINOR_VERSION=0 -I/usr/local/include -I/usr/local/include/python2.2 -c demo.c -o build/temp.linux-i686-2.2/demo.o

gcc -shared build/temp.linux-i686-2.2/demo.o -L/usr/local/lib -ltcl83 -o build/lib.linux-i686-2.2/demo.so
\end{verbatim}

�����Υ��ޥ�ɥ饤��ϼ±���Ū�ǽ񤫤줿��ΤǤ�; distutils ��
�桼���� distutils �����������ޥ�ɤ�¹Ԥ���ȿ��Ѥ��Ƥ���������

\section{��ĥ�⥸�塼�������
     \label{distributing}}

��ĥ�⥸�塼��򤦤ޤ��ӥ�ɤǤ����顢���̤�λȤ���������ޤ���

����ɥ桼�������̥⥸�塼��򥤥󥹥ȡ��뤷�褦�ȹͤ��ޤ�;
����ˤ�

\begin{verbatim}
python setup.py install
\end{verbatim}
��¹Ԥ��ޤ���

�⥸�塼����ƥʤϥ������ѥå�������������ޤ�; ����ˤ�

\begin{verbatim}
python setup.py sdist
\end{verbatim}
��¹Ԥ��ޤ���

���ˤ�äƤϡ�����������ʪ���ɲäΥե������ޤ��ɬ�פ�
����ޤ�; ����ˤ�\file{MANIFEST.in} �ե������Ȥ��ޤ�;
�ܤ����� distutils �Υɥ�����Ȥ򻲾Ȥ��Ƥ���������

����������������ʪ�򤦤ޤ����ۤǤ����顢���ƥʤϥХ��ʥ�����ʪ��
�����Ǥ��ޤ����ץ�åȥե�����˱����ơ��ʲ��Υ��ޥ�ɤΤ����줫��
�Ȥ��ޤ���

\begin{verbatim}
python setup.py bdist_wininst
python setup.py bdist_rpm
python setup.py bdist_dumb
\end{verbatim}


\chapter{MS Windows Specific Services}


This chapter describes modules that are only available on MS Windows
platforms.


\localmoduletable

\chapter{¾�Υ��ץꥱ�������ؤ� Python ��������
     \label{embedding}}

���ϤǤϡ� Python ���ĥ������ˡ�����ʤ�� C �ؿ��Υ饤�֥���
Python �˷�ӤĤ��Ƶ�ǽ���ĥ������ˡ�ˤĤ��ƽҤ٤ޤ�����
Ʊ���褦�ʤ��Ȥ��̤���ˡ�Ǥ�¹ԤǤ��ޤ�: ����ϡ���ʬ�� C/\Cpp{}
���ץꥱ�������� Python ��������ǵ�ǽ�򶯲����롢�Ȥ�����ΤǤ���
�����ߤ�Ԥ����Ȥǡ����ץꥱ�������β��餫�ε�ǽ�� C �� \Cpp
������� Python �Ǽ����Ǥ���褦�ˤʤ�ޤ���
�����ߤ�¿�������Ӥ����ѤǤ��ޤ�; �桼���� Python �ǥ�����ץȤ�񤭡�
���ץꥱ��������ʬ���ߤ˻�Ω�Ƥ���褦�ˤ��롢�Ȥ����Τ�
���ΰ���Ǥ����ץ�����ޤ�������ε�ǽ�� Python �Ǥ��ڤ˽񤱤����
��ʬ���ȤΤ���������ߤ�Ԥ����Ȥ�Ǥ��ޤ���

Python �������ߤ� Python �γ�ĥ�Ȼ��Ƥ��ޤ���������Ʊ���Ȥ���
�櫓�ǤϤ���ޤ��󡣤��ΰ㤤�ϡ�Python ���ĥ�������ˤ�
���ץꥱ�������Υᥤ��ץ������ϰ����Ȥ��� Python ���󥿥ץ꥿
�Ǥ�������� Python ���Ȥ߹��ߤ�����ˤϡ��ᥤ��ץ������ˤ�
Python ���ط����ʤ� --- ��������ˡ����ץꥱ�������Τ������ʬ
������ Python ���󥿥ץ꥿��ƤӽФ��Ʋ��餫�� Python �����ɤ�
�¹Ԥ����� --- ���⤷��ʤ����Ȥ������ȤǤ���

���äơ� Python �������ߤ�Ԥ���硢����Υᥤ��ץ�������
�󶡤��ʤ���Фʤ�ޤ��󡣥ᥤ��ץ�����ब���ʤ���Фʤ�ʤ�����
�ΰ�Ĥˡ� Python ���󥿥ץ꥿�ν����������ޤ����Ȥˤ������ʤ��Ȥ�
�ؿ� \cfunction{Py_Initialize()} 
(Mac OS �ʤ�\cfunction{PyMac_Initialize()}) ��ƤӽФ��ͤФʤ�ޤ���
���ץ����Ȥ��ơ�Python ¦�˥��ޥ�ɥ饤��������Ϥ�����˴ؿ��ƤӽФ�
��Ԥ��ޤ������θ塢���ץꥱ�������Τɤ��Ǥ⥤�󥿥ץ꥿��
�ƤӽФ���褦�ˤʤ�ޤ���

���󥿥ץ꥿��ƤӽФ��ˤϡ��ۤʤ뤤���Ĥ�����ˡ������ޤ�:
Python ʸ�����ä�ʸ�����\cfunction{PyRun_SimpleString()} ���Ϥ���
stdio �ե�����ݥ��󥿤ȥե�����̾ (����ϥ��顼��å��������
�����ɤ��̤��뤿������Τ�ΤǤ�) ��
\cfunction{PyRun_SimpleFile()} ���Ϥ����Ȥ��ä����Ǥ���
����ޤǤγƾϤ������������������ƤӽФ��ơ�Python ���֥������Ȥ�
���ۤ�������Ѥ������Ǥ��ޤ���

Python �������ߤ�ԤäƤ����ñ�ʥǥ�ϡ�����������ʪ��
\file{Demo/embed/} �ǥ��쥯�ȥ�ˤ���ޤ���


\begin{seealso}
  \seetitle[../api/api.html]{Python/C API ��ե���󥹥ޥ˥奢��}{
Python  C ���󥿥ե������ξܺ٤Ϥ��Υޥ˥奢��˽񤫤�Ƥ��ޤ���
ɬ�פʾ��������ʬ�Ϥ����ˤ���Ϥ��Ǥ���}
\end{seealso}


\section{�����������
         \label{high-level-embedding}}

Python �������ߤκǤ��ñ�ʷ����ϡ�Ķ���।�󥿥ե�������
���ѤǤ���
���Υ��󥿥ե������ϡ����ץꥱ�������Ȥ���ꤹ��ɬ�פ��ʤ�
Python ������ץȤ�¹Ԥ��뤿��Τ�ΤǤ���
�㤨�Ф���ϡ���ĤΥե������Dz��餫������¸�����Τ�����
�Ǥ��ޤ���

\begin{verbatim}
#include <Python.h>

int
main(int argc, char *argv[])
{
  Py_Initialize();
  PyRun_SimpleString("from time import time,ctime\n"
                     "print 'Today is',ctime(time())\n");
  Py_Finalize();
  return 0;
}
\end{verbatim}

��Υ����ɤǤϡ��ޤ� Python ���󥿥ץ꥿��
\cfunction{Py_Initialize()} �ǵ�ư����³���ƥϡ��ɥ����ɤ��줿
Python ������ץȤ����դȻ��֤ν��Ϥ�¹Ԥ��ޤ���
���θ塢\cfunction{Py_Finalize()} �θƤӽФ��ǥ��󥿥ץ꥿��
��λ��, �ץ������ν�λ��³���ޤ���
�ºݤΥץ������Ǥϡ�Python ������ץȤ�¾�Υ������������餯
�ƥ����ȥ��ǥ����롼�����ե����롢�ǡ����١���������Ф�������
�ͤ��뤫�⤷��ޤ���Python �����ɤ�ե����뤫����Ф��ˤϡ�
\cfunction{PyRun_SimpleFile()} �ؿ���Ȥ��Τ��褤�Ǥ��礦��
���δؿ��ϥ������ݤ��ơ��ե���������Ƥ�����ɤ����֤�
�ʤ��Ƥ���ޤ���


\section{Ķ����������ߤ���Ƨ�߽Ф�: ����
         \label{lower-level-embedding}}

���।�󥿥ե������ϡ�����Ū�� Python �����ɤ򥢥ץꥱ������󤫤�
�¹ԤǤ���褦�ˤ��Ƥ���ޤ��������ץꥱ�������� Python �����ɤ�
�֤ǤΥǡ����Τ����ϡ�������˸��äƤ��Ѥ路����ΤǤ���
�ǡ����Τ����򤷤����ʤ顢�������Υ��󥿥ե������ƤӽФ���
���Ѥ��ʤ��ƤϤʤ�ޤ��󡣤��¿�� C �����ɤ�񤫤ͤФʤ�ʤ�
����ˡ��ۤܲ��Ǥ�Ǥ���褦�ˤʤ�ޤ���

Python �γ�ĥ�������ߤϡ���ݤ����㤨��Ʊ����ȤǤ���Ȥ�������
�����դ��ͤФʤ�ޤ��󡣤���ޤǤξϤǵ������Ƥ����ȥԥå���
�ۤȤ�ɤ������ߤǤ⤢�ƤϤޤ�ޤ�������򼨤�����ˡ�
Python ���� C �ؤγ�ĥ��Ԥ������ɤ��ºݤˤϲ��򤹤뤫�ͤ���
�ߤޤ��礦:

\begin{enumerate}
    \item �ǡ����ͤ� Python ���� C ���Ѵ����롣
    \item �Ѵ����줿�ͤ�Ȥä� C �롼����δؿ��ƤӽФ���Ԥ���
    \item �ƤӽФ�������줿�ǡ����� C ���� Python ���Ѵ����롣
\end{enumerate}

Python ����������ˤϡ����󥿥ե����������ɤ��Ԥ���Ȥϰʲ���
�褦�ˤʤ�ޤ�:

\begin{enumerate}
    \item �ǡ����ͤ� C ���� Python ���Ѵ����롣
    \item �Ѵ����줿�ͤ�Ȥä� Python ���󥿥ե������롼�����
�ؿ��ƤӽФ���Ԥ���
    \item �ƤӽФ�������줿�ǡ����� Python ���� C ���Ѵ����롣
\end{enumerate}

�츫����ʬ����褦�ˡ��ǡ����Ѵ��Υ��ƥåפϡ�����֤ǥǡ�����
ž�������������Ѥ�ä��Τ˹�碌��ñ�������ؤ��������Ǥ���
ͣ���������ϡ��ǡ����Ѵ��δ֤ˤ���롼����Ǥ���
��ĥ��Ԥ��ݤˤ� C �롼�����ƤӽФ��ޤ����������ߤ�
�ݤˤ� Python �롼�����ƤӽФ��ޤ���

���ξϤǤϡ�Python ���� C �ء������Ƥ��εդؤȥǡ������Ѵ�����
��ˡ�ˤĤ��Ƥϵ������ޤ��󡣤ޤ������������ȤλȤ����䥨�顼��
�������ˤĤ��Ƥ��Ǥ����򤷤Ƥ����ΤȲ��ꤷ�ޤ���
������¦�̤ˤĤ��Ƥϥ��󥿥ץ꥿�γ�ĥ�Ȳ����Ѥ��Ȥ�����
�ʤ��Τǡ�ɬ�פʾ���ˤĤ��Ƥϰ����ξϤ򻲾ȤǤ��ޤ���


\section{����������
         \label{pure-embedding}}

�ǽ���㼨����ץ������ϡ�Python ������ץ���δؿ���¹Ԥ���
����Τ�ΤǤ���Ķ���।�󥿥ե������˴ؤ�����ǵ󤲤�
���Ʊ�ͤˡ�Python ���󥿥ץ꥿�ϥ��ץꥱ��������ľ��
���Ȥ�Ϥ��ޤ��� (����������Ǥ��Ȥꤹ��褦�ѹ����ޤ�)��

Python ������ץ�����������Ƥ���ؿ���¹Ԥ��뤿���
�����ɤϰʲ��Τ褦�ˤʤ�ޤ�:

\verbatiminput{run-func.c}

���Υ����ɤ�\code{argv[1]} ��Ȥä� Python ������ץȤ�����ɤ���
\code{argv[2]} ��˻��ꤵ�줿̾���δؿ���ƤӽФ��ޤ���
�ؿ������������� \code{argv} �������¾���ͤˤʤ�ޤ���
���Υץ������򥳥�ѥ��뤷�ƥ�󥯤� (�Ǥ����¹Բ�ǽ������
\program{call} �ȸƤӤޤ��礦)���ʲ��Τ褦�� Python
������ץȤ�¹Ԥ��뤳�Ȥˤ��ޤ�:

\begin{verbatim}
def multiply(a,b):
    print "Will compute", a, "times", b
    c = 0
    for i in range(0, a):
        c = c + b
    return c
\end{verbatim}

�¹Է�̤ϰʲ��Τ褦�ˤʤ�Ϥ��Ǥ�:

\begin{verbatim}
$ call multiply multiply 3 2
Will compute 3 times 2
Result of call: 6
\end{verbatim} % $

�������٤ε�ǽ��¸�����ˤϥץ�����ब���������礭�����ޤ�����
�ۤȤ�ɤ� Python ���� C �ؤΥǡ����Ѵ��䥨�顼���Τ����
�����ɤǤ���Python �������ߤȤ�����������Ǥⶽ̣������ʬ��
�ʲ��Υ����ɡ�

\begin{verbatim}
    Py_Initialize();
    pName = PyString_FromString(argv[1]);
    /* pName �Υ��顼�����å��Ͼ�ά���Ƥ��� */
    pModule = PyImport_Import(pName);
\end{verbatim}

����Ϥޤ���ʬ�Ǥ���

���󥿥ץ꥿�ν�����塢������ץȤ�
\cfunction{PyImport_Import()} ��Ȥä��ɤ߹��ޤ�ޤ���
���Υ롼����� Python ʸ���������˼��ɬ�פ����ꡢ
�ǡ����Ѵ��롼����\cfunction{PyString_FromString()} ��
���ۤ��ޤ���

\begin{verbatim}
    pFunc = PyObject_GetAttrString(pModule, argv[2]);
    /* pFunc �Ͽ����ʻ��� */

    if (pFunc && PyCallable_Check(pFunc)) {
        ...
    }
    Py_XDECREF(pFunc);
\end{verbatim}

�ҤȤ��ӥ�����ץȤ��ɤ߹��ޤ��ȡ�
\cfunction{PyObject_GetAttrString()} ��Ȥä�ɬ�פ�̾�������
�Ǥ��ޤ���̾����������ץ����¸�ߤ��������������֥������Ȥ�
�ƤӽФ���ǽ���֥������ȤǤ���С����Υ��֥������Ȥ��ؿ��Ǥ����
�ͤ��ƺ����٤��ʤ��Ǥ��礦�������ǥץ����������Фɤ���˰�����
���ץ빽�ۤ˿ʤߤޤ������θ塢Python �ؿ���ʲ��Υ����ɤ�
�ƤӽФ��ޤ�:

\begin{verbatim}
    pValue = PyObject_CallObject(pFunc, pArgs);
\end{verbatim}

�ؿ����������᤹�ݡ�\code{pValue} �� \NULL{} �ˤʤ뤫��
�ؿ�������ͤؤλ��Ȥ����äƤ��ޤ���
�ͤ�Ĵ�٤���ˤ�˺�줺�˻��Ȥ�������Ƥ���������


\section{�����ޤ줿 Python �γ�ĥ
         \label{extending-with-embedding}}

�����ޤǤϡ������� Python ���󥿥ץ꥿�ϥ��ץꥱ����������Τ�
��ǽ�˥�������������ʤ�����ޤ���Ǥ�����
Python API ��Ȥ��ȡ������ߥ��󥿥ץ꥿���ĥ���뤳�Ȥ�
���ץꥱ����������ΤؤΥ����������ǽ�ˤ��ޤ���
�Ĥޤꡢ���ץꥱ���������󶡤���Ƥ���롼�����Ȥäơ�
�����ߥ��󥿥ץ꥿���ĥ����ΤǤ���
ʣ���ʤ��ȤΤ褦�˻פ��ޤ���������ۤɤҤɤ��櫓�ǤϤ���ޤ���
���������äơ����ץꥱ������� Python ���󥿥ץ꥿��ư������
�������Ȥ����ä�˺��ƤߤƤ����������������ꡢ���ץꥱ�������
���֥롼����ν��ޤ�ǡ������������̤� Python ��ĥ�⥸�塼���
�񤯤��Τ褦�ˡ�Python ����ƥ롼����˥��������Ǥ���褦�ˤ���
���롼(glue, ��) �����ɤ�񤯤ȹͤ��Ƥ����������㤨�аʲ��Τ褦�ˤǤ�:

\begin{verbatim}
static int numargs=0;

/* ���ץꥱ�������Υ��ޥ�ɥ饤������θĿ����֤� */
static PyObject*
emb_numargs(PyObject *self, PyObject *args)
{
    if(!PyArg_ParseTuple(args, ":numargs"))
        return NULL;
    return Py_BuildValue("i", numargs);
}

static PyMethodDef EmbMethods[] = {
    {"numargs", emb_numargs, METH_VARARGS,
     "Return the number of arguments received by the process."},
    {NULL, NULL, 0, NULL}
};
\end{verbatim}

��Υ����ɤ� \cfunction{main()} �ؿ��Τ�������������ޤ���
�ޤ����ʲ�����Ĥ�ʸ��\cfunction{Py_Initialize()} ��ľ��
���������ޤ�:

\begin{verbatim}
    numargs = argc;
    Py_InitModule("emb", EmbMethods);
\end{verbatim}

�������ĤιԤ�\code{numargs} �ѿ�����������
������ Python ���󥿥ץ꥿����\function{emb.numargs()} �ؿ���
���������Ǥ���褦�ˤ��ޤ��������γ�ĥ�⥸�塼��ؿ���Ȥ��ȡ�
Python ������ץȤ�

\begin{verbatim}
import emb
print "Number of arguments", emb.numargs()
\end{verbatim}

�Τ褦�ʤ��Ȥ��Ǥ��ޤ���

�ºݤΥ��ץꥱ�������Ǥϡ����������᥽�åɤǥ��ץꥱ����������
API �� Python �˸������뤳�Ȥˤʤ�ޤ���


%\section{For the future}
%
%You don't happen to have a nice library to get textual
%equivalents of numeric values do you :-) ?
%Callbacks here ? (I may be using information from that section
%?!)
%threads
%code examples do not really behave well if errors happen
% (what to watch out for)


\section{\Cpp �ˤ�� Python ��������
     \label{embeddingInCplusplus}}

\Cpp{} �ץ��������ˤ� Python ��������ޤ�; ��̩�˸����ȡ�
�ɤ���ä������फ�ϻȤäƤ���\Cpp{} �����Ϥξܺ٤˰�¸���ޤ�;
����Ū�ˤϡ��ᥤ��ץ�������\Cpp �ǽ񤭡�\Cpp{} ����ѥ���
��Ȥäƥץ������򥳥�ѥ��롦��󥯤���ɬ�פ�����Ǥ��礦��
Python ���Τ� \Cpp �ǥ���ѥ��뤷�ʤ���ɬ�פϤ���ޤ���


\section{��󥯤˴ؤ����׷�
         \label{link-reqs}}

Python �������Ȱ��ˤĤ��Ƥ���\program{configure} ������ץȤ�
ưŪ�˥�󥯤�����ĥ�⥸�塼�뤬ɬ�פȤ��륷��ܥ���������褦
�������� Python ��ӥ�ɤ��ޤ��������ε�ǽ��
Python �饤�֥�����Ū��������褦�ʥ��ץꥱ�������ˤ�
�Ѿ�����ޤ��󡣾��ʤ��Ȥ� \UNIX{} �ǤϤ����Ǥ���
����ϡ����ץꥱ���������Ū�ʼ¹Ի��饤�֥�� (\file{libpython.a})
�˥�󥯤���Ƥ��ơ����� (\file{.so} �ե�����Ȥ��Ƽ�������Ƥ���) 
ưŪ�����ɤ����褦�ʳ�ĥ�⥸�塼�������ɤ���ɬ�פ��������
����������Ǥ���

����ˤʤ�Τϡ���ĥ�⥸�塼�뤬�Ȥ����륨��ȥ�ݥ���Ȥ�
Python ��󥿥���������������Ƥ���Ȥ��������Ǥ���
�����ߤ�Ԥ����ץꥱ�������¦��������������ȥ�ݥ���Ȥ�
�����Ȥ�ʤ���硢��󥫤ˤ�äƤϥ���ȥ�ݥ���Ȥ�ǽ�Ū��
���������¹Բ�ǽ�����Υ���ܥ�ơ��֥���˴ޤ�ޤ���
����������硢��󥫤��ɲäΥ��ץ�����Ϳ���ơ������Υ���ܥ��
����ʤ��褦������ɬ�פ�����ޤ���

�ץ�åȥե����ऴ�Ȥ����������ץ��������ΤϤ��ʤ꺤��Ǥ���
�ȤϤ����������ʤ��Ȥˡ����ץ����� Python �Υӥ��������ˤ��Ǥ�
����ޤ������󥹥ȡ���Ѥߤ� Python ���󥿥ץ꥿���饪�ץ�����
���Ф��ˤϡ����å��󥿥ץ꥿��ư���ơ��ʲ��Τ褦��û��
���å�����¹Ԥ��ޤ�:

\begin{verbatim}
>>> import distutils.sysconfig
>>> distutils.sysconfig.get_config_var('LINKFORSHARED')
'-Xlinker -export-dynamic'
\end{verbatim}
\refstmodindex{distutils.sysconfig}

ɽ�����줿ʸ��������Ƥ����ӥ�ɻ��˻Ȥ��٤����ץ����Ǥ���
ʸ���󤬶��Ǥ���С��ä��ɲä��٤����ץ����Ϥ���ޤ���
\constant{LINKFORSHARED} ��������Ƥϡ� Python �Υȥåץ�٥�
\file{Makefile} ���Ʊ̾���ѿ����б����Ƥ��ޤ���






\appendix
\chapter{Reporting Bugs}
\label{reporting-bugs}

Python is a mature programming language which has established a
reputation for stability.  In order to maintain this reputation, the
developers would like to know of any deficiencies you find in Python
or its documentation.

Before submitting a report, you will be required to log into SourceForge;
this will make it possible for the developers to contact you
for additional information if needed.  It is not possible to submit a
bug report anonymously.

All bug reports should be submitted via the Python Bug Tracker on
SourceForge (\url{http://sourceforge.net/bugs/?group_id=5470}).  The
bug tracker offers a Web form which allows pertinent information to be
entered and submitted to the developers.

The first step in filing a report is to determine whether the problem
has already been reported.  The advantage in doing so, aside from
saving the developers time, is that you learn what has been done to
fix it; it may be that the problem has already been fixed for the next
release, or additional information is needed (in which case you are
welcome to provide it if you can!).  To do this, search the bug
database using the search box on the left side of the page.

If the problem you're reporting is not already in the bug tracker, go
back to the Python Bug Tracker
(\url{http://sourceforge.net/bugs/?group_id=5470}).  Select the
``Submit a Bug'' link at the top of the page to open the bug reporting
form.

The submission form has a number of fields.  The only fields that are
required are the ``Summary'' and ``Details'' fields.  For the summary,
enter a \emph{very} short description of the problem; less than ten
words is good.  In the Details field, describe the problem in detail,
including what you expected to happen and what did happen.  Be sure to
include the version of Python you used, whether any extension modules
were involved, and what hardware and software platform you were using
(including version information as appropriate).

The only other field that you may want to set is the ``Category''
field, which allows you to place the bug report into a broad category
(such as ``Documentation'' or ``Library'').

Each bug report will be assigned to a developer who will determine
what needs to be done to correct the problem.  You will
receive an update each time action is taken on the bug.


\begin{seealso}
  \seetitle[http://www-mice.cs.ucl.ac.uk/multimedia/software/documentation/ReportingBugs.html]{How
        to Report Bugs Effectively}{Article which goes into some
        detail about how to create a useful bug report.  This
        describes what kind of information is useful and why it is
        useful.}

  \seetitle[http://www.mozilla.org/quality/bug-writing-guidelines.html]{Bug
        Writing Guidelines}{Information about writing a good bug
        report.  Some of this is specific to the Mozilla project, but
        describes general good practices.}
\end{seealso}


\chapter{History and License}
\section{History of the software}

Python was created in the early 1990s by Guido van Rossum at Stichting
Mathematisch Centrum (CWI, see \url{http://www.cwi.nl/}) in the Netherlands
as a successor of a language called ABC.  Guido remains Python's
principal author, although it includes many contributions from others.

In 1995, Guido continued his work on Python at the Corporation for
National Research Initiatives (CNRI, see \url{http://www.cnri.reston.va.us/})
in Reston, Virginia where he released several versions of the
software.

In May 2000, Guido and the Python core development team moved to
BeOpen.com to form the BeOpen PythonLabs team.  In October of the same
year, the PythonLabs team moved to Digital Creations (now Zope
Corporation; see \url{http://www.zope.com/}).  In 2001, the Python
Software Foundation (PSF, see \url{http://www.python.org/psf/}) was
formed, a non-profit organization created specifically to own
Python-related Intellectual Property.  Zope Corporation is a
sponsoring member of the PSF.

All Python releases are Open Source (see
\url{http://www.opensource.org/} for the Open Source Definition).
Historically, most, but not all, Python releases have also been
GPL-compatible; the table below summarizes the various releases.

\begin{tablev}{c|c|c|c|c}{textrm}%
  {Release}{Derived from}{Year}{Owner}{GPL compatible?}
  \linev{0.9.0 thru 1.2}{n/a}{1991-1995}{CWI}{yes}
  \linev{1.3 thru 1.5.2}{1.2}{1995-1999}{CNRI}{yes}
  \linev{1.6}{1.5.2}{2000}{CNRI}{no}
  \linev{2.0}{1.6}{2000}{BeOpen.com}{no}
  \linev{1.6.1}{1.6}{2001}{CNRI}{no}
  \linev{2.1}{2.0+1.6.1}{2001}{PSF}{no}
  \linev{2.0.1}{2.0+1.6.1}{2001}{PSF}{yes}
  \linev{2.1.1}{2.1+2.0.1}{2001}{PSF}{yes}
  \linev{2.2}{2.1.1}{2001}{PSF}{yes}
  \linev{2.1.2}{2.1.1}{2002}{PSF}{yes}
  \linev{2.1.3}{2.1.2}{2002}{PSF}{yes}
  \linev{2.2.1}{2.2}{2002}{PSF}{yes}
  \linev{2.2.2}{2.2.1}{2002}{PSF}{yes}
  \linev{2.2.3}{2.2.2}{2002-2003}{PSF}{yes}
  \linev{2.3}{2.2.2}{2002-2003}{PSF}{yes}
  \linev{2.3.1}{2.3}{2002-2003}{PSF}{yes}
  \linev{2.3.2}{2.3.1}{2003}{PSF}{yes}
  \linev{2.3.3}{2.3.2}{2003}{PSF}{yes}
  \linev{2.3.4}{2.3.3}{2004}{PSF}{yes}
  \linev{2.3.5}{2.3.4}{2005}{PSF}{yes}
  \linev{2.4}{2.3}{2004}{PSF}{yes}
  \linev{2.4.1}{2.4}{2005}{PSF}{yes}
  \linev{2.4.2}{2.4.1}{2005}{PSF}{yes}
  \linev{2.4.3}{2.4.2}{2006}{PSF}{yes}
  \linev{2.5}{2.4}{2006}{PSF}{yes}
\end{tablev}

\note{GPL-compatible doesn't mean that we're distributing
Python under the GPL.  All Python licenses, unlike the GPL, let you
distribute a modified version without making your changes open source.
The GPL-compatible licenses make it possible to combine Python with
other software that is released under the GPL; the others don't.}

Thanks to the many outside volunteers who have worked under Guido's
direction to make these releases possible.


\section{Terms and conditions for accessing or otherwise using Python}

\centerline{\strong{PSF LICENSE AGREEMENT FOR PYTHON \version}}

\begin{enumerate}
\item
This LICENSE AGREEMENT is between the Python Software Foundation
(``PSF''), and the Individual or Organization (``Licensee'') accessing
and otherwise using Python \version{} software in source or binary
form and its associated documentation.

\item
Subject to the terms and conditions of this License Agreement, PSF
hereby grants Licensee a nonexclusive, royalty-free, world-wide
license to reproduce, analyze, test, perform and/or display publicly,
prepare derivative works, distribute, and otherwise use Python
\version{} alone or in any derivative version, provided, however, that
PSF's License Agreement and PSF's notice of copyright, i.e.,
``Copyright \copyright{} 2001-2006 Python Software Foundation; All
Rights Reserved'' are retained in Python \version{} alone or in any
derivative version prepared by Licensee.

\item
In the event Licensee prepares a derivative work that is based on
or incorporates Python \version{} or any part thereof, and wants to
make the derivative work available to others as provided herein, then
Licensee hereby agrees to include in any such work a brief summary of
the changes made to Python \version.

\item
PSF is making Python \version{} available to Licensee on an ``AS IS''
basis.  PSF MAKES NO REPRESENTATIONS OR WARRANTIES, EXPRESS OR
IMPLIED.  BY WAY OF EXAMPLE, BUT NOT LIMITATION, PSF MAKES NO AND
DISCLAIMS ANY REPRESENTATION OR WARRANTY OF MERCHANTABILITY OR FITNESS
FOR ANY PARTICULAR PURPOSE OR THAT THE USE OF PYTHON \version{} WILL
NOT INFRINGE ANY THIRD PARTY RIGHTS.

\item
PSF SHALL NOT BE LIABLE TO LICENSEE OR ANY OTHER USERS OF PYTHON
\version{} FOR ANY INCIDENTAL, SPECIAL, OR CONSEQUENTIAL DAMAGES OR
LOSS AS A RESULT OF MODIFYING, DISTRIBUTING, OR OTHERWISE USING PYTHON
\version, OR ANY DERIVATIVE THEREOF, EVEN IF ADVISED OF THE
POSSIBILITY THEREOF.

\item
This License Agreement will automatically terminate upon a material
breach of its terms and conditions.

\item
Nothing in this License Agreement shall be deemed to create any
relationship of agency, partnership, or joint venture between PSF and
Licensee.  This License Agreement does not grant permission to use PSF
trademarks or trade name in a trademark sense to endorse or promote
products or services of Licensee, or any third party.

\item
By copying, installing or otherwise using Python \version, Licensee
agrees to be bound by the terms and conditions of this License
Agreement.
\end{enumerate}


\centerline{\strong{BEOPEN.COM LICENSE AGREEMENT FOR PYTHON 2.0}}

\centerline{\strong{BEOPEN PYTHON OPEN SOURCE LICENSE AGREEMENT VERSION 1}}

\begin{enumerate}
\item
This LICENSE AGREEMENT is between BeOpen.com (``BeOpen''), having an
office at 160 Saratoga Avenue, Santa Clara, CA 95051, and the
Individual or Organization (``Licensee'') accessing and otherwise
using this software in source or binary form and its associated
documentation (``the Software'').

\item
Subject to the terms and conditions of this BeOpen Python License
Agreement, BeOpen hereby grants Licensee a non-exclusive,
royalty-free, world-wide license to reproduce, analyze, test, perform
and/or display publicly, prepare derivative works, distribute, and
otherwise use the Software alone or in any derivative version,
provided, however, that the BeOpen Python License is retained in the
Software, alone or in any derivative version prepared by Licensee.

\item
BeOpen is making the Software available to Licensee on an ``AS IS''
basis.  BEOPEN MAKES NO REPRESENTATIONS OR WARRANTIES, EXPRESS OR
IMPLIED.  BY WAY OF EXAMPLE, BUT NOT LIMITATION, BEOPEN MAKES NO AND
DISCLAIMS ANY REPRESENTATION OR WARRANTY OF MERCHANTABILITY OR FITNESS
FOR ANY PARTICULAR PURPOSE OR THAT THE USE OF THE SOFTWARE WILL NOT
INFRINGE ANY THIRD PARTY RIGHTS.

\item
BEOPEN SHALL NOT BE LIABLE TO LICENSEE OR ANY OTHER USERS OF THE
SOFTWARE FOR ANY INCIDENTAL, SPECIAL, OR CONSEQUENTIAL DAMAGES OR LOSS
AS A RESULT OF USING, MODIFYING OR DISTRIBUTING THE SOFTWARE, OR ANY
DERIVATIVE THEREOF, EVEN IF ADVISED OF THE POSSIBILITY THEREOF.

\item
This License Agreement will automatically terminate upon a material
breach of its terms and conditions.

\item
This License Agreement shall be governed by and interpreted in all
respects by the law of the State of California, excluding conflict of
law provisions.  Nothing in this License Agreement shall be deemed to
create any relationship of agency, partnership, or joint venture
between BeOpen and Licensee.  This License Agreement does not grant
permission to use BeOpen trademarks or trade names in a trademark
sense to endorse or promote products or services of Licensee, or any
third party.  As an exception, the ``BeOpen Python'' logos available
at http://www.pythonlabs.com/logos.html may be used according to the
permissions granted on that web page.

\item
By copying, installing or otherwise using the software, Licensee
agrees to be bound by the terms and conditions of this License
Agreement.
\end{enumerate}


\centerline{\strong{CNRI LICENSE AGREEMENT FOR PYTHON 1.6.1}}

\begin{enumerate}
\item
This LICENSE AGREEMENT is between the Corporation for National
Research Initiatives, having an office at 1895 Preston White Drive,
Reston, VA 20191 (``CNRI''), and the Individual or Organization
(``Licensee'') accessing and otherwise using Python 1.6.1 software in
source or binary form and its associated documentation.

\item
Subject to the terms and conditions of this License Agreement, CNRI
hereby grants Licensee a nonexclusive, royalty-free, world-wide
license to reproduce, analyze, test, perform and/or display publicly,
prepare derivative works, distribute, and otherwise use Python 1.6.1
alone or in any derivative version, provided, however, that CNRI's
License Agreement and CNRI's notice of copyright, i.e., ``Copyright
\copyright{} 1995-2001 Corporation for National Research Initiatives;
All Rights Reserved'' are retained in Python 1.6.1 alone or in any
derivative version prepared by Licensee.  Alternately, in lieu of
CNRI's License Agreement, Licensee may substitute the following text
(omitting the quotes): ``Python 1.6.1 is made available subject to the
terms and conditions in CNRI's License Agreement.  This Agreement
together with Python 1.6.1 may be located on the Internet using the
following unique, persistent identifier (known as a handle):
1895.22/1013.  This Agreement may also be obtained from a proxy server
on the Internet using the following URL:
\url{http://hdl.handle.net/1895.22/1013}.''

\item
In the event Licensee prepares a derivative work that is based on
or incorporates Python 1.6.1 or any part thereof, and wants to make
the derivative work available to others as provided herein, then
Licensee hereby agrees to include in any such work a brief summary of
the changes made to Python 1.6.1.

\item
CNRI is making Python 1.6.1 available to Licensee on an ``AS IS''
basis.  CNRI MAKES NO REPRESENTATIONS OR WARRANTIES, EXPRESS OR
IMPLIED.  BY WAY OF EXAMPLE, BUT NOT LIMITATION, CNRI MAKES NO AND
DISCLAIMS ANY REPRESENTATION OR WARRANTY OF MERCHANTABILITY OR FITNESS
FOR ANY PARTICULAR PURPOSE OR THAT THE USE OF PYTHON 1.6.1 WILL NOT
INFRINGE ANY THIRD PARTY RIGHTS.

\item
CNRI SHALL NOT BE LIABLE TO LICENSEE OR ANY OTHER USERS OF PYTHON
1.6.1 FOR ANY INCIDENTAL, SPECIAL, OR CONSEQUENTIAL DAMAGES OR LOSS AS
A RESULT OF MODIFYING, DISTRIBUTING, OR OTHERWISE USING PYTHON 1.6.1,
OR ANY DERIVATIVE THEREOF, EVEN IF ADVISED OF THE POSSIBILITY THEREOF.

\item
This License Agreement will automatically terminate upon a material
breach of its terms and conditions.

\item
This License Agreement shall be governed by the federal
intellectual property law of the United States, including without
limitation the federal copyright law, and, to the extent such
U.S. federal law does not apply, by the law of the Commonwealth of
Virginia, excluding Virginia's conflict of law provisions.
Notwithstanding the foregoing, with regard to derivative works based
on Python 1.6.1 that incorporate non-separable material that was
previously distributed under the GNU General Public License (GPL), the
law of the Commonwealth of Virginia shall govern this License
Agreement only as to issues arising under or with respect to
Paragraphs 4, 5, and 7 of this License Agreement.  Nothing in this
License Agreement shall be deemed to create any relationship of
agency, partnership, or joint venture between CNRI and Licensee.  This
License Agreement does not grant permission to use CNRI trademarks or
trade name in a trademark sense to endorse or promote products or
services of Licensee, or any third party.

\item
By clicking on the ``ACCEPT'' button where indicated, or by copying,
installing or otherwise using Python 1.6.1, Licensee agrees to be
bound by the terms and conditions of this License Agreement.
\end{enumerate}

\centerline{ACCEPT}



\centerline{\strong{CWI LICENSE AGREEMENT FOR PYTHON 0.9.0 THROUGH 1.2}}

Copyright \copyright{} 1991 - 1995, Stichting Mathematisch Centrum
Amsterdam, The Netherlands.  All rights reserved.

Permission to use, copy, modify, and distribute this software and its
documentation for any purpose and without fee is hereby granted,
provided that the above copyright notice appear in all copies and that
both that copyright notice and this permission notice appear in
supporting documentation, and that the name of Stichting Mathematisch
Centrum or CWI not be used in advertising or publicity pertaining to
distribution of the software without specific, written prior
permission.

STICHTING MATHEMATISCH CENTRUM DISCLAIMS ALL WARRANTIES WITH REGARD TO
THIS SOFTWARE, INCLUDING ALL IMPLIED WARRANTIES OF MERCHANTABILITY AND
FITNESS, IN NO EVENT SHALL STICHTING MATHEMATISCH CENTRUM BE LIABLE
FOR ANY SPECIAL, INDIRECT OR CONSEQUENTIAL DAMAGES OR ANY DAMAGES
WHATSOEVER RESULTING FROM LOSS OF USE, DATA OR PROFITS, WHETHER IN AN
ACTION OF CONTRACT, NEGLIGENCE OR OTHER TORTIOUS ACTION, ARISING OUT
OF OR IN CONNECTION WITH THE USE OR PERFORMANCE OF THIS SOFTWARE.


\section{Licenses and Acknowledgements for Incorporated Software}

This section is an incomplete, but growing list of licenses and
acknowledgements for third-party software incorporated in the
Python distribution.


\subsection{Mersenne Twister}

The \module{_random} module includes code based on a download from
\url{http://www.math.keio.ac.jp/~matumoto/MT2002/emt19937ar.html}.
The following are the verbatim comments from the original code:

\begin{verbatim}
A C-program for MT19937, with initialization improved 2002/1/26.
Coded by Takuji Nishimura and Makoto Matsumoto.

Before using, initialize the state by using init_genrand(seed)
or init_by_array(init_key, key_length).

Copyright (C) 1997 - 2002, Makoto Matsumoto and Takuji Nishimura,
All rights reserved.

Redistribution and use in source and binary forms, with or without
modification, are permitted provided that the following conditions
are met:

 1. Redistributions of source code must retain the above copyright
    notice, this list of conditions and the following disclaimer.

 2. Redistributions in binary form must reproduce the above copyright
    notice, this list of conditions and the following disclaimer in the
    documentation and/or other materials provided with the distribution.

 3. The names of its contributors may not be used to endorse or promote
    products derived from this software without specific prior written
    permission.

THIS SOFTWARE IS PROVIDED BY THE COPYRIGHT HOLDERS AND CONTRIBUTORS
"AS IS" AND ANY EXPRESS OR IMPLIED WARRANTIES, INCLUDING, BUT NOT
LIMITED TO, THE IMPLIED WARRANTIES OF MERCHANTABILITY AND FITNESS FOR
A PARTICULAR PURPOSE ARE DISCLAIMED.  IN NO EVENT SHALL THE COPYRIGHT OWNER OR
CONTRIBUTORS BE LIABLE FOR ANY DIRECT, INDIRECT, INCIDENTAL, SPECIAL,
EXEMPLARY, OR CONSEQUENTIAL DAMAGES (INCLUDING, BUT NOT LIMITED TO,
PROCUREMENT OF SUBSTITUTE GOODS OR SERVICES; LOSS OF USE, DATA, OR
PROFITS; OR BUSINESS INTERRUPTION) HOWEVER CAUSED AND ON ANY THEORY OF
LIABILITY, WHETHER IN CONTRACT, STRICT LIABILITY, OR TORT (INCLUDING
NEGLIGENCE OR OTHERWISE) ARISING IN ANY WAY OUT OF THE USE OF THIS
SOFTWARE, EVEN IF ADVISED OF THE POSSIBILITY OF SUCH DAMAGE.


Any feedback is very welcome.
http://www.math.keio.ac.jp/matumoto/emt.html
email: matumoto@math.keio.ac.jp
\end{verbatim}



\subsection{Sockets}

The \module{socket} module uses the functions, \function{getaddrinfo},
and \function{getnameinfo}, which are coded in separate source files
from the WIDE Project, \url{http://www.wide.ad.jp/about/index.html}.

\begin{verbatim}      
Copyright (C) 1995, 1996, 1997, and 1998 WIDE Project.
All rights reserved.
 
Redistribution and use in source and binary forms, with or without
modification, are permitted provided that the following conditions
are met:
1. Redistributions of source code must retain the above copyright
   notice, this list of conditions and the following disclaimer.
2. Redistributions in binary form must reproduce the above copyright
   notice, this list of conditions and the following disclaimer in the
   documentation and/or other materials provided with the distribution.
3. Neither the name of the project nor the names of its contributors
   may be used to endorse or promote products derived from this software
   without specific prior written permission.

THIS SOFTWARE IS PROVIDED BY THE PROJECT AND CONTRIBUTORS ``AS IS'' AND
GAI_ANY EXPRESS OR IMPLIED WARRANTIES, INCLUDING, BUT NOT LIMITED TO, THE
IMPLIED WARRANTIES OF MERCHANTABILITY AND FITNESS FOR A PARTICULAR PURPOSE
ARE DISCLAIMED.  IN NO EVENT SHALL THE PROJECT OR CONTRIBUTORS BE LIABLE
FOR GAI_ANY DIRECT, INDIRECT, INCIDENTAL, SPECIAL, EXEMPLARY, OR CONSEQUENTIAL
DAMAGES (INCLUDING, BUT NOT LIMITED TO, PROCUREMENT OF SUBSTITUTE GOODS
OR SERVICES; LOSS OF USE, DATA, OR PROFITS; OR BUSINESS INTERRUPTION)
HOWEVER CAUSED AND ON GAI_ANY THEORY OF LIABILITY, WHETHER IN CONTRACT, STRICT
LIABILITY, OR TORT (INCLUDING NEGLIGENCE OR OTHERWISE) ARISING IN GAI_ANY WAY
OUT OF THE USE OF THIS SOFTWARE, EVEN IF ADVISED OF THE POSSIBILITY OF
SUCH DAMAGE.
\end{verbatim}



\subsection{Floating point exception control}

The source for the \module{fpectl} module includes the following notice:

\begin{verbatim}
     ---------------------------------------------------------------------  
    /                       Copyright (c) 1996.                           \ 
   |          The Regents of the University of California.                 |
   |                        All rights reserved.                           |
   |                                                                       |
   |   Permission to use, copy, modify, and distribute this software for   |
   |   any purpose without fee is hereby granted, provided that this en-   |
   |   tire notice is included in all copies of any software which is or   |
   |   includes  a  copy  or  modification  of  this software and in all   |
   |   copies of the supporting documentation for such software.           |
   |                                                                       |
   |   This  work was produced at the University of California, Lawrence   |
   |   Livermore National Laboratory under  contract  no.  W-7405-ENG-48   |
   |   between  the  U.S.  Department  of  Energy and The Regents of the   |
   |   University of California for the operation of UC LLNL.              |
   |                                                                       |
   |                              DISCLAIMER                               |
   |                                                                       |
   |   This  software was prepared as an account of work sponsored by an   |
   |   agency of the United States Government. Neither the United States   |
   |   Government  nor the University of California nor any of their em-   |
   |   ployees, makes any warranty, express or implied, or  assumes  any   |
   |   liability  or  responsibility  for the accuracy, completeness, or   |
   |   usefulness of any information,  apparatus,  product,  or  process   |
   |   disclosed,   or  represents  that  its  use  would  not  infringe   |
   |   privately-owned rights. Reference herein to any specific  commer-   |
   |   cial  products,  process,  or  service  by trade name, trademark,   |
   |   manufacturer, or otherwise, does not  necessarily  constitute  or   |
   |   imply  its endorsement, recommendation, or favoring by the United   |
   |   States Government or the University of California. The views  and   |
   |   opinions  of authors expressed herein do not necessarily state or   |
   |   reflect those of the United States Government or  the  University   |
   |   of  California,  and shall not be used for advertising or product   |
    \  endorsement purposes.                                              / 
     ---------------------------------------------------------------------
\end{verbatim}



\subsection{MD5 message digest algorithm}

The source code for the \module{md5} module contains the following notice:

\begin{verbatim}
  Copyright (C) 1999, 2002 Aladdin Enterprises.  All rights reserved.

  This software is provided 'as-is', without any express or implied
  warranty.  In no event will the authors be held liable for any damages
  arising from the use of this software.

  Permission is granted to anyone to use this software for any purpose,
  including commercial applications, and to alter it and redistribute it
  freely, subject to the following restrictions:

  1. The origin of this software must not be misrepresented; you must not
     claim that you wrote the original software. If you use this software
     in a product, an acknowledgment in the product documentation would be
     appreciated but is not required.
  2. Altered source versions must be plainly marked as such, and must not be
     misrepresented as being the original software.
  3. This notice may not be removed or altered from any source distribution.

  L. Peter Deutsch
  ghost@aladdin.com

  Independent implementation of MD5 (RFC 1321).

  This code implements the MD5 Algorithm defined in RFC 1321, whose
  text is available at
	http://www.ietf.org/rfc/rfc1321.txt
  The code is derived from the text of the RFC, including the test suite
  (section A.5) but excluding the rest of Appendix A.  It does not include
  any code or documentation that is identified in the RFC as being
  copyrighted.

  The original and principal author of md5.h is L. Peter Deutsch
  <ghost@aladdin.com>.  Other authors are noted in the change history
  that follows (in reverse chronological order):

  2002-04-13 lpd Removed support for non-ANSI compilers; removed
	references to Ghostscript; clarified derivation from RFC 1321;
	now handles byte order either statically or dynamically.
  1999-11-04 lpd Edited comments slightly for automatic TOC extraction.
  1999-10-18 lpd Fixed typo in header comment (ansi2knr rather than md5);
	added conditionalization for C++ compilation from Martin
	Purschke <purschke@bnl.gov>.
  1999-05-03 lpd Original version.
\end{verbatim}



\subsection{Asynchronous socket services}

The \module{asynchat} and \module{asyncore} modules contain the
following notice:

\begin{verbatim}      
 Copyright 1996 by Sam Rushing

                         All Rights Reserved

 Permission to use, copy, modify, and distribute this software and
 its documentation for any purpose and without fee is hereby
 granted, provided that the above copyright notice appear in all
 copies and that both that copyright notice and this permission
 notice appear in supporting documentation, and that the name of Sam
 Rushing not be used in advertising or publicity pertaining to
 distribution of the software without specific, written prior
 permission.

 SAM RUSHING DISCLAIMS ALL WARRANTIES WITH REGARD TO THIS SOFTWARE,
 INCLUDING ALL IMPLIED WARRANTIES OF MERCHANTABILITY AND FITNESS, IN
 NO EVENT SHALL SAM RUSHING BE LIABLE FOR ANY SPECIAL, INDIRECT OR
 CONSEQUENTIAL DAMAGES OR ANY DAMAGES WHATSOEVER RESULTING FROM LOSS
 OF USE, DATA OR PROFITS, WHETHER IN AN ACTION OF CONTRACT,
 NEGLIGENCE OR OTHER TORTIOUS ACTION, ARISING OUT OF OR IN
 CONNECTION WITH THE USE OR PERFORMANCE OF THIS SOFTWARE.
\end{verbatim}


\subsection{Cookie management}

The \module{Cookie} module contains the following notice:

\begin{verbatim}
 Copyright 2000 by Timothy O'Malley <timo@alum.mit.edu>

                All Rights Reserved

 Permission to use, copy, modify, and distribute this software
 and its documentation for any purpose and without fee is hereby
 granted, provided that the above copyright notice appear in all
 copies and that both that copyright notice and this permission
 notice appear in supporting documentation, and that the name of
 Timothy O'Malley  not be used in advertising or publicity
 pertaining to distribution of the software without specific, written
 prior permission.

 Timothy O'Malley DISCLAIMS ALL WARRANTIES WITH REGARD TO THIS
 SOFTWARE, INCLUDING ALL IMPLIED WARRANTIES OF MERCHANTABILITY
 AND FITNESS, IN NO EVENT SHALL Timothy O'Malley BE LIABLE FOR
 ANY SPECIAL, INDIRECT OR CONSEQUENTIAL DAMAGES OR ANY DAMAGES
 WHATSOEVER RESULTING FROM LOSS OF USE, DATA OR PROFITS,
 WHETHER IN AN ACTION OF CONTRACT, NEGLIGENCE OR OTHER TORTIOUS
 ACTION, ARISING OUT OF OR IN CONNECTION WITH THE USE OR
 PERFORMANCE OF THIS SOFTWARE.
\end{verbatim}      



\subsection{Profiling}

The \module{profile} and \module{pstats} modules contain
the following notice:

\begin{verbatim}
 Copyright 1994, by InfoSeek Corporation, all rights reserved.
 Written by James Roskind

 Permission to use, copy, modify, and distribute this Python software
 and its associated documentation for any purpose (subject to the
 restriction in the following sentence) without fee is hereby granted,
 provided that the above copyright notice appears in all copies, and
 that both that copyright notice and this permission notice appear in
 supporting documentation, and that the name of InfoSeek not be used in
 advertising or publicity pertaining to distribution of the software
 without specific, written prior permission.  This permission is
 explicitly restricted to the copying and modification of the software
 to remain in Python, compiled Python, or other languages (such as C)
 wherein the modified or derived code is exclusively imported into a
 Python module.

 INFOSEEK CORPORATION DISCLAIMS ALL WARRANTIES WITH REGARD TO THIS
 SOFTWARE, INCLUDING ALL IMPLIED WARRANTIES OF MERCHANTABILITY AND
 FITNESS. IN NO EVENT SHALL INFOSEEK CORPORATION BE LIABLE FOR ANY
 SPECIAL, INDIRECT OR CONSEQUENTIAL DAMAGES OR ANY DAMAGES WHATSOEVER
 RESULTING FROM LOSS OF USE, DATA OR PROFITS, WHETHER IN AN ACTION OF
 CONTRACT, NEGLIGENCE OR OTHER TORTIOUS ACTION, ARISING OUT OF OR IN
 CONNECTION WITH THE USE OR PERFORMANCE OF THIS SOFTWARE.
\end{verbatim}



\subsection{Execution tracing}

The \module{trace} module contains the following notice:

\begin{verbatim}
 portions copyright 2001, Autonomous Zones Industries, Inc., all rights...
 err...  reserved and offered to the public under the terms of the
 Python 2.2 license.
 Author: Zooko O'Whielacronx
 http://zooko.com/
 mailto:zooko@zooko.com

 Copyright 2000, Mojam Media, Inc., all rights reserved.
 Author: Skip Montanaro

 Copyright 1999, Bioreason, Inc., all rights reserved.
 Author: Andrew Dalke

 Copyright 1995-1997, Automatrix, Inc., all rights reserved.
 Author: Skip Montanaro

 Copyright 1991-1995, Stichting Mathematisch Centrum, all rights reserved.


 Permission to use, copy, modify, and distribute this Python software and
 its associated documentation for any purpose without fee is hereby
 granted, provided that the above copyright notice appears in all copies,
 and that both that copyright notice and this permission notice appear in
 supporting documentation, and that the name of neither Automatrix,
 Bioreason or Mojam Media be used in advertising or publicity pertaining to
 distribution of the software without specific, written prior permission.
\end{verbatim} 



\subsection{UUencode and UUdecode functions}

The \module{uu} module contains the following notice:

\begin{verbatim}
 Copyright 1994 by Lance Ellinghouse
 Cathedral City, California Republic, United States of America.
                        All Rights Reserved
 Permission to use, copy, modify, and distribute this software and its
 documentation for any purpose and without fee is hereby granted,
 provided that the above copyright notice appear in all copies and that
 both that copyright notice and this permission notice appear in
 supporting documentation, and that the name of Lance Ellinghouse
 not be used in advertising or publicity pertaining to distribution
 of the software without specific, written prior permission.
 LANCE ELLINGHOUSE DISCLAIMS ALL WARRANTIES WITH REGARD TO
 THIS SOFTWARE, INCLUDING ALL IMPLIED WARRANTIES OF MERCHANTABILITY AND
 FITNESS, IN NO EVENT SHALL LANCE ELLINGHOUSE CENTRUM BE LIABLE
 FOR ANY SPECIAL, INDIRECT OR CONSEQUENTIAL DAMAGES OR ANY DAMAGES
 WHATSOEVER RESULTING FROM LOSS OF USE, DATA OR PROFITS, WHETHER IN AN
 ACTION OF CONTRACT, NEGLIGENCE OR OTHER TORTIOUS ACTION, ARISING OUT
 OF OR IN CONNECTION WITH THE USE OR PERFORMANCE OF THIS SOFTWARE.

 Modified by Jack Jansen, CWI, July 1995:
 - Use binascii module to do the actual line-by-line conversion
   between ascii and binary. This results in a 1000-fold speedup. The C
   version is still 5 times faster, though.
 - Arguments more compliant with python standard
\end{verbatim}



\subsection{XML Remote Procedure Calls}

The \module{xmlrpclib} module contains the following notice:

\begin{verbatim}
     The XML-RPC client interface is

 Copyright (c) 1999-2002 by Secret Labs AB
 Copyright (c) 1999-2002 by Fredrik Lundh

 By obtaining, using, and/or copying this software and/or its
 associated documentation, you agree that you have read, understood,
 and will comply with the following terms and conditions:

 Permission to use, copy, modify, and distribute this software and
 its associated documentation for any purpose and without fee is
 hereby granted, provided that the above copyright notice appears in
 all copies, and that both that copyright notice and this permission
 notice appear in supporting documentation, and that the name of
 Secret Labs AB or the author not be used in advertising or publicity
 pertaining to distribution of the software without specific, written
 prior permission.

 SECRET LABS AB AND THE AUTHOR DISCLAIMS ALL WARRANTIES WITH REGARD
 TO THIS SOFTWARE, INCLUDING ALL IMPLIED WARRANTIES OF MERCHANT-
 ABILITY AND FITNESS.  IN NO EVENT SHALL SECRET LABS AB OR THE AUTHOR
 BE LIABLE FOR ANY SPECIAL, INDIRECT OR CONSEQUENTIAL DAMAGES OR ANY
 DAMAGES WHATSOEVER RESULTING FROM LOSS OF USE, DATA OR PROFITS,
 WHETHER IN AN ACTION OF CONTRACT, NEGLIGENCE OR OTHER TORTIOUS
 ACTION, ARISING OUT OF OR IN CONNECTION WITH THE USE OR PERFORMANCE
 OF THIS SOFTWARE.
\end{verbatim}


\end{document}
